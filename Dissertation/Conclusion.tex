% -*- root: Dissertation.tex -*-
\documentclass[Dissertation.tex]{subfiles}
\begin{document}
\chapter{Conclusions and Future Directions}

The goal of this work has been to develop a proof of concept for a space-time discontinuous Petrov-Galerkin
finite element method with applications to fluid flow applications.
Chapter~\ref{sec:introduction} provided motivations for applying DPG to transient fluid problems and
explored some of the alternatives in the field.
Local conservation is an important property to computational fluid dynamics practitioners.
In Chapter~\ref{sec:conservation} we developed a variant of DPG that is locally conservative through
the addition of Lagrange multipliers to the system.
This locally conservative DPG method was proved to be stable and robust 
and shown to dramatically improve coarse mesh numerical results on several test problems.

In Chapter~\ref{sec:robust} we develop a theory for space-time DPG applied to convection-diffusion
type problems.
We use an ultra-weak formulation where the conservation equation is placed in a space-time
divergence form. 
This allows us to define physically meaningful fluxes related to conservation principles and eases the transition to Navier-Stokes.
We propose new test norms for space-time convection-diffusion and prove that they provide
near optimal convergence of the primary variable.
Numerical results confirm the theory, showing that the energy norm in which the solution is optimal robustly bounds the $L^2$ norm.

Chapter~\ref{sec:incompressible} and Chapter~\ref{sec:compressible} develop space-time DPG methods 
for transient incompressible and compressible Navier-Stokes
by drawing analogies to transient convection-diffusion. 
This includes the analogous ultra-weak formulations in space-time divergence form
and robust test norms.
Numerical verifications of the theory show the expected behavior.

Several side projects are explored in the appendices. 
An implicit Runge-Kutta time stepping strategy for DPG is described in Appendix~\ref{sec:timestepping}
and shown to converge at the expected rates.
We developed space-time DPG implementations of compressible Navier-Stokes under three popular
variable transformations in Appendix~\ref{sec:VariableComparison}.
Physically meaningful test norms for compressible Navier-Stokes inspired by entropy
were then proposed in Appendix~\ref{sec:EntropyNorm}, though numerical experiments 
seemed to prefer the standard non-entropy scaled test norms.

\section{Accomplishments}
On the theoretical side, I have developed and proven robustness of both locally conservative and space-time
discontinuous Petrov-Galerkin finite element methods for convection-diffusion problems.
This included the development of robust test norms for both of these formulations.
I also used the concept of entropy to derive new test norms for compressible Navier-Stokes
such that the residual is minimized in a physically consistent way.

On the numerical and computational side, I confirmed numerically the robustness of my test norms for
locally conservative and space-time DPG.
I also demonstrated convergence of space-time DPG for incompressible Navier-Stokes and 
obtained various shock tube results for compressible Navier-Stokes.
I implemented space-time DPG for various variable transformations of the compressible Navier-Stokes
equations and compared the numerical results.
Within the primitive variable formulation, I implemented entropy scaled test norms and 
compared to the standard test norms inspired by convection-diffusion.
I also implemented an ESDIRK (explicit first step singly diagonal implicit Runge-Kutta) time stepping
strategy for DPG.
Finally, I've been an active contributor to the parallel $hp$-adaptive DPG code base Camellia\cite{CamelliaDPG}
from which all of these results were generated.

This dissertation includes applications of both locally conservative and space-time DPG to problems
in convection-diffusion, Burgers' equation, Stokes flow, incompressible Navier-Stokes,
and compressible Navier-Stokes.
Of particular note are simulations of Stokes flow over a cylinder and a backward facing step,
incompressible Navier-Stokes simulations of Taylor-Green vortices, and several shock tube simulations
of compressible Navier-Stokes including a problem with a moving boundary.

\section{Future Work}
This work was really a proof of concept and much work remains in order to make this a competitive numerical method
for transient fluid flow problems.
% As is common with any research, much work remains to be done.

\subsection{Improve Scaling}
The most pressing issue before pursuing further work on space-time DPG is to improve the scaling
of our global solve.
Past explorations of DPG were primarily focused on two dimensional solves.
In space-time this two spatial dimensions requires a full 3D solve.
Much to our chagrin after working to implement a 3D adaptive code, we discovered that 
our global solvers did not scale nearly as well as we expected on these higher dimensional problems.
We further explore this issue and some possible solutions in Appendix~\ref{sec:Scaling}.

\subsection{Shock Capturing}
The strength of DPG lie in its stability and adaptivity properties. 
Shock capturing for the Euler and compressible Navier-Stokes equations has more to do with 
limiting Gibbs phenomenon of overshoots and undershoots around shocks on meshes coarser than
the viscous length scale.
As such, if we were serious about applying DPG to shock problems, we would want to augment it
with some sort of shock capturing strategy, preferably a consistent one that reduces to the original
equations in the limit as we fully resolve solution features.

Another possible solution that we've begun exploring is the development of DPG for non-Hilbert $L^p$
Banach spaces. 
Gibbs phenomenon is well known to be less pronounced in $L^1$ spaces than the $L^2$ spaces at
the foundation of most finite element theory.
The downside to this approach is that any finite element theory built around Hilbert spaces is no longer applicable
and previously linear problems like convection-diffusion become nonlinear in non-Hilbert spaces.

\subsection{More Extensive 2D Results}
With the implementation of the previous two topics, we open the door to many more interesting 2D 
transient problems.
The issue of scaling prevented us from producing meaningful results for unsteady incompressible
flow over a cylinder as we originally planned.
This would also allow us to consider classical problems like vortex shedding off of an oscillating
airfoil.
It would be worthwhile to see if our lack of wall heating on the 1D Noh problem carries over to the 
2D case as well. 
In the current state of things, undershoots around shocks cause the density to dip negative 
which causes the equations to be ill posed for the next Newton iterate. 
If we perform a line search on the Newton update to keep density positive, the line 
search drops below $10^{-6}$, effectively stalling the Newton iteration.
We believe that shock capturing could regularize the solution and allow us to converge to a solution.

\subsection{Anisotropic Refinements}
Anisotropic refinements in space-time are a necessary first step in order to make
time slabs a more attractive option, a point that is illustrated in Appendix~\ref{sec:Scaling}.
Jesse Chan developed an anisotropic refinement strategy for 2D computations in Camellia, 
but this process gets significantly more difficult in 3D or higher space-time meshes.

\subsection{3D Results}
We've implemented space-time as a tensor product of a spatial mesh and a temporal line. 
In theory this means that 3D space-time shouldn't be significantly more complicated to implement,
but we expect the costs to blow up even more than they did from 2D to 3D, as we would now be performing
4D global solves.
Additionally, the mesh partitioning libraries we leverage to distribute elements across processors are 
not set up to handle 4D meshes.
The pursuit of 3D problems would force us to fundamentally rethink how we implement space-time DPG.

\end{document}
