%%%%%%%%%%%%%%%%%%%%%%%%%%%%%%%%%%%%%%%%%%%%%%%%%%%%%%%%%%%%%%%
%%  Dissertation.tex, to be compiled with latex.
%%  08 April 2002	Version 4
%%%%%%%%%%%%%%%%%%%%%%%%%%%%%%%%%%%%%%%%%%%%%%%%%%%%%%%%%%%%%%%
%%
%%  Writing a Doctoral Dissertation with LaTeX at
%%	the University of Texas at Austin
%%
%%  (Modify this ``template'' for your own dissertation.)
%%
%%%%%%%%%%%%%%%%%%%%%%%%%%%%%%%%%%%%%%%%%%%%%%%%%%%%%%%%%%%%%%%%


\documentclass[12pt]{report} % The documentclass must be ``report''.
% -----PACKAGES
%\usepackage[shortend,titlenumbered]{algorithm2e}
%\usepackage{algorithmic}
%\usepackage[plain]{algorithm}
\usepackage{multicol}
\usepackage{color}
\usepackage{multirow}
\usepackage{fancybox}
%\usepackage{index}
\usepackage{varioref}
\usepackage{psfrag}
\usepackage{epsfig}
\usepackage{boxedminipage}
\usepackage{graphicx}
\usepackage{rotating}
\usepackage{amsmath}
\usepackage{amssymb}
%\usepackage{amsfont}
\usepackage{latexsym}
\usepackage{alltt}
%\usepackage[small,bf]{caption}
\usepackage{url}
%\usepackage{citesort}
%\usepackage{crop}
\usepackage{array}
\usepackage{subfigure}
\usepackage{dcolumn}

% -----SETLENGTH
%\setlength{\captionmargin}{20pt} 

% -----NEWCOMMANDS
\newcommand{\nc}{\newcommand}
\nc{\mathsm}[1]{\text{\small{$#1$}}}
\nc{\ubar}[1]{\underset{-}{#1}}
\nc{\optype}{\textrm}
\nc{\EQ}[1]{(\ref{eq:#1})}
\nc{\TAB}[1]{\ref{tab:#1}}
\nc{\FIG}[1]{\ref{fig:#1}}
\nc{\SEC}[1]{\ref{sec:#1}}
\nc{\ALG}[1]{\ref{alg:#1}}
\nc{\CHAP}[1]{\ref{chap:#1}}
\nc{\mtrx}[1]{\boldsymbol{\mathbf{#1}}}
\nc{\vctr}[1]{\boldsymbol{\mathbf{#1}}}
\nc{\grad}{\mbox{\boldmath$\nabla$}}
\nc{\gradient}{\textsl{grad}\,}
\nc{\hessian}{\textsl{grad\,}^2}
\nc{\ii}{\iota}
\nc{\dd}{d}
\nc{\ee}{\mathrm{e}}
\nc{\pdiv}[2]{\partial{#1}/\partial{#2}}
\nc{\dpdiv}[2]{\displaystyle{\frac{\partial{#1}}{\partial{#2}}}}
\nc{\ddiv}[2]{\displaystyle{\frac{\dd{#1}}{\dd{#2}}}}
\nc{\inpr}{\hspace{-1pt}\cdot\hspace{-1pt}}
\nc{\IR}{\mathbb{R}}
\nc{\IN}{\mathbb{N}}
\nc{\IZ}{\mathbb{Z}}
\nc{\IC}{\mathbb{C}}
\nc{\half}{\frac{1}{2}}
\nc{\shalf}{\scriptstyle{\half}} 
\nc{\ds}[1]{\displaystyle{#1}}
\nc{\ts}[1]{\textstyle{#1}}
\nc{\sign}{\optype{sign}}
\nc{\spr}{\optype{spr}}
\nc{\dist}{\optype{dist}}
\nc{\rank}{\optype{rank}}
\nc{\codim}{\optype{codim}}
\nc{\supp}{\optype{supp}}
\nc{\diag}{\optype{diag}}
\nc{\meas}{\optype{meas}}
\nc{\cond}{\optype{cond}}
\nc{\kernel}{\optype{kernel}}
\nc{\spa}{\optype{span}}
\nc{\order}{\mathcal{O}}
\nc{\Fr}{\mathrm{Fr}}
\nc{\Rey}{\mathrm{Re}}
\nc{\Ord}{O}
\nc{\ord}{o}
\nc{\st}{\:{:}\:}
\nc{\closure}[1]{\overline{#1}}
\nc{\emin}[1]{\emph{#1}\index{#1}\/}
\nc{\rmin}[1]{#1\index{{}@{#1}}}
\nc{\Laplace}{\Delta}
\nc{\ie}{i.e.}
\nc{\eg}{e.g.}
%\nc{\union}{\cup}
\nc{\Union}{\bigcup}
\nc{\lf}[1]{\mathsf{#1}}
\nc{\dbar}[1]{\bar{\bar{#1}}}
\nc{\ul}[1]{\underline{#1}}
\nc{\hpt}{\hspace{0.5pt}}
\nc{\E}[1]{\times{}10^{#1}}
\nc{\inp}[2]{\langle{#1},{#2}\rangle}
\nc{\tmpcommand}{}

% -----RENEWCOMMANDS
\renewcommand{\baselinestretch}{1}
\renewcommand{\exp}{\optype{exp}\,}
\renewcommand{\cosh}{\optype{cosh}\,}
\renewcommand{\tanh}{\optype{tanh}\,}
\renewcommand{\sinh}{\optype{sinh}\,}
\renewcommand{\div}[1]{\optype{div}\,{#1}}
\renewcommand{\half}{\mbox{$\frac{1}{2}$}}
%\renewcommand{\descriptionlabel}[1]{\hspace{\labelsep}\emph{#1}}

% -----ETC
\raggedbottom


\DeclareMathOperator{\curl}{\bf curl}
\DeclareMathOperator{\rot}{\rm curl}
\DeclareMathOperator{\divv}{\rm div}
\newcommand{\tro}{\gamma_0}
\newcommand{\trt}{\gamma_{\sft}}
\newcommand{\trn}{\gamma_{\sfn}}

\newcommand{\PT}{{\partial T}}
\newcommand{\bbN}{{\mathbb{N}}}
\newcommand{\bbP}{{\mathbb{P}}}

\newcommand{\scC}{{\mathscr{C}}}
\newcommand{\caD}{{\mathcal{D}}}
\newcommand{\caL}{{\mathcal{L}}}

\newcommand{\sfe}{{\mathsf{e}}}
\newcommand{\sff}{{\mathsf{f}}}
\newcommand{\sft}{{\boldsymbol{\mathsf{t}}}}
\newcommand{\sfn}{{\boldsymbol{\mathsf{n}}}}

%   Common caligraphic abbrevs
\newcommand{\BB}{\mathcal{B}}
\newcommand{\CC}{\mathcal{C}}
\newcommand{\DD}{\mathcal{D}}
\newcommand{\EE}{\mathcal{E}}
\newcommand{\FF}{\mathcal{F}}
\newcommand{\GG}{\mathcal{G}}
\newcommand{\II}{\mathcal{I}}
\newcommand{\JJ}{\mathcal{J}}
\newcommand{\KK}{\mathcal{K}}
\newcommand{\LL}{\mathcal{L}}
\newcommand{\OO}{\mathcal{O}}
\newcommand{\QQ}{\mathcal{Q}}
\newcommand{\RR}{\mathcal{R}}
\newcommand{\TT}{\mathcal{T}}


 %% JAY'S PREAMBLE
 %%========================

%   Math symbol definitions
\def\d{\partial}
%\newsymbol\lee 132E
\newcommand{\union}{\mathop{\bigcup}}
\newcommand{\intersect}{\mathop{\bigcap}}
\newcommand{\binomial}[2]{\ensuremath{
		\begin{pmatrix}{#1}\\{#2}\end{pmatrix}}}
\newcommand{\smallbinomial}[2]{\ensuremath{
		(\begin{smallmatrix}{#1}\\{#2}\end{smallmatrix})}}
\newcommand{\tang}[1]{\ensuremath{{#1}_{\intercal}}} % can use \top
						     % also
\newcommand{\hypergeom}[2]{\ensuremath{\sideset{_{#1}}{_{#2}}{\mathop{F}}}}
%   Difficult names
\newcommand{\Babuska}{Babu{\v{s}}ka}       % Remember: Usage is \Babuska\
\newcommand{\Cea}{C{\'e}a}                 % with trailing `\' to give space
\newcommand{\Poincare}{Poincar{\'{e}}}     % when needed, but when ending
\newcommand{\Nedelec}{N{\'{e}}d{\'{e}}lec} % sentence use \Babuska.
\newcommand{\Frechet}{Fr{\'{e}}chet}
\newcommand{\Muller}{M{\"u}ller}
\newcommand{\LHospital}{L'H{\^{o}}spital}
%   Bold and beautiful
\newcommand{\ba}{{\boldsymbol{a}}}
\newcommand{\bA}{\boldsymbol{A}}
\newcommand{\balpha}{{\boldsymbol{\alpha}}}
\newcommand{\bB}{{\boldsymbol{B}}}
\newcommand{\bb}{{\boldsymbol{b}}}
\newcommand{\bbeta}{{\boldsymbol{\beta}}}
\newcommand{\etab}{{\boldsymbol{\eta}}}
\newcommand{\bC}{{\boldsymbol{C}}}
\newcommand{\bc}{{\boldsymbol{c}}}
\newcommand{\bD}{{\boldsymbol{D}}}
\newcommand{\bd}{{\boldsymbol{d}}}
\newcommand{\db}{{\boldsymbol{\d}}}
\newcommand{\bdelta}{{\boldsymbol{\delta}}}
\newcommand{\bDelta}{{\boldsymbol{\Delta}}}
\newcommand{\beps}{{\boldsymbol{\varepsilon}}}
\newcommand{\be}{{\boldsymbol{e}}}
\newcommand{\bg}{{\boldsymbol{g}}}
\newcommand{\bm}{{\boldsymbol{m}}}
\newcommand{\bn}{{\boldsymbol{n}}}
\newcommand{\bN}{{\boldsymbol{N}}}
\newcommand{\bp}{{\boldsymbol{p}}}
\newcommand{\bpsi}{{\boldsymbol{\psi}}}
\newcommand{\bq}{{\boldsymbol{q}}}
\newcommand{\bxi}{{\boldsymbol{\xi}}}
\newcommand{\bE}{{\boldsymbol{E}}}
\newcommand{\bF}{{\boldsymbol{F}}}
\newcommand{\bh}{{\boldsymbol{h}}}
\newcommand{\bH}{{\boldsymbol{H}}}
\newcommand{\bI}{{\boldsymbol{I}}}
\newcommand{\bj}{{\boldsymbol{j}}}
\newcommand{\bJ}{{\boldsymbol{J}}}
\newcommand{\bK}{{\boldsymbol{K}}}
\newcommand{\bk}{{\boldsymbol{k}}}
\newcommand{\bll}{{\boldsymbol{\ell}}}
\newcommand{\bL}{{\boldsymbol{L}}}
\newcommand{\blambda}{{\boldsymbol{\lambda}}}
\newcommand{\bmu}{{\boldsymbol{\mu}}}
\newcommand{\bM}{{\boldsymbol{M}}}
\newcommand{\bomega}{{\boldsymbol{\omega}}}
\newcommand{\bP}{{\boldsymbol{P}}}
\newcommand{\bphi}{{\boldsymbol{\phi}}}
\newcommand{\bQ}{{\boldsymbol{Q}}}
\newcommand{\bG}{{\boldsymbol{G}}}
\newcommand{\bu}{{\boldsymbol{u}}}
\newcommand{\bU}{{\boldsymbol{U}}}
\newcommand{\bV}{{\boldsymbol{V}}}
\newcommand{\bX}{{\boldsymbol{X}}}
\newcommand{\bv}{{\boldsymbol{v}}}
\newcommand{\bw}{{\boldsymbol{w}}}
\newcommand{\bW}{{\boldsymbol{W}}}
\newcommand{\bR}{{\boldsymbol{R}}}
\newcommand{\br}{{\boldsymbol{r}}}
\newcommand{\bS}{{\boldsymbol{S}}}
\newcommand{\bT}{{\boldsymbol{T}}}
\newcommand{\btau}{{\boldsymbol{\tau}}}
\newcommand{\bt}{{\boldsymbol{t}}}
\newcommand{\bx}{{\boldsymbol{x}}}
\newcommand{\by}{{\boldsymbol{y}}}
\newcommand{\bz}{{\boldsymbol{z}}}
\newcommand{\bzero}{{\boldsymbol{0}}}
\newcommand{\bZ}{{\boldsymbol{Z}}}
%   Common scalar fields
\newcommand{\RRR}{\mathbb{R}}
\newcommand{\CCC}{\mathbb{C}}
\newcommand{\ZZZ}{\mathbb{Z}}
\newcommand{\NNN}{\mathbb{N}}
%   Differential operators
\newcommand{\dive}{\mathop\mathrm{div}}
%\newcommand{\grad}{\ensuremath{\mathop{{\bf{grad}}}}}
%\newcommand{\curl}{{\ensuremath\mathop{\mathbf{curl}\,}}}
\newcommand{\Curl}{ {\bf Curl}}
\newcommand{\dx}{\ensuremath{\mathrm{d}x}}
\newcommand{\dy}{\ensuremath{\mathrm{d}y}}
\newcommand{\dr}{\ensuremath{\mathrm{d}r}}
\newcommand{\dR}{\ensuremath{\mathrm{d}R}}
\newcommand{\drho}{\ensuremath{\mathrm{d}\rho}}
\newcommand{\dz}{\ensuremath{\mathrm{d}z}}
\newcommand{\dzeta}{\ensuremath{\mathrm{d}\zeta}}
%   Wordy math symbols
\newcommand{\card}{\ensuremath{\mathop\mathrm{card}}}
%\newcommand{\diag}{\ensuremath{\mathop\mathrm{diag}}}
\newcommand{\diam}{\ensuremath{\mathop\mathrm{diam}}}
%\newcommand{\dist}{\mathop\mathrm{dist}}
\newcommand{\Ker}{\mathop\mathrm{Ker}}
\newcommand{\Range}{\mathop\mathrm{Range}}
%\newcommand{\rank}{\mathop\mathrm{rank}}
%\newcommand{\meas}{\mathop\mathrm{meas}}
\newcommand{\Forall}{\quad\text{for all }}
%\newcommand{\supp}{\mathop\mathrm{supp}}
\newcommand{\Span}{\mathop\mathrm{Span}}
\newcommand{\Hdiv}[1]{\bH(\dive,#1)}
%\newcommand{\Hcurl}[1]{\bH(\curl,#1)}
%   Common caligraphic abbrevs
%\newcommand{\BB}{\mathcal{B}}
%\newcommand{\CC}{\mathcal{C}}
%\newcommand{\DD}{\mathcal{D}}
%\newcommand{\EE}{\mathcal{E}}
%\newcommand{\FF}{\mathcal{F}}
%\newcommand{\GG}{\mathcal{G}}
%\newcommand{\II}{\mathcal{I}}
%\newcommand{\JJ}{\mathcal{J}}
%\newcommand{\KK}{\mathcal{K}}
%\newcommand{\LL}{\mathcal{L}}
%\newcommand{\OO}{\mathcal{O}}
%\newcommand{\QQ}{\mathcal{Q}}
%\newcommand{\RR}{\mathcal{R}}
%\newcommand{\TT}{\mathcal{T}}
%   Variations on standard symbols
\newcommand{\veps}{\varepsilon}
\newcommand{\vlam}{\varLambda}
\newcommand{\vpi}{\varPi}
\newcommand{\vPi}{\boldsymbol{\varPi}}
\newcommand{\vsig}{\varSigma}
\newcommand{\vbt}{\boldsymbol{\varTheta}}
\newcommand{\vPsi}{\boldsymbol{\varPsi}}
%\newcommand{\ii}{\hat{\imath}}
%   Innerproducts, norms, etc
\newcommand{\ntrip}[1]{|\!|\!| {#1} |\!|\!|}
\newcommand{\ip}[1]{\langle {#1} \rangle}
%   Utilities
\newcommand{\blnk}{\underline{\hspace{3cm}}\;}
\newcommand{\marg}[1]{\marginpar{\tiny{\framebox{\parbox{1.7cm}{#1}}}}}
\newcommand{\degreeC}[1]{\ensuremath{{#1\,}^\circ\!\text{C}}}
                        % try also  \textcelsius of textcomp package
%   Trademarked names \texttrademark, \textregistered
\newcommand{\matlab}{MATLAB\textregistered\renewcommand{\matlab}{MATLAB}}
\newcommand{\femlab}{FEMLAB\textregistered\renewcommand{\femlab}{FEMLAB}}

%   Style preferences
\renewcommand{\thefootnote}{\fnsymbol{footnote}} % Use symbols instead of
						 % numbers for footnotes
						 

\newcommand{\Eg}{\EE^\mathrm{grad}}
\newcommand{\Ec}{\boldsymbol{\EE}^\mathrm{curl}}
\newcommand{\Ed}{\boldsymbol{\EE}^\mathrm{div}}


\newcommand{\bfdu}{\mbox{\boldmath $\delta u$}}
\newcommand{\bfdv}{\mbox{\boldmath $\delta v$}}
\newcommand{\du}{{\delta u}}
\newcommand{\dv}{{\delta v}}
\newcommand{\bfnabt}{\widetilde{\bfnab}}
\newcommand{\bfepst}{\widetilde{\bfeps}}

\usepackage{subfiles}

\usepackage{utdiss2}  		 % Dissertation package style file.


%%%%%%%%%%%%%%%%%%%%%%%%%%%%%%%%%%%%%%%%%%%%%%%%%%%%%%%%%%%%%%%%%%%%
% Optional packages used for this sample dissertation. If you don't
% need a capability in your dissertation, feel free to comment out
% the package usage command.
%%%%%%%%%%%%%%%%%%%%%%%%%%%%%%%%%%%%%%%%%%%%%%%%%%%%%%%%%%%%%%%%%%%%

\usepackage{amsmath,amsthm,amsfonts,amscd} % Some packages to write mathematics.
% \usepackage{eucal} 	 	% Euler fonts
\usepackage{verbatim}      	% Allows quoting source with commands.
\usepackage{makeidx}       	% Package to make an index.
\usepackage{psfig}         	% Allows inclusion of eps files.
\usepackage{epsfig}         % Allows inclusion of eps files.
\usepackage{citesort}
\usepackage{url}		    % Allows good typesetting of web URLs.
\usepackage{longtable}
\usepackage{framed}
\usepackage{color}
% \usepackage[usenames,dvipsnames]{xcolor}
\usepackage{stmaryrd}
\usepackage{epigraph}
\setlength\epigraphwidth{10cm}
\setlength\epigraphrule{0pt}
\usepackage{etoolbox}
\usepackage{rotating}

% \SetSymbolFont{stmry}{bold}{U}{stmry}{m}{n}
% \SetSymbolFont{stmry}{bold}{U}{stmry}{b}{n}

\makeatletter
\patchcmd{\epigraph}{\@epitext{#1}}{\itshape\@epitext{#1}}{}{}
\makeatother

%\usepackage{draftcopy}
% Uncomment this line to have the
% word, "DRAFT," as a background
% "watermark" on all of the pages of
% of your draft versions. When ready
% to generate your final copy, re-comment
% it out with a percent sign to remove
% the word draft before you re-run
% Makediss for the last time.

\newtheorem{theorem}{Theorem}[section]
\newtheorem{proposition}{Proposition}[section]
\newtheorem{lemma}{Lemma}[section]
\newtheorem{corollary}{Corollary}[section]
\newtheorem{remark}{Remark}[section]
\newtheorem{definition}{Definition}[section]



\author{Truman Ellis}  	% Required

\address{504 Nelray Blvd\\ Austin, Texas 78751}  % Required

\title{Space-Time Discontinuous Petrov-Galerkin Finite Elements for Transient Fluid Mechanics}
                                                    % Required

%%%%%%%%%%%%%%%%%%%%%%%%%%%%%%%%%%%%%%%%%%%%%%%%%%%%%%%%%%%%%%%%%%%%%%
% NOTICE: The total number of supervisors and other members %%%%%%%%%%
%%%%%%%%%%%%%%% MUST be seven (7) or less! If you put in more, %%%%%%%
%%%%%%%%%%%%%%% they are put on the page after the Committee %%%%%%%%%
%%%%%%%%%%%%%%% Certification of Approved Version page. %%%%%%%%%%%%%%
%%%%%%%%%%%%%%%%%%%%%%%%%%%%%%%%%%%%%%%%%%%%%%%%%%%%%%%%%%%%%%%%%%%%%%

%%%%%%%%%%%%%%%%%%%%%%%%%%%%%%%%%%%%%%%%%%%%%%%%%%%%%%%%%%%%%%%%%%%%%%
%
% Enter names of the supervisor and co-supervisor(s), if any,
% of your dissertation committee. Put one name per line with
% the name in square brackets. The name on the last line, however,
% must be in curly braces.
%
% If you have only one supervisor, the entry below will read:
%
%	\supervisor
%		{Supervisor's Name}
%
% NOTE: Maximum three supervisors. Minimum one supervisor.
% NOTE: The Office of Graduate Studies will accept only two supervisors!
%
%
\supervisor
	[Leszek F. Demkowicz]
	{Robert D. Moser}

%%%%%%%%%%%%%%%%%%%%%%%%%%%%%%%%%%%%%%%%%%%%%%%%%%%%%%%%%%%%%%%%%%%%%%
%
% Enter names of the other (non-supervisor) members(s) of your
% dissertation committee. Put one name per line with the name
% in square brackets. The name on the last line, however, must
% be in curly braces.
%
% NOTE: Maximum six other members. Minimum zero other members.
% NOTE: The Office of Graduate Studies may restrict you to a total
%	of six committee members.
%
%
\committeemembers
	[Thomas J.R. Hughes]
	[Clint N. Dawson]
	{Jayathi Y. Murthy}

%%%%%%%%%%%%%%%%%%%%%%%%%%%%%%%%%%%%%%%%%%%%%%%%%%%%%%%%%%%%%%%%%%%%%%

\previousdegrees{B.S., M.S.}
     % The abbreviated form of your previous degree(s).
     % E.g., \previousdegrees{B.S., MBA}.
     %
     % The default value is `B.S., M.S.'

\graduationmonth{May}
     % Graduation month, either May, August, or December, in the form
     % as `\graduationmonth{May}'. Do not abbreviate.
     %
     % The default value (either May, August, or December) is guessed
     % according to the time of running LaTeX.

\graduationyear{2016}
     % Graduation year, in the form as `\graduationyear{2001}'.
     % Use a 4 digit (not a 2 digit) number.
     %
     % The default value is guessed according to the time of
     % running LaTeX.

\typist{the author}
     % The name(s) of typist(s), put `the author' if you do it yourself.
     % E.g., `\typist{Maryann Hersey and the author}'.
     %
     % The default value is `the author'.


%%%%%%%%%%%%%%%%%%%%%%%%%%%%%%%%%%%%%%%%%%%%%%%%%%%%%%%%%%%%%%%%%%%%%%
% Commands for master's theses and reports.			     %
%%%%%%%%%%%%%%%%%%%%%%%%%%%%%%%%%%%%%%%%%%%%%%%%%%%%%%%%%%%%%%%%%%%%%%
%
% If the degree you're seeking is NOT Doctor of Philosophy, uncomment
% (remove the % in front of) the following two command lines (the ones
% that have the \ as their second character).
%
%\degree{MASTER OF ARTS}
%\degreeabbr{M.A.}

% Uncomment the line below that corresponds to the type of master's
% document you are writing.
%
%\masterreport
%\masterthesis


%%%%%%%%%%%%%%%%%%%%%%%%%%%%%%%%%%%%%%%%%%%%%%%%%%%%%%%%%%%%%%%%%%%%%%
% Some optional commands to change the document's defaults.	     %
%%%%%%%%%%%%%%%%%%%%%%%%%%%%%%%%%%%%%%%%%%%%%%%%%%%%%%%%%%%%%%%%%%%%%%
%
%\singlespacing
%\oneandonehalfspacing

%\singlespacequote
\oneandonehalfspacequote

\topmargin 0.125in	% Adjust this value if the PostScript file output
			% of your dissertation has incorrect top and
			% bottom margins. Print a copy of at least one
			% full page of your dissertation (not the first
			% page of a chapter) and measure the top and
			% bottom margins with a ruler. You must have
			% a top margin of 1.5" and a bottom margin of
			% at least 1.25". The page numbers must be at
			% least 1.00" from the bottom of the page.
			% If the margins are not correct, adjust this
			% value accordingly and re-compile and print again.
			%
			% The default value is 0.125"

		% If you want to adjust other margins, they are in the
		% utdiss2-nn.sty file near the top. If you are using
		% the shell script Makediss on a Unix/Linux system, make
		% your changes in the utdiss2-nn.sty file instead of
		% utdiss2.sty because Makediss will overwrite any changes
		% made to utdiss2.sty.

%%%%%%%%%%%%%%%%%%%%%%%%%%%%%%%%%%%%%%%%%%%%%%%%%%%%%%%%%%%%%%%%%%%%%%
% Some optional commands to be tested.				     %
%%%%%%%%%%%%%%%%%%%%%%%%%%%%%%%%%%%%%%%%%%%%%%%%%%%%%%%%%%%%%%%%%%%%%%

% If there are 10 or more sections, 10 or more subsections for a section,
% etc., you need to make an adjustment to the Table of Contents with the
% command \longtocentry.
%
%\longtocentry



%%%%%%%%%%%%%%%%%%%%%%%%%%%%%%%%%%%%%%%%%%%%%%%%%%%%%%%%%%%%%%%%%%%%%%
%	Some math support.					     %
%%%%%%%%%%%%%%%%%%%%%%%%%%%%%%%%%%%%%%%%%%%%%%%%%%%%%%%%%%%%%%%%%%%%%%
%
%	Theorem environments (these need the amsthm package)
%
%% \theoremstyle{plain} %% This is the default

%\newtheorem{thm}{Theorem}[section]
%\newtheorem{cor}[thm]{Corollary}
%\newtheorem{lem}[thm]{Lemma}
%\newtheorem{prop}[thm]{Proposition}
%\newtheorem{ax}{Axiom}
%
%\theoremstyle{definition}
%\newtheorem{defn}{Definition}[section]
%
%\theoremstyle{remark}
%\newtheorem{rem}{Remark}[section]
%\newtheorem*{notation}{Notation}

%\numberwithin{equation}{section}


%%%%%%%%%%%%%%%%%%%%%%%%%%%%%%%%%%%%%%%%%%%%%%%%%%%%%%%%%%%%%%%%%%%%%%
%	Macros.							     %
%%%%%%%%%%%%%%%%%%%%%%%%%%%%%%%%%%%%%%%%%%%%%%%%%%%%%%%%%%%%%%%%%%%%%%
%
%	Here some macros that are needed in this document:


\newcommand{\latexe}{{\LaTeX\kern.125em2%
                      \lower.5ex\hbox{$\varepsilon$}}}

\newcommand{\amslatex}{\AmS-\LaTeX{}}

\chardef\bslash=`\\	% \bslash makes a backslash (in tt fonts)
			%	p. 424, TeXbook

\newcommand{\cn}[1]{\texttt{\bslash #1}}

\makeatletter		% Starts section where @ is considered a letter
			% and thus may be used in commands.
\def\square{\RIfM@\bgroup\else$\bgroup\aftergroup$\fi
  \vcenter{\hrule\hbox{\vrule\@height.6em\kern.6em\vrule}%
                                              \hrule}\egroup}
\makeatother		% Ends sections where @ is considered a letter.
			% Now @ cannot be used in commands.

\makeindex    % Make the index

%%%%%%%%%%%%%%%%%%%%%%%%%%%%%%%%%%%%%%%%%%%%%%%%%%%%%%%%%%%%%%%%%%%%%%
%		The document starts here.			     %
%%%%%%%%%%%%%%%%%%%%%%%%%%%%%%%%%%%%%%%%%%%%%%%%%%%%%%%%%%%%%%%%%%%%%%

\begin{document}

\copyrightpage          % Produces the copyright page.


%
% NOTE: In a doctoral dissertation, the Committee Certification page
%		(with signatures) is BEFORE the Title page.
%	In a masters thesis or report, the Signature page
%		(with signatures) is AFTER the Title page.
%
%	If you are writing a masters thesis or report, you MUST REVERSE
%	the order of the \commcertpage and \titlepage commands below.
%
\commcertpage           % Produces the Committee Certification
			%   of Approved Version page (doctoral)
			%   or Signature page (masters).
			%		20 Mar 2002	cwm

\titlepage  % Produces the title page.



%%%%%%%%%%%%%%%%%%%%%%%%%%%%%%%%%%%%%%%%%%%%%%%%%%%%%%%%%%%%%%%%%%%%%%
% Dedication and/or epigraph are optional, but must occur here.      %
%%%%%%%%%%%%%%%%%%%%%%%%%%%%%%%%%%%%%%%%%%%%%%%%%%%%%%%%%%%%%%%%%%%%%%
%
\begin{dedication}
\index{Dedication@\emph{Dedication}}%
Dedicated to my parents.
\end{dedication}


\begin{acknowledgments}		% Optional
\index{Acknowledgments@\emph{Acknowledgments}}%
Foremost, I want to express my greatest gratitude to my supervisor, Professor Thomas J.R. Hughes, for his guidance and inspiration. This work truly benefits from his profound insights in the broad areas of mechanics, mathematics, and scientific computing.

I would like to thank Professors Todd Arbogast, Omar Ghattas, Hector Gomez, Chad M. Landis, and Alexis F. Vasseur for serving my dissertation committee.

I am indebted to Dr. Luca Ded\`e and Prof. John A. Evans for teaching me the surviving skills in computational mechanics. I want to thank Prof. Mike Borden for many helpful suggestions on finite element programming. I am grateful to Dr. Shaolie Hossain and Mr. Fred Nugen for their continuing encouragement during my Ph.D. study.

I am thankful to all my friends in Austin, in particular Nick Alger, Jie Bai, Henry Chang, Mike Harmon, Ying He, Talea Mayo, Yusuke Sakamoto, Kent van Vels, Ni Wang, Wenhao Wang, Hailong Xiao, Shan Yang, Wenqi Zhao, and Hongyu Zhu.

I want to thank my parents and my girlfriend. My accomplishment is impossible without your sacrifices and love.

\end{acknowledgments}


% The abstract is required. Note the use of ``utabstract'' instead of
% ``abstract''! This was necessary to fix a page numbering problem.
% The abstract heading is generated automatically.
% Do NOT use \begin{abstract} ... \end{abstract}.
%
\utabstract
\index{Abstract}%
\indent
Multiphase flow is a familiar phenomenon from daily life and occupies an important role in physics, engineering, and medicine. The understanding of multiphase flows relies largely on the theory of interfaces, which is not well understood in many cases. To date, the Navier-Stokes-Korteweg equations \cite{Korteweg1901,Waals1979} and the Cahn-Hilliard equation \cite{Cahn1958} have represented two major branches of phase-field modeling. The Navier-Stokes-Korteweg equations describe a single component fluid material with multiple states of matter, e.g., water and water vapor; the Cahn-Hilliard type models describe multi-component materials with immiscible interfaces, e.g., air and water. In this dissertation, a unified multiphase fluid modeling framework is developed based on rigorous mathematical and thermodynamic principles. This framework does not assume any ad hoc modeling procedures and is capable of formulating meaningful new models with an arbitrary number of different types of interfaces.

In addition to the modeling, novel numerical technologies are developed in this dissertation focusing on the Navier-Stokes-Korteweg equations. First, the notion of entropy variables is properly generalized to the functional setting, which results in an entropy-dissipative semi-discrete formulation. Second, a family of quadrature rules is developed and applied to generate fully discrete schemes. The resulting schemes are featured with two main properties: they are provably dissipative in entropy and second-order accurate in time. In the presence of complex geometries and high-order differential terms, isogeometric analysis \cite{Hughes2005} is invoked to provide accurate representations of computational geometries and robust numerical tools. A novel periodic transformation operator technology is also developed within the isogeometric context. It significantly simplifies the procedure of the strong imposition of periodic boundary conditions. These attributes make the proposed technologies an ideal candidate for credible numerical simulation of multiphase flows.

A general-purpose parallel computing software, named \texttt{PERIGEE}, is developed in this work to provide an implementation framework for the above numerical methods. A comprehensive set of numerical examples has been studied to corroborate the aforementioned theories. Additionally, a variety of application examples have been investigated, culminating with the boiling simulation. Importantly, the boiling model overcomes several challenges for traditional boiling models, owing to its thermodynamically consistent nature. The numerical results indicate the promising potential of the proposed methodology for a wide range of multiphase flow problems.



\tableofcontents   % Table of Contents will be automatically
                   % generated and placed here.

\listoftables      % List of Tables and List of Figures will be placed
\listoffigures     % here, if applicable.



%%%%%%%%%%%%%%%%%%%%%%%%%%%%%%%%%%%%%%%%%%%%%%%%%%%%%%%%%%%%%%%%%%%%%%
% Actual text starts here.					     %
%%%%%%%%%%%%%%%%%%%%%%%%%%%%%%%%%%%%%%%%%%%%%%%%%%%%%%%%%%%%%%%%%%%%%%
%
% Including external files for each chapter makes this document simpler,
% makes each chapter simpler, and allows for generating test documents
% with as few as zero chapters (by commenting out the include statements).
% This allows quicker processing by the Makediss command file in case you
% are not working on a specific, long and slow to compile chapter. You
% can even change the chapter order by merely interchanging the order
% of the include statements (something I found helpful in my own
% dissertation).
%

\subfile{Introduction.tex}

\subfile{LocalConservation.tex}

\subfile{TimeStepping.tex}

\subfile{RobustConvectionDiffusion.tex}

\subfile{Incompressible.tex}

\subfile{Compressible.tex}

% \subfile{FutureWork.tex}

%%%%%%%%%%%%%%%%%%%%%%%%%%%%%%%%%%%%%%%%%%%%%%%%%%%%%%%%%%%%%%%%%%%%%%
% Appendix/Appendices                                                %
%%%%%%%%%%%%%%%%%%%%%%%%%%%%%%%%%%%%%%%%%%%%%%%%%%%%%%%%%%%%%%%%%%%%%%
%
% If you have only one appendix, use the command \appendix instead
% of \appendices.
%
\appendices
\index{Appendices@\emph{Appendices}}%

\subfile{VariableComparison.tex}

\subfile{EntropyNorm.tex}

\subfile{Scaling.tex}


%%%%%%%%%%%%%%%%%%%%%%%%%%%%%%%%%%%%%%%%%%%%%%%%%%%%%%%%%%%%%%%%%%%%%%
% Generate the bibliography.					     %
%%%%%%%%%%%%%%%%%%%%%%%%%%%%%%%%%%%%%%%%%%%%%%%%%%%%%%%%%%%%%%%%%%%%%%
%								     %
% NOTE: For master's theses and reports, NOTHING is permitted to     %
%	come between the bibliography and the vita. The command      %
%	to generate the index (if used) MUST be moved to before      %
%	this section.						     %
%								     %
% \nocite{*}      % This command causes all items in the 		     %
%                 % bibliographic database to be added to 	     %
%                 % the bibliography, even if they are not 	     %
%                 % explicitly cited in the text. 		     %
% 		%						     %
\bibliographystyle{plain}  % Here the bibliography 		     %
\bibliography{../Papers}        % is inserted.			     %
\index{Bibliography@\emph{Bibliography}}%			     %
%%%%%%%%%%%%%%%%%%%%%%%%%%%%%%%%%%%%%%%%%%%%%%%%%%%%%%%%%%%%%%%%%%%%%%


%%%%%%%%%%%%%%%%%%%%%%%%%%%%%%%%%%%%%%%%%%%%%%%%%%%%%%%%%%%%%%%%%%%%%%
% Generate the index.						     %
%%%%%%%%%%%%%%%%%%%%%%%%%%%%%%%%%%%%%%%%%%%%%%%%%%%%%%%%%%%%%%%%%%%%%%
%								     %
% NOTE: For master's theses and reports, NOTHING is permitted to     %
%	come between the bibliography and the vita. This section     %
%	to generate the index (if used) MUST be moved to before      %
%	the bibliography section.				     %
%								     %
%\printindex%    % Include the index here. Comment out this line      %
%		% with a percent sign if you do not want an index.   %
%%%%%%%%%%%%%%%%%%%%%%%%%%%%%%%%%%%%%%%%%%%%%%%%%%%%%%%%%%%%%%%%%%%%%%


%%%%%%%%%%%%%%%%%%%%%%%%%%%%%%%%%%%%%%%%%%%%%%%%%%%%%%%%%%%%%%%%%%%%%%
% Vita page.							     %
%%%%%%%%%%%%%%%%%%%%%%%%%%%%%%%%%%%%%%%%%%%%%%%%%%%%%%%%%%%%%%%%%%%%%%

\begin{vita}
Ju Liu received the Bachelor of Science degree in Computational Mathematics from Xi'an Jiaotong University in 2008. He immediately entered the Computational and Applied Mathematics program at the University of Texas at Austin under the supervision of Prof. T.J.R. Hughes. His graduate research work has been awarded the Robert J. Melosh medal from the Duke University in 2013. Upon completion of his doctoral degree, he will work as a postdoctoral fellow at the Institute for Computational Engineering and Sciences.

\end{vita}

\end{document}
