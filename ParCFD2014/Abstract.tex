\documentclass[letterpaper]{article}
\usepackage[margin=1.0in]{geometry}
\usepackage{authblk}
\usepackage{amsmath, amssymb, amsthm, mathtools}
\usepackage{graphicx}
\usepackage{caption}
\usepackage{color}
\usepackage{subcaption}
\usepackage{morefloats}

\title{Space-Time DPG: Designing a Method for Massively Parallel CFD}
\author[1]{Truman E. Ellis}
\author[1]{Leszek F. Demkowicz}
\author[2]{Nathan V. Roberts}
\author[3]{Jesse L. Chan}
\affil[1]{Institute for Computational Engineering and Sciences, University of Texas at Austin}

\affil[2]{Argonne Leadership Computing Facility, Argonne National Laboratory}

\affil[3]{Computational and Applied Math, Rice University}
\date{}

\begin{document}
\maketitle

\begin{abstract}
The discontinuous Petrov-Galerkin method is a novel finite element framework with exceptional stability and adaptivity properties.
Initial mesh design is a time consuming and expensive part of CFD simulations as a domain expert has to manually design the 
mesh to achieve near resolution in all parts of the domain lest the numerical method become unstable.
DPG in contrast does not have a pre-asymptotic regime, allowing simulations to start on the coarsest mesh that can adequately represent the domain geometry.
\emph{A posteriori} error estimation and adaptivity can also be done very naturally as DPG comes with an error representation function 
that indicates error in the energy norm.

Automatic adaptivity and pre-asymptotic stability produce a powerful synergy when combined with high performance computing.
Human intervention to correct or adapt a failed massively parallel simulation can be a costly endeavor, prompting the desire for a
numerical technology that will automatically adapt to changing physical dynamics while avoiding ``crashing'' due to under-resolved meshes.
An ideal parallel algorithm should be locally compute intensive while maintaining minimal memory requirements.
These are all observed properties of the discontinuous Petrov-Galerkin finite element method.
Locally computed \emph{optimal test functions} ensure stability on convection dominated flows as well as resolved viscous flow.
Individual contributions from each element to the global stiffness matrix can be computed completely independently due to the discontinuous nature of DPG.
Furthermore, static condensation allows the global solve to concern only the \emph{trace} degrees of freedom; 
trace variables are defined on the mesh skeleton.  
This allows significant reduction of the cost of the global solve.  
The internal degrees of freedom can then be resolved using a fully parallel post-processing step.

We recently began exploring the extension of these attractive DPG properties to space-time domains, allowing us to automate and localize temporal adaptivity
in the same way that we already do with spatial adaptivity. Preliminary results have been generated with Camellia\cite{Roberts2011} for spatially 1D flows,
and a rewrite of Camellia for higher dimensions is currently in progress.
\end{abstract}

\bibliographystyle{plain}
\bibliography{../LCPaper}
\end{document}

