\documentclass[letterpaper]{article}
\usepackage[margin=1.0in]{geometry}
\usepackage{authblk}
\usepackage{amsmath, amssymb, amsthm, mathtools}
\usepackage{graphicx}
\usepackage{caption}
\usepackage{color}
\usepackage{subcaption}
\usepackage{morefloats}
\usepackage{url}

\title{Space-Time DPG: Designing a Method for Massively Parallel CFD}
\author{Truman Ellis\\
Committee: Leszek~Demkowicz (co-advisor), Robert~Moser (co-advisor), Thomas~Hughes, Clint~Dawson, Jayathi~Murthy}
\date{}

\begin{document}
\maketitle

% \textbf{Motivation} 
\paragraph{DPG - a robust adaptive method for CFD}
The discontinuous Petrov-Galerkin method\cite{DPGOverview} is a novel finite element framework with exceptional stability and adaptivity properties.
Initial mesh design is a time consuming and expensive part of CFD simulations as a domain expert has to manually design the 
mesh to achieve near resolution in all parts of the domain lest the numerical method become unstable.
DPG in contrast does not have a pre-asymptotic regime, allowing simulations to start on the coarsest mesh that can adequately represent the domain geometry.
\emph{A posteriori} error estimation and adaptivity can also be done very naturally as DPG comes with an error representation function 
that indicates error in the energy norm.

Automatic adaptivity and pre-asymptotic stability produce a powerful synthesis when combined with high performance computing.
Human intervention to correct or adapt a failed massively parallel simulation can be a costly endeavor, prompting the desire for a
numerical technology that will automatically adapt to changing physical dynamics while avoiding ``crashing'' due to under-resolved meshes.
An ideal parallel algorithm should be locally compute intensive while maintaining minimal memory requirements\cite{BlastWebPage}.
These are all observed properties of the discontinuous Petrov-Galerkin finite element method.
Locally computed \emph{optimal test functions} ensure stability on convection dominated flows as well as resolved viscous flow.
Individual contributions from each element to the global stiffness matrix can be computed completely independently due to the discontinuous nature of DPG.
Furthermore, static condensation allows the global solve to concern only the \emph{trace} degrees of freedom; 
trace variables are defined on the mesh skeleton.  
This allows significant reduction of the cost of the global solve.  
The internal degrees of freedom can then be resolved using a fully parallel post-processing step.

\paragraph{Importance of local conservation}
Local conservation was an early concern about applying DPG to fluid problems. 
This property is tied to whether or not constants are represented in a method's test space.
Many of the top numerical methods for CFD are automatically conservative by construction, 
but DPG carried no such guarantee since the test space is constructed on the fly.
I ran a number of test problems and found that indeed, standard DPG is not exactly conservative but converges to near exact local conservation 
as it resolves the flow features under consideration.
Leveraging Lagrange multipliers, I developed a new locally conservative DPG formulation\cite{Ellis2013Report} 
that maintains exact conservation even on coarse meshes.
This was found to dramatically improve the coarse mesh quality on several Stokes flow test problems.
We also proved stability and robustness estimates for this variant formulation.

\paragraph{Extending DPG to transient problems}
Most transient CFD simulation tools are usually a combination of some spatial solver coupled with a time stepping algorithm.
Often every spatial element is advanced in lock-step -- requiring the time step to conform to the strictest element in the whole domain.
This approach can be very ineffective for an adaptive simulation where element sizes can vary by several orders of magnitude.
Technologies exist to address these inefficiencies (such as asynchronous variational integrators\cite{Lew2003}), but bolting on another technology
causes us to lose a lot of the desirable DPG properties such as automatic adaptivity and stability.

The goal of this dissertation is to extend the current technology of adaptive steady state DPG to transient problems in space-time.
The emphasis will be on transient incompressible and compressible Navier-Stokes.
We have already pursued some preliminary research in this direction with a space-time DPG solver for spatially 1D flows.
We've applied this solver to a number of test problems for the heat equation, convection-diffusion, Burgers' equation, 
and compressible Navier-Stokes shock tube problems.
On problems with analytical solutions, we have observed optimal rates of convergence.
The application of DPG to massively parallel CFD is the ultimate goal of this research, but this dissertation will be largely developmental.
% A method with the ability to locally adapt in both space and time while performing most of the spatial-temporal work in parallel will
% have profound benefits when it comes to parallel computing.

% \textbf{Contributions}
% My contributions can be divided into the following area categories.


\paragraph{Area A: Applicable mathematics}
DPG is a method built on rigorous mathematical theory.
While rooted in the standard theory of finite elements, DPG is novel enough that many of the old analysis tools do not directly apply.
As a consequence, much of the early DPG literature has been laden with mathematical proofs of stability, convergence, and robustness.
While developing the theory for a locally conservative DPG formulation we performed a stability and robustness analysis.
Another robustness analysis will be necessary as we explore space-time convection-diffusion since the equation is elliptic spatially but hyperbolic in time.
Time permitting, I will start to look at iterative methods and preconditioners for the global solve.

\paragraph{Area B: Numerical analysis and scientific computation}
I will support development of the DPG software framework \emph{Camellia}\cite{Roberts2011} including running verification on several test problems.
This will include the development and verification of spatially 2D space-time simulations.
I will also develop a means to perform time stepping via space-time slabs which should reduce the 
overall run time and memory requirements for longer simulations.
I've already contributed several auxiliary features to \emph{Camellia} including mesh readers and solution export, 
but I also plan on looking into adding HDF5 support along with parallel solution output.

A space-time implementation adds additional dimensionality to the problem under consideration increasing the problem size and memory requirements.
This makes parallel simulation an increasingly important aspect of this work.
The 16 core \emph{Nozomi} cluster is sufficient for early experimentation, but as we ramp up to larger problems, I plan to run on allocations at TACC and on Mira at Argonne National Laboratory.

The major bottleneck to parallel simulations at the moment is that \emph{Camellia} still relies on a serial direct solver for the global solve.
Time permitting, I would like to investigate the implementation of an effective iterative solver for DPG.
% \paragraph{Completed: Collaborative work with Nathan Robers on high order parallel adaptive DPG code Camellia}
% Previous work focused on enabling locally conservative computations with Camellia. 
% My work to this point has primarily been on the application side -- implementing new test problems and exploring how DPG (and Camellia) perform. 
% I have also been instrumental in adding new features to facilitate these tests. 
% I implemented mesh readers to read the GMSH and Triangle mesh formats to enable computations on nontrivial domains. 
% I also wrote output code to interface Camellia with the VTK library, allowing us to visualize our results.
%  % \paragraph{Proposed: Continued development of Camellia with emphasis on enabling space-time DPG}
% Since the proposed work is largely exploratory, the emphasis is not so much on high performance computing, algorithms, and optimization, but more on discovering whether the space-time DPG technology holds promise for the future. 
% We focus on extending the adaptive nature of previous DPG work with local space-time adaptivity.

% HPC systems

% We are also interested in adding many auxiliary features to Camellia in the process. Solution output is currently done completely in serial, but parallel output is a desired feature as we move forward. We are also interested in removing our VTK dependency and switching to a new IO format using HDF5 and XDMF.

\paragraph{Area C: Mathematical modeling and applications}
As we are exploring a method that is still very much in development, I will be running a number of standard test problems.
I've already explored the conservation properties of DPG through a couple of simulations of Stokes flow around a cylinder and over
a backward facing step.

The steady state DPG formulation failed to converge on the Carter plate problem for Reynolds numbers of $10^7$. 
We suspected that this was due to transient effects on the unresolved mesh which pushed the time step requirements very low.
The hope was that by using a space-time formulation with temporal adaptivity, we would be able to resolve these temporal oscillations
and they would die away to the correct steady state solution.
Therefore, repeating this problem in space-time at $Re=10^7$ will be my first priority once we get a 2D space-time code working.
If this proves to be an effective way of handling transient shock problems, 
I would also like to simulate the Sedov explosion problem and the Noh implosion problem.
Time permitting, I would also like to run the triple point shock interaction problem and a Rayleigh-Taylor instability problem.
These will be ideal for demonstrating the local adaptivity of DPG.

That said, I believe that the most promising application for space-time DPG will be the incompressible Navier-Stokes equations.
It would be interesting to run verification on the lid driven cavity problem to make sure we get the same results as the steady formulation.
There are several vortex shedding problems that would be of interest including flow over a triangle, square, and cylinder.

% \nocite{DPGOverview}

\bibliographystyle{plain}
\bibliography{../Papers}
\end{document}

