% -*- root: Proposal.tex -*-
\documentclass[Proposal.tex]{subfiles} 
\begin{document}

% \paragraph{Area A: Applicable mathematics.}
% DPG is a method built on rigorous mathematical theory.
% While rooted in the standard theory of finite elements, DPG is novel enough that many of the old analysis tools do not directly apply.
% As a consequence, much of the early DPG literature has been laden with mathematical proofs of stability, convergence, and robustness.
% While developing the theory for a locally conservative DPG formulation we performed a stability and robustness analysis.
% Another robustness analysis will be necessary as we explore space-time convection-diffusion since the equation parabolic in space-time.
% Past DPG robustness analysis focused on purely elliptic problems.

% One significant remaining obstacle to obtaining robust solutions for compressible Navier-Stokes is guaranteeing positivity of the density and
% energy in the presence of unresolved shocks. 
% Positivity is an auxiliary stability condition that is not guaranteed by traditional stabilization 
% techniques (including DPG), so it we are going to need to augment an additional technology to the method in order to guarantee physically
% realizable densities and energies. 
% Unfortunately, positivity preservation is itself an incredibly difficult problem. 
% We have several ideas we want to try in the context of DPG, but solving the issue may not lie within the scope of this dissertation.

% \paragraph{Area B: Numerical analysis and scientific computation.}
% I will support development of the DPG software framework \emph{Camellia}\cite{Roberts2011} including running verification on several test problems.
% This will include the development and verification of spatially 2D space-time simulations.
% I will also develop a means to perform time stepping via space-time slabs which should reduce the 
% overall run time and memory requirements for longer simulations.
% I've already contributed several auxiliary features to \emph{Camellia} including mesh readers and solution export, 
% but I also plan on looking into adding HDF5 support along with parallel solution output.

% A space-time implementation adds additional dimensionality to the problem under consideration increasing the problem size and memory requirements.
% This makes parallel simulation an increasingly important aspect of this work.
% The 16 core \emph{Nozomi} cluster is sufficient for early experimentation, but as we ramp up to larger problems, I plan to run on allocations at TACC and on Mira at Argonne National Laboratory.

% The major bottleneck to parallel simulations at the moment is that \emph{Camellia} still relies on a serial direct solver for the global solve.
% Time permitting, I would like to investigate the implementation of an effective iterative solver for DPG.

% \paragraph{Area C: Mathematical modeling and applications.}
% As we are exploring a method that is still very much in development, I will be running a number of standard test problems.
% I've already explored the conservation properties of DPG through a couple of simulations of Stokes flow around a cylinder and over
% a backward facing step.

% The steady state DPG formulation failed to converge on the Carter plate problem for Reynolds numbers of $10^7$. 
% One hypothesis is that this was due to transient effects on the unresolved mesh which pushed the time step requirements very low.
% The hope is that by using a space-time formulation with temporal adaptivity, we would be able to resolve these temporal oscillations
% and they would die away to the correct steady state solution.
% Another hypothesis is that the non-convergence was caused by Gibbs phenomenon near the unresolved leading edge driving the density negative.
% Our line search algorithm scales the Newton updates to prevent negative densities, but sometimes the lines search factor can be driven so small
% as to effectively stall convergence.
% Therefore my first priority will be to investigate which factor is stalling convergence and attempt to solve it either via adaptive space-time
% refinements or some sort of positivity preserving technology yet to be decided upon.
% If this proves to be an effective way of handling transient shock problems, 
% I would also like to simulate the Sedov explosion problem and the Noh implosion problem.
% Time permitting, I would also like to run the triple point shock interaction problem and a Rayleigh-Taylor instability problem.
% These will be ideal for demonstrating the local adaptivity of DPG.

% That said, I believe that the most promising application for space-time DPG will be the incompressible Navier-Stokes equations.
% Here we don't run into any of the same issues with Gibbs phenomenon and we can use vanilla space-time DPG.
% The Taylor-Green vortex problem is an obvious choice since it has a exact solution that we can compare to.
% There are several vortex shedding problems that would be of interest including flow over a triangle, square, and cylinder.

\end{document}