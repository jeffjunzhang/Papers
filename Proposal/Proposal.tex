\documentclass[12pt]{report}
\usepackage{./utproposal}
\usepackage{}
% -----PACKAGES
%\usepackage[shortend,titlenumbered]{algorithm2e}
%\usepackage{algorithmic}
%\usepackage[plain]{algorithm}
\usepackage{multicol}
\usepackage{color}
\usepackage{multirow}
\usepackage{fancybox}
%\usepackage{index}
\usepackage{varioref}
\usepackage{psfrag}
\usepackage{epsfig}
\usepackage{boxedminipage}
\usepackage{graphicx}
\usepackage{rotating}
\usepackage{amsmath}
\usepackage{amssymb}
%\usepackage{amsfont}
\usepackage{latexsym}
\usepackage{alltt}
%\usepackage[small,bf]{caption}
\usepackage{url}
%\usepackage{citesort}
%\usepackage{crop}
\usepackage{array}
\usepackage{subfigure}
\usepackage{dcolumn}

% -----SETLENGTH
%\setlength{\captionmargin}{20pt} 

% -----NEWCOMMANDS
\newcommand{\nc}{\newcommand}
\nc{\mathsm}[1]{\text{\small{$#1$}}}
\nc{\ubar}[1]{\underset{-}{#1}}
\nc{\optype}{\textrm}
\nc{\EQ}[1]{(\ref{eq:#1})}
\nc{\TAB}[1]{\ref{tab:#1}}
\nc{\FIG}[1]{\ref{fig:#1}}
\nc{\SEC}[1]{\ref{sec:#1}}
\nc{\ALG}[1]{\ref{alg:#1}}
\nc{\CHAP}[1]{\ref{chap:#1}}
\nc{\mtrx}[1]{\boldsymbol{\mathbf{#1}}}
\nc{\vctr}[1]{\boldsymbol{\mathbf{#1}}}
\nc{\grad}{\mbox{\boldmath$\nabla$}}
\nc{\gradient}{\textsl{grad}\,}
\nc{\hessian}{\textsl{grad\,}^2}
\nc{\ii}{\iota}
\nc{\dd}{d}
\nc{\ee}{\mathrm{e}}
\nc{\pdiv}[2]{\partial{#1}/\partial{#2}}
\nc{\dpdiv}[2]{\displaystyle{\frac{\partial{#1}}{\partial{#2}}}}
\nc{\ddiv}[2]{\displaystyle{\frac{\dd{#1}}{\dd{#2}}}}
\nc{\inpr}{\hspace{-1pt}\cdot\hspace{-1pt}}
\nc{\IR}{\mathbb{R}}
\nc{\IN}{\mathbb{N}}
\nc{\IZ}{\mathbb{Z}}
\nc{\IC}{\mathbb{C}}
\nc{\half}{\frac{1}{2}}
\nc{\shalf}{\scriptstyle{\half}} 
\nc{\ds}[1]{\displaystyle{#1}}
\nc{\ts}[1]{\textstyle{#1}}
\nc{\sign}{\optype{sign}}
\nc{\spr}{\optype{spr}}
\nc{\dist}{\optype{dist}}
\nc{\rank}{\optype{rank}}
\nc{\codim}{\optype{codim}}
\nc{\supp}{\optype{supp}}
\nc{\diag}{\optype{diag}}
\nc{\meas}{\optype{meas}}
\nc{\cond}{\optype{cond}}
\nc{\kernel}{\optype{kernel}}
\nc{\spa}{\optype{span}}
\nc{\order}{\mathcal{O}}
\nc{\Fr}{\mathrm{Fr}}
\nc{\Rey}{\mathrm{Re}}
\nc{\Ord}{O}
\nc{\ord}{o}
\nc{\st}{\:{:}\:}
\nc{\closure}[1]{\overline{#1}}
\nc{\emin}[1]{\emph{#1}\index{#1}\/}
\nc{\rmin}[1]{#1\index{{}@{#1}}}
\nc{\Laplace}{\Delta}
\nc{\ie}{i.e.}
\nc{\eg}{e.g.}
%\nc{\union}{\cup}
\nc{\Union}{\bigcup}
\nc{\lf}[1]{\mathsf{#1}}
\nc{\dbar}[1]{\bar{\bar{#1}}}
\nc{\ul}[1]{\underline{#1}}
\nc{\hpt}{\hspace{0.5pt}}
\nc{\E}[1]{\times{}10^{#1}}
\nc{\inp}[2]{\langle{#1},{#2}\rangle}
\nc{\tmpcommand}{}

% -----RENEWCOMMANDS
\renewcommand{\baselinestretch}{1}
\renewcommand{\exp}{\optype{exp}\,}
\renewcommand{\cosh}{\optype{cosh}\,}
\renewcommand{\tanh}{\optype{tanh}\,}
\renewcommand{\sinh}{\optype{sinh}\,}
\renewcommand{\div}[1]{\optype{div}\,{#1}}
\renewcommand{\half}{\mbox{$\frac{1}{2}$}}
%\renewcommand{\descriptionlabel}[1]{\hspace{\labelsep}\emph{#1}}

% -----ETC
\raggedbottom


\DeclareMathOperator{\curl}{\bf curl}
\DeclareMathOperator{\rot}{\rm curl}
\DeclareMathOperator{\divv}{\rm div}
\newcommand{\tro}{\gamma_0}
\newcommand{\trt}{\gamma_{\sft}}
\newcommand{\trn}{\gamma_{\sfn}}

\newcommand{\PT}{{\partial T}}
\newcommand{\bbN}{{\mathbb{N}}}
\newcommand{\bbP}{{\mathbb{P}}}

\newcommand{\scC}{{\mathscr{C}}}
\newcommand{\caD}{{\mathcal{D}}}
\newcommand{\caL}{{\mathcal{L}}}

\newcommand{\sfe}{{\mathsf{e}}}
\newcommand{\sff}{{\mathsf{f}}}
\newcommand{\sft}{{\boldsymbol{\mathsf{t}}}}
\newcommand{\sfn}{{\boldsymbol{\mathsf{n}}}}

%   Common caligraphic abbrevs
\newcommand{\BB}{\mathcal{B}}
\newcommand{\CC}{\mathcal{C}}
\newcommand{\DD}{\mathcal{D}}
\newcommand{\EE}{\mathcal{E}}
\newcommand{\FF}{\mathcal{F}}
\newcommand{\GG}{\mathcal{G}}
\newcommand{\II}{\mathcal{I}}
\newcommand{\JJ}{\mathcal{J}}
\newcommand{\KK}{\mathcal{K}}
\newcommand{\LL}{\mathcal{L}}
\newcommand{\OO}{\mathcal{O}}
\newcommand{\QQ}{\mathcal{Q}}
\newcommand{\RR}{\mathcal{R}}
\newcommand{\TT}{\mathcal{T}}


 %% JAY'S PREAMBLE
 %%========================

%   Math symbol definitions
\def\d{\partial}
%\newsymbol\lee 132E
\newcommand{\union}{\mathop{\bigcup}}
\newcommand{\intersect}{\mathop{\bigcap}}
\newcommand{\binomial}[2]{\ensuremath{
		\begin{pmatrix}{#1}\\{#2}\end{pmatrix}}}
\newcommand{\smallbinomial}[2]{\ensuremath{
		(\begin{smallmatrix}{#1}\\{#2}\end{smallmatrix})}}
\newcommand{\tang}[1]{\ensuremath{{#1}_{\intercal}}} % can use \top
						     % also
\newcommand{\hypergeom}[2]{\ensuremath{\sideset{_{#1}}{_{#2}}{\mathop{F}}}}
%   Difficult names
\newcommand{\Babuska}{Babu{\v{s}}ka}       % Remember: Usage is \Babuska\
\newcommand{\Cea}{C{\'e}a}                 % with trailing `\' to give space
\newcommand{\Poincare}{Poincar{\'{e}}}     % when needed, but when ending
\newcommand{\Nedelec}{N{\'{e}}d{\'{e}}lec} % sentence use \Babuska.
\newcommand{\Frechet}{Fr{\'{e}}chet}
\newcommand{\Muller}{M{\"u}ller}
\newcommand{\LHospital}{L'H{\^{o}}spital}
%   Bold and beautiful
\newcommand{\ba}{{\boldsymbol{a}}}
\newcommand{\bA}{\boldsymbol{A}}
\newcommand{\balpha}{{\boldsymbol{\alpha}}}
\newcommand{\bB}{{\boldsymbol{B}}}
\newcommand{\bb}{{\boldsymbol{b}}}
\newcommand{\bbeta}{{\boldsymbol{\beta}}}
\newcommand{\etab}{{\boldsymbol{\eta}}}
\newcommand{\bC}{{\boldsymbol{C}}}
\newcommand{\bc}{{\boldsymbol{c}}}
\newcommand{\bD}{{\boldsymbol{D}}}
\newcommand{\bd}{{\boldsymbol{d}}}
\newcommand{\db}{{\boldsymbol{\d}}}
\newcommand{\bdelta}{{\boldsymbol{\delta}}}
\newcommand{\bDelta}{{\boldsymbol{\Delta}}}
\newcommand{\beps}{{\boldsymbol{\varepsilon}}}
\newcommand{\be}{{\boldsymbol{e}}}
\newcommand{\bg}{{\boldsymbol{g}}}
\newcommand{\bm}{{\boldsymbol{m}}}
\newcommand{\bn}{{\boldsymbol{n}}}
\newcommand{\bN}{{\boldsymbol{N}}}
\newcommand{\bp}{{\boldsymbol{p}}}
\newcommand{\bpsi}{{\boldsymbol{\psi}}}
\newcommand{\bq}{{\boldsymbol{q}}}
\newcommand{\bxi}{{\boldsymbol{\xi}}}
\newcommand{\bE}{{\boldsymbol{E}}}
\newcommand{\bF}{{\boldsymbol{F}}}
\newcommand{\bh}{{\boldsymbol{h}}}
\newcommand{\bH}{{\boldsymbol{H}}}
\newcommand{\bI}{{\boldsymbol{I}}}
\newcommand{\bj}{{\boldsymbol{j}}}
\newcommand{\bJ}{{\boldsymbol{J}}}
\newcommand{\bK}{{\boldsymbol{K}}}
\newcommand{\bk}{{\boldsymbol{k}}}
\newcommand{\bll}{{\boldsymbol{\ell}}}
\newcommand{\bL}{{\boldsymbol{L}}}
\newcommand{\blambda}{{\boldsymbol{\lambda}}}
\newcommand{\bmu}{{\boldsymbol{\mu}}}
\newcommand{\bM}{{\boldsymbol{M}}}
\newcommand{\bomega}{{\boldsymbol{\omega}}}
\newcommand{\bP}{{\boldsymbol{P}}}
\newcommand{\bphi}{{\boldsymbol{\phi}}}
\newcommand{\bQ}{{\boldsymbol{Q}}}
\newcommand{\bG}{{\boldsymbol{G}}}
\newcommand{\bu}{{\boldsymbol{u}}}
\newcommand{\bU}{{\boldsymbol{U}}}
\newcommand{\bV}{{\boldsymbol{V}}}
\newcommand{\bX}{{\boldsymbol{X}}}
\newcommand{\bv}{{\boldsymbol{v}}}
\newcommand{\bw}{{\boldsymbol{w}}}
\newcommand{\bW}{{\boldsymbol{W}}}
\newcommand{\bR}{{\boldsymbol{R}}}
\newcommand{\br}{{\boldsymbol{r}}}
\newcommand{\bS}{{\boldsymbol{S}}}
\newcommand{\bT}{{\boldsymbol{T}}}
\newcommand{\btau}{{\boldsymbol{\tau}}}
\newcommand{\bt}{{\boldsymbol{t}}}
\newcommand{\bx}{{\boldsymbol{x}}}
\newcommand{\by}{{\boldsymbol{y}}}
\newcommand{\bz}{{\boldsymbol{z}}}
\newcommand{\bzero}{{\boldsymbol{0}}}
\newcommand{\bZ}{{\boldsymbol{Z}}}
%   Common scalar fields
\newcommand{\RRR}{\mathbb{R}}
\newcommand{\CCC}{\mathbb{C}}
\newcommand{\ZZZ}{\mathbb{Z}}
\newcommand{\NNN}{\mathbb{N}}
%   Differential operators
\newcommand{\dive}{\mathop\mathrm{div}}
%\newcommand{\grad}{\ensuremath{\mathop{{\bf{grad}}}}}
%\newcommand{\curl}{{\ensuremath\mathop{\mathbf{curl}\,}}}
\newcommand{\Curl}{ {\bf Curl}}
\newcommand{\dx}{\ensuremath{\mathrm{d}x}}
\newcommand{\dy}{\ensuremath{\mathrm{d}y}}
\newcommand{\dr}{\ensuremath{\mathrm{d}r}}
\newcommand{\dR}{\ensuremath{\mathrm{d}R}}
\newcommand{\drho}{\ensuremath{\mathrm{d}\rho}}
\newcommand{\dz}{\ensuremath{\mathrm{d}z}}
\newcommand{\dzeta}{\ensuremath{\mathrm{d}\zeta}}
%   Wordy math symbols
\newcommand{\card}{\ensuremath{\mathop\mathrm{card}}}
%\newcommand{\diag}{\ensuremath{\mathop\mathrm{diag}}}
\newcommand{\diam}{\ensuremath{\mathop\mathrm{diam}}}
%\newcommand{\dist}{\mathop\mathrm{dist}}
\newcommand{\Ker}{\mathop\mathrm{Ker}}
\newcommand{\Range}{\mathop\mathrm{Range}}
%\newcommand{\rank}{\mathop\mathrm{rank}}
%\newcommand{\meas}{\mathop\mathrm{meas}}
\newcommand{\Forall}{\quad\text{for all }}
%\newcommand{\supp}{\mathop\mathrm{supp}}
\newcommand{\Span}{\mathop\mathrm{Span}}
\newcommand{\Hdiv}[1]{\bH(\dive,#1)}
%\newcommand{\Hcurl}[1]{\bH(\curl,#1)}
%   Common caligraphic abbrevs
%\newcommand{\BB}{\mathcal{B}}
%\newcommand{\CC}{\mathcal{C}}
%\newcommand{\DD}{\mathcal{D}}
%\newcommand{\EE}{\mathcal{E}}
%\newcommand{\FF}{\mathcal{F}}
%\newcommand{\GG}{\mathcal{G}}
%\newcommand{\II}{\mathcal{I}}
%\newcommand{\JJ}{\mathcal{J}}
%\newcommand{\KK}{\mathcal{K}}
%\newcommand{\LL}{\mathcal{L}}
%\newcommand{\OO}{\mathcal{O}}
%\newcommand{\QQ}{\mathcal{Q}}
%\newcommand{\RR}{\mathcal{R}}
%\newcommand{\TT}{\mathcal{T}}
%   Variations on standard symbols
\newcommand{\veps}{\varepsilon}
\newcommand{\vlam}{\varLambda}
\newcommand{\vpi}{\varPi}
\newcommand{\vPi}{\boldsymbol{\varPi}}
\newcommand{\vsig}{\varSigma}
\newcommand{\vbt}{\boldsymbol{\varTheta}}
\newcommand{\vPsi}{\boldsymbol{\varPsi}}
%\newcommand{\ii}{\hat{\imath}}
%   Innerproducts, norms, etc
\newcommand{\ntrip}[1]{|\!|\!| {#1} |\!|\!|}
\newcommand{\ip}[1]{\langle {#1} \rangle}
%   Utilities
\newcommand{\blnk}{\underline{\hspace{3cm}}\;}
\newcommand{\marg}[1]{\marginpar{\tiny{\framebox{\parbox{1.7cm}{#1}}}}}
\newcommand{\degreeC}[1]{\ensuremath{{#1\,}^\circ\!\text{C}}}
                        % try also  \textcelsius of textcomp package
%   Trademarked names \texttrademark, \textregistered
\newcommand{\matlab}{MATLAB\textregistered\renewcommand{\matlab}{MATLAB}}
\newcommand{\femlab}{FEMLAB\textregistered\renewcommand{\femlab}{FEMLAB}}

%   Style preferences
\renewcommand{\thefootnote}{\fnsymbol{footnote}} % Use symbols instead of
						 % numbers for footnotes
						 

\newcommand{\Eg}{\EE^\mathrm{grad}}
\newcommand{\Ec}{\boldsymbol{\EE}^\mathrm{curl}}
\newcommand{\Ed}{\boldsymbol{\EE}^\mathrm{div}}


\newcommand{\bfdu}{\mbox{\boldmath $\delta u$}}
\newcommand{\bfdv}{\mbox{\boldmath $\delta v$}}
\newcommand{\du}{{\delta u}}
\newcommand{\dv}{{\delta v}}
\newcommand{\bfnabt}{\widetilde{\bfnab}}
\newcommand{\bfepst}{\widetilde{\bfeps}}


\author{Truman E. Ellis}
\address{712 Upson St.\\ Austin, Texas 78703}  % Required

\title{A Space-Time DPG Method for Fluid Dynamics}

\supervisor
	[Leszek Demkowicz]
	{Robert Moser}

\committeemembers
	[Thomas Hughes]
	[Todd Arbogast]
	{Clinton Dawson}

\previousdegrees{M.S.}

\oneandonehalfspacequote

\topmargin 0.125in

\makeindex

%%%%%%%%%%%%%%% Start of thesis %%%%%%%%%%%%%%%%%%%

\begin{document}
%\copyrightpage          % Produces the copyright page.
%\commcertpage           % Produces the Committee Certification
\titlepage

\tableofcontents   % Table of Contents will be automatically
                   % generated and placed here.

%\listoftables      % List of Tables and List of Figures will be placed
%\listoffigures     % here, if applicable.


\chapter{Introduction}



\section{Motivation} 

% \subsection{Classes of problems}
Computational science has revolutionized the engineering design process -- enabling design analysis
and optimization to be done virtually before expensive physical prototypes need to be built.
However, some fields of engineering analysis lend themselves to a computational approach much easier
than others. 
Fluid dynamics has long been one of the most challenging engineering disciplines to simulate via numerical techniques.
Aside from the inherent modeling challenges presented by fluid turbulence, many fluid flows can be characterized as singularly perturbed problems 
-- problems in which the viscosity length scale is many orders of maginitude smaller than the large scale features of the flow.
This has necessitated the need for meshes with large gradations in resolution to enable resolution of boundary layers while being computationally efficient in the free stream.
Traditionally, these meshes would be custom designed by a domain expert who could predict which parts of the domain would need more resolution than others. 
On top of this, many numerical techniques would fail to converge unless the presented initial mesh was in the ``asymptotic regime'', 
i.e. the physics could by somewhat sufficiently represented.
These requirements made mesh generation a laborious and far from automated procedure.

The failure of many numerical methods in the ``pre-asymptotic regime'' can be characterized mathematically as a loss of stability on coarse meshes.
The stability characteristics of a broad class of finite element methods can be analyzed according to the Lady\v{z}enskaja-Babu\v{s}ka-Brezzi condition.
Leszek Demkowicz and Jay Gopalakrishnan first proposed the discontinuous Petrov-Galerkin method in 2009\cite{DPG1} in order to address stability issues for a 
very broad class of problems. The DPG method automatically satisfies stability criteria by construction which enables DPG simulations to remain stable and 
convergent even in the pre-asymptotic regime. 
By nature, the DPG method also comes with a built-in error representation function, effectively eliminating the need for other a posteriori error estimators.
Practically, this means that a simulation could start with just the coarsest mesh necessary to represent the geometry of the solution and adaptively refine toward a resolved solution in a very automatic way.
Carried to its logical conclusion, this capability could significantly cut down on the time intensive manual mesh generation (and tweaking) that dominates a good amount of simulation and analysis time.
Where a current numerical method might falter on a poorly designed mesh, necessitating an engineer to manually enter the problem and fix the offending mesh nodes, a DPG simulation would converge on the poor mesh, mark the offending cells, refine, and continue toward a solution.

Another benefit to the enhanced stability properties of DPG is the ability to consider high order and $hp$-adaptive methods. 
Many popular numerical methods for CFD (such as the discontinuous Galerkin method) are stable for low polynomial orders, but require additional stabilizing terms for higher orders. 
Additionally, one of the longstanding issues with $hp$-adaptive techniques was that they suffered stability problems when the polynomial order rose to high. 
Polynomial order presents no issue at all to DPG methods -- allowing us to recover the high order convergence rates of high uniform $p$ methods or even the exponential convergence rates of $hp$ methods.

The biggest limitation to past explorations of the DPG method is that they were all limited to steady state problems.

% The Navier-Stokes equations govern many systems of engineering interest.
% transient compressible and incompressible laminar Navier-Stokes, extreme Re,
% transonic/supersonic
% This work is ultimately concerned with solving the transient incompressible and compressible Navier-Stokes equations, but we consider several model problems in the process.
% \paragraph{Higher-order methods}
% gives higher convergence rates, but seen as more sensitive (Gibbs)
% \paragraph{Automatic adaptivity}
% CFD inherently requires meshes with magnitudes of variation in scale
% resolution.
% limitations exist, requires a field expert to design meshes and run
% simulations
% \paragraph{Time step limited by smallest mesh element} \cite{Lew2003}

\subsection{Investigating a new methodology}
Much of science is driven by curiosity, and this especially holds for computational science. 
There is inherent value in exploring new methodologies because they may hold the keys to solving new problems or old problems in a better way.
A new method may also help us to better understand existing methods. 
The variational multiscale approach to finite element analysis helped to illucidate on some of the success of the much older streamwise upwind Petrov-Galerkin method while generalizing and improving it.
The DPG method itself can be viewed as a generalization of least-squares finite elements or even of mixed methods. 
% Aside from the practical incentives of pursuing \ldots
% Generalizing to space-time, investigating conservation


\section{Literature review}

\subsection{Computational fluid dynamics}
FD, FV, FE (SUPG, Stabilized, DG) 3 pages

\subsubsection{Finite difference and finite volume methods}

\subsubsection{Stabilized finite element methods}
\paragraph{SUPG}
\paragraph{VMS}
\paragraph{DG}
\paragraph{HDG}

\subsection{Space-time finite elements}
Oden (first to propose), Bob Haber, Tayfun Tezduyar, Neum\"{u}ller
\cite{Klaij2006}
\cite{Rhebergen2013}
\cite{Haber2006}

\subsection{DPG}
General ideas 1-2 pages


\section{Goal}



\chapter{Conservation in steady-state}
insert paper, 1-15 pages


\section{DPG is a minimum residual method}


\section{Element conservative convection-diffusion}

\subsection{Derivation}

\subsection{Stability analysis}
\subsubsection{Robustness analysis}

\subsection{Robust test norms}
\subsubsection{A model problem}
\subsubsection{A modification of the robust test norm}
\subsubsection{Adaptation for a locally conservative formulation}
\subsubsection{Proof of robust stability estimate}


\section{Application to other fluid model problems}

\subsection{Inviscid Burgers' equation}

\subsection{Stokes flow}


\section{Numerical Experiments}

\subsection{Erickson-Johnson model problem}

\subsection{Vortex problem}

\subsection{Discontinuous source problem}

\subsection{Inviscid Burgers' problem}

\subsection{Stokes flow around a cylinder}

\subsection{Stokes flow over a backward facing step}



\chapter{Space-time DPG}



\section{Heat equation}

\subsection{Derivation}

\subsection{Problems considered}


\section{Convection-diffusion}

\subsection{Derivation}

\subsection{Problems considered}


\section{Burgers'}

\subsection{Derivation}

\subsection{Problems considered}


\section{Time-slabs vs full-time}



\chapter{Proposed work}



\section{Space-time DPG for 2D Incompressible Navier-Stokes}

\subsection{Derivation}

\subsection{Problems to consider}


\section{Space-time DPG for 2D Compressible Navier-Stokes}

\subsection{Derivation}

\subsection{Problems to consider}


\section{Area requirements}

\subsection{Area A: Applicable mathematics}
\subsubsection*{Completed:}
\subsubsection*{Proposed:}

\subsection{Area B: Numerical analysis and scientific computation}
\subsubsection*{Completed: Collaborative work with Nathan Robers on high order parallel adaptive DPG code Camellia}
Previous work focused on enabling locally conservative computations with Camellia. My work to this point has primarily been on the application side -- implementing new test problems and exploring how DPG (and Camellia) perform. I have also been instrumental in adding new features to facilitate these tests. I implemented mesh readers to read the GMSH and Triangle mesh formats to enable computations on nontrivial domains. I also wrote output code to interface Camellia with the VTK library, allowing us to visualize our results.
 \subsubsection*{Proposed: Continued development of Camellia with emphasis on enabling space-time DPG}
Since the proposed work is largely exploratory, the emphasis is not so much on high performance computing, algorithms, and optimization, but more on discovering whether the space-time DPG technology holds promise for the future. We focus on extending the adaptive nature of previous DPG work with local space-time adaptivity.

We are also interested in adding many auxiliary featurs to Camellia in the process. Solution output is currently done completely in serial, but parallel output is a desired feature as we move forward. We are also interested in removing our VTK dependency and switching to a new IO format using HDF5 and XDMF.

\subsection{Area C: Mathematical modeling and applications}
\subsubsection*{Completed:}
\subsubsection*{Proposed:}

\bibliographystyle{plain}
\bibliography{../DPG}

\end{document}
