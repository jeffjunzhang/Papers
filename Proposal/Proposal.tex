\documentclass{report}
\usepackage{./utproposal}
\usepackage{}
% -----PACKAGES
%\usepackage[shortend,titlenumbered]{algorithm2e}
%\usepackage{algorithmic}
%\usepackage[plain]{algorithm}
\usepackage{multicol}
\usepackage{color}
\usepackage{multirow}
\usepackage{fancybox}
%\usepackage{index}
\usepackage{varioref}
\usepackage{psfrag}
\usepackage{epsfig}
\usepackage{boxedminipage}
\usepackage{graphicx}
\usepackage{rotating}
\usepackage{amsmath}
\usepackage{amssymb}
%\usepackage{amsfont}
\usepackage{latexsym}
\usepackage{alltt}
%\usepackage[small,bf]{caption}
\usepackage{url}
%\usepackage{citesort}
%\usepackage{crop}
\usepackage{array}
\usepackage{subfigure}
\usepackage{dcolumn}

% -----SETLENGTH
%\setlength{\captionmargin}{20pt} 

% -----NEWCOMMANDS
\newcommand{\nc}{\newcommand}
\nc{\mathsm}[1]{\text{\small{$#1$}}}
\nc{\ubar}[1]{\underset{-}{#1}}
\nc{\optype}{\textrm}
\nc{\EQ}[1]{(\ref{eq:#1})}
\nc{\TAB}[1]{\ref{tab:#1}}
\nc{\FIG}[1]{\ref{fig:#1}}
\nc{\SEC}[1]{\ref{sec:#1}}
\nc{\ALG}[1]{\ref{alg:#1}}
\nc{\CHAP}[1]{\ref{chap:#1}}
\nc{\mtrx}[1]{\boldsymbol{\mathbf{#1}}}
\nc{\vctr}[1]{\boldsymbol{\mathbf{#1}}}
\nc{\grad}{\mbox{\boldmath$\nabla$}}
\nc{\gradient}{\textsl{grad}\,}
\nc{\hessian}{\textsl{grad\,}^2}
\nc{\ii}{\iota}
\nc{\dd}{d}
\nc{\ee}{\mathrm{e}}
\nc{\pdiv}[2]{\partial{#1}/\partial{#2}}
\nc{\dpdiv}[2]{\displaystyle{\frac{\partial{#1}}{\partial{#2}}}}
\nc{\ddiv}[2]{\displaystyle{\frac{\dd{#1}}{\dd{#2}}}}
\nc{\inpr}{\hspace{-1pt}\cdot\hspace{-1pt}}
\nc{\IR}{\mathbb{R}}
\nc{\IN}{\mathbb{N}}
\nc{\IZ}{\mathbb{Z}}
\nc{\IC}{\mathbb{C}}
\nc{\half}{\frac{1}{2}}
\nc{\shalf}{\scriptstyle{\half}} 
\nc{\ds}[1]{\displaystyle{#1}}
\nc{\ts}[1]{\textstyle{#1}}
\nc{\sign}{\optype{sign}}
\nc{\spr}{\optype{spr}}
\nc{\dist}{\optype{dist}}
\nc{\rank}{\optype{rank}}
\nc{\codim}{\optype{codim}}
\nc{\supp}{\optype{supp}}
\nc{\diag}{\optype{diag}}
\nc{\meas}{\optype{meas}}
\nc{\cond}{\optype{cond}}
\nc{\kernel}{\optype{kernel}}
\nc{\spa}{\optype{span}}
\nc{\order}{\mathcal{O}}
\nc{\Fr}{\mathrm{Fr}}
\nc{\Rey}{\mathrm{Re}}
\nc{\Ord}{O}
\nc{\ord}{o}
\nc{\st}{\:{:}\:}
\nc{\closure}[1]{\overline{#1}}
\nc{\emin}[1]{\emph{#1}\index{#1}\/}
\nc{\rmin}[1]{#1\index{{}@{#1}}}
\nc{\Laplace}{\Delta}
\nc{\ie}{i.e.}
\nc{\eg}{e.g.}
%\nc{\union}{\cup}
\nc{\Union}{\bigcup}
\nc{\lf}[1]{\mathsf{#1}}
\nc{\dbar}[1]{\bar{\bar{#1}}}
\nc{\ul}[1]{\underline{#1}}
\nc{\hpt}{\hspace{0.5pt}}
\nc{\E}[1]{\times{}10^{#1}}
\nc{\inp}[2]{\langle{#1},{#2}\rangle}
\nc{\tmpcommand}{}

% -----RENEWCOMMANDS
\renewcommand{\baselinestretch}{1}
\renewcommand{\exp}{\optype{exp}\,}
\renewcommand{\cosh}{\optype{cosh}\,}
\renewcommand{\tanh}{\optype{tanh}\,}
\renewcommand{\sinh}{\optype{sinh}\,}
\renewcommand{\div}[1]{\optype{div}\,{#1}}
\renewcommand{\half}{\mbox{$\frac{1}{2}$}}
%\renewcommand{\descriptionlabel}[1]{\hspace{\labelsep}\emph{#1}}

% -----ETC
\raggedbottom


\DeclareMathOperator{\curl}{\bf curl}
\DeclareMathOperator{\rot}{\rm curl}
\DeclareMathOperator{\divv}{\rm div}
\newcommand{\tro}{\gamma_0}
\newcommand{\trt}{\gamma_{\sft}}
\newcommand{\trn}{\gamma_{\sfn}}

\newcommand{\PT}{{\partial T}}
\newcommand{\bbN}{{\mathbb{N}}}
\newcommand{\bbP}{{\mathbb{P}}}

\newcommand{\scC}{{\mathscr{C}}}
\newcommand{\caD}{{\mathcal{D}}}
\newcommand{\caL}{{\mathcal{L}}}

\newcommand{\sfe}{{\mathsf{e}}}
\newcommand{\sff}{{\mathsf{f}}}
\newcommand{\sft}{{\boldsymbol{\mathsf{t}}}}
\newcommand{\sfn}{{\boldsymbol{\mathsf{n}}}}

%   Common caligraphic abbrevs
\newcommand{\BB}{\mathcal{B}}
\newcommand{\CC}{\mathcal{C}}
\newcommand{\DD}{\mathcal{D}}
\newcommand{\EE}{\mathcal{E}}
\newcommand{\FF}{\mathcal{F}}
\newcommand{\GG}{\mathcal{G}}
\newcommand{\II}{\mathcal{I}}
\newcommand{\JJ}{\mathcal{J}}
\newcommand{\KK}{\mathcal{K}}
\newcommand{\LL}{\mathcal{L}}
\newcommand{\OO}{\mathcal{O}}
\newcommand{\QQ}{\mathcal{Q}}
\newcommand{\RR}{\mathcal{R}}
\newcommand{\TT}{\mathcal{T}}


 %% JAY'S PREAMBLE
 %%========================

%   Math symbol definitions
\def\d{\partial}
%\newsymbol\lee 132E
\newcommand{\union}{\mathop{\bigcup}}
\newcommand{\intersect}{\mathop{\bigcap}}
\newcommand{\binomial}[2]{\ensuremath{
		\begin{pmatrix}{#1}\\{#2}\end{pmatrix}}}
\newcommand{\smallbinomial}[2]{\ensuremath{
		(\begin{smallmatrix}{#1}\\{#2}\end{smallmatrix})}}
\newcommand{\tang}[1]{\ensuremath{{#1}_{\intercal}}} % can use \top
						     % also
\newcommand{\hypergeom}[2]{\ensuremath{\sideset{_{#1}}{_{#2}}{\mathop{F}}}}
%   Difficult names
\newcommand{\Babuska}{Babu{\v{s}}ka}       % Remember: Usage is \Babuska\
\newcommand{\Cea}{C{\'e}a}                 % with trailing `\' to give space
\newcommand{\Poincare}{Poincar{\'{e}}}     % when needed, but when ending
\newcommand{\Nedelec}{N{\'{e}}d{\'{e}}lec} % sentence use \Babuska.
\newcommand{\Frechet}{Fr{\'{e}}chet}
\newcommand{\Muller}{M{\"u}ller}
\newcommand{\LHospital}{L'H{\^{o}}spital}
%   Bold and beautiful
\newcommand{\ba}{{\boldsymbol{a}}}
\newcommand{\bA}{\boldsymbol{A}}
\newcommand{\balpha}{{\boldsymbol{\alpha}}}
\newcommand{\bB}{{\boldsymbol{B}}}
\newcommand{\bb}{{\boldsymbol{b}}}
\newcommand{\bbeta}{{\boldsymbol{\beta}}}
\newcommand{\etab}{{\boldsymbol{\eta}}}
\newcommand{\bC}{{\boldsymbol{C}}}
\newcommand{\bc}{{\boldsymbol{c}}}
\newcommand{\bD}{{\boldsymbol{D}}}
\newcommand{\bd}{{\boldsymbol{d}}}
\newcommand{\db}{{\boldsymbol{\d}}}
\newcommand{\bdelta}{{\boldsymbol{\delta}}}
\newcommand{\bDelta}{{\boldsymbol{\Delta}}}
\newcommand{\beps}{{\boldsymbol{\varepsilon}}}
\newcommand{\be}{{\boldsymbol{e}}}
\newcommand{\bg}{{\boldsymbol{g}}}
\newcommand{\bm}{{\boldsymbol{m}}}
\newcommand{\bn}{{\boldsymbol{n}}}
\newcommand{\bN}{{\boldsymbol{N}}}
\newcommand{\bp}{{\boldsymbol{p}}}
\newcommand{\bpsi}{{\boldsymbol{\psi}}}
\newcommand{\bq}{{\boldsymbol{q}}}
\newcommand{\bxi}{{\boldsymbol{\xi}}}
\newcommand{\bE}{{\boldsymbol{E}}}
\newcommand{\bF}{{\boldsymbol{F}}}
\newcommand{\bh}{{\boldsymbol{h}}}
\newcommand{\bH}{{\boldsymbol{H}}}
\newcommand{\bI}{{\boldsymbol{I}}}
\newcommand{\bj}{{\boldsymbol{j}}}
\newcommand{\bJ}{{\boldsymbol{J}}}
\newcommand{\bK}{{\boldsymbol{K}}}
\newcommand{\bk}{{\boldsymbol{k}}}
\newcommand{\bll}{{\boldsymbol{\ell}}}
\newcommand{\bL}{{\boldsymbol{L}}}
\newcommand{\blambda}{{\boldsymbol{\lambda}}}
\newcommand{\bmu}{{\boldsymbol{\mu}}}
\newcommand{\bM}{{\boldsymbol{M}}}
\newcommand{\bomega}{{\boldsymbol{\omega}}}
\newcommand{\bP}{{\boldsymbol{P}}}
\newcommand{\bphi}{{\boldsymbol{\phi}}}
\newcommand{\bQ}{{\boldsymbol{Q}}}
\newcommand{\bG}{{\boldsymbol{G}}}
\newcommand{\bu}{{\boldsymbol{u}}}
\newcommand{\bU}{{\boldsymbol{U}}}
\newcommand{\bV}{{\boldsymbol{V}}}
\newcommand{\bX}{{\boldsymbol{X}}}
\newcommand{\bv}{{\boldsymbol{v}}}
\newcommand{\bw}{{\boldsymbol{w}}}
\newcommand{\bW}{{\boldsymbol{W}}}
\newcommand{\bR}{{\boldsymbol{R}}}
\newcommand{\br}{{\boldsymbol{r}}}
\newcommand{\bS}{{\boldsymbol{S}}}
\newcommand{\bT}{{\boldsymbol{T}}}
\newcommand{\btau}{{\boldsymbol{\tau}}}
\newcommand{\bt}{{\boldsymbol{t}}}
\newcommand{\bx}{{\boldsymbol{x}}}
\newcommand{\by}{{\boldsymbol{y}}}
\newcommand{\bz}{{\boldsymbol{z}}}
\newcommand{\bzero}{{\boldsymbol{0}}}
\newcommand{\bZ}{{\boldsymbol{Z}}}
%   Common scalar fields
\newcommand{\RRR}{\mathbb{R}}
\newcommand{\CCC}{\mathbb{C}}
\newcommand{\ZZZ}{\mathbb{Z}}
\newcommand{\NNN}{\mathbb{N}}
%   Differential operators
\newcommand{\dive}{\mathop\mathrm{div}}
%\newcommand{\grad}{\ensuremath{\mathop{{\bf{grad}}}}}
%\newcommand{\curl}{{\ensuremath\mathop{\mathbf{curl}\,}}}
\newcommand{\Curl}{ {\bf Curl}}
\newcommand{\dx}{\ensuremath{\mathrm{d}x}}
\newcommand{\dy}{\ensuremath{\mathrm{d}y}}
\newcommand{\dr}{\ensuremath{\mathrm{d}r}}
\newcommand{\dR}{\ensuremath{\mathrm{d}R}}
\newcommand{\drho}{\ensuremath{\mathrm{d}\rho}}
\newcommand{\dz}{\ensuremath{\mathrm{d}z}}
\newcommand{\dzeta}{\ensuremath{\mathrm{d}\zeta}}
%   Wordy math symbols
\newcommand{\card}{\ensuremath{\mathop\mathrm{card}}}
%\newcommand{\diag}{\ensuremath{\mathop\mathrm{diag}}}
\newcommand{\diam}{\ensuremath{\mathop\mathrm{diam}}}
%\newcommand{\dist}{\mathop\mathrm{dist}}
\newcommand{\Ker}{\mathop\mathrm{Ker}}
\newcommand{\Range}{\mathop\mathrm{Range}}
%\newcommand{\rank}{\mathop\mathrm{rank}}
%\newcommand{\meas}{\mathop\mathrm{meas}}
\newcommand{\Forall}{\quad\text{for all }}
%\newcommand{\supp}{\mathop\mathrm{supp}}
\newcommand{\Span}{\mathop\mathrm{Span}}
\newcommand{\Hdiv}[1]{\bH(\dive,#1)}
%\newcommand{\Hcurl}[1]{\bH(\curl,#1)}
%   Common caligraphic abbrevs
%\newcommand{\BB}{\mathcal{B}}
%\newcommand{\CC}{\mathcal{C}}
%\newcommand{\DD}{\mathcal{D}}
%\newcommand{\EE}{\mathcal{E}}
%\newcommand{\FF}{\mathcal{F}}
%\newcommand{\GG}{\mathcal{G}}
%\newcommand{\II}{\mathcal{I}}
%\newcommand{\JJ}{\mathcal{J}}
%\newcommand{\KK}{\mathcal{K}}
%\newcommand{\LL}{\mathcal{L}}
%\newcommand{\OO}{\mathcal{O}}
%\newcommand{\QQ}{\mathcal{Q}}
%\newcommand{\RR}{\mathcal{R}}
%\newcommand{\TT}{\mathcal{T}}
%   Variations on standard symbols
\newcommand{\veps}{\varepsilon}
\newcommand{\vlam}{\varLambda}
\newcommand{\vpi}{\varPi}
\newcommand{\vPi}{\boldsymbol{\varPi}}
\newcommand{\vsig}{\varSigma}
\newcommand{\vbt}{\boldsymbol{\varTheta}}
\newcommand{\vPsi}{\boldsymbol{\varPsi}}
%\newcommand{\ii}{\hat{\imath}}
%   Innerproducts, norms, etc
\newcommand{\ntrip}[1]{|\!|\!| {#1} |\!|\!|}
\newcommand{\ip}[1]{\langle {#1} \rangle}
%   Utilities
\newcommand{\blnk}{\underline{\hspace{3cm}}\;}
\newcommand{\marg}[1]{\marginpar{\tiny{\framebox{\parbox{1.7cm}{#1}}}}}
\newcommand{\degreeC}[1]{\ensuremath{{#1\,}^\circ\!\text{C}}}
                        % try also  \textcelsius of textcomp package
%   Trademarked names \texttrademark, \textregistered
\newcommand{\matlab}{MATLAB\textregistered\renewcommand{\matlab}{MATLAB}}
\newcommand{\femlab}{FEMLAB\textregistered\renewcommand{\femlab}{FEMLAB}}

%   Style preferences
\renewcommand{\thefootnote}{\fnsymbol{footnote}} % Use symbols instead of
						 % numbers for footnotes
						 

\newcommand{\Eg}{\EE^\mathrm{grad}}
\newcommand{\Ec}{\boldsymbol{\EE}^\mathrm{curl}}
\newcommand{\Ed}{\boldsymbol{\EE}^\mathrm{div}}


\newcommand{\bfdu}{\mbox{\boldmath $\delta u$}}
\newcommand{\bfdv}{\mbox{\boldmath $\delta v$}}
\newcommand{\du}{{\delta u}}
\newcommand{\dv}{{\delta v}}
\newcommand{\bfnabt}{\widetilde{\bfnab}}
\newcommand{\bfepst}{\widetilde{\bfeps}}


\author{Truman E. Ellis}
\address{712 Upson St.\\ Austin, Texas 78703}  % Required

\title{A Space-Time DPG Method for Fluid Dynamics}

\supervisor
	[Leszek Demkowicz]
	{Robert Moser}

\committeemembers
	[Thomas Hughes]
	[Todd Arbogast]
	{Clinton Dawson}

\previousdegrees{M.S.}

\oneandonehalfspacequote

\topmargin 0.125in

\makeindex

%%%%%%%%%%%%%%% Start of thesis %%%%%%%%%%%%%%%%%%%

\begin{document}
%\copyrightpage          % Produces the copyright page.
%\commcertpage           % Produces the Committee Certification
\titlepage

\tableofcontents   % Table of Contents will be automatically
                   % generated and placed here.

%\listoftables      % List of Tables and List of Figures will be placed
%\listoffigures     % here, if applicable.


\chapter{Introduction}



\section{Motivation} 

% \subsection{Classes of problems}
Computational science has revolutionized the engineering design process -- enabling design analysis
and optimization to be done virtually before expensive physical prototypes need to be built.
However, some fields of engineering analysis lend themselves to a computational approach much easier
than others. 
Fluid dynamics has long been one of the most challenging engineering disciplines to simulate via numerical techniques.
Aside from the inherent modeling challenges presented by fluid turbulence, many fluid flows can be characterized as singularly perturbed problems 
-- problems in which the viscosity length scale is many orders of maginitude smaller than the large scale features of the flow.
This has necessitated the need for meshes with large gradations in resolution to enable resolution of boundary layers while being computationally efficient in the free stream.
Traditionally, these meshes would be custom designed by a domain expert who could predict which parts of the domain would need more resolution than others. 
On top of this, many numerical techniques would fail to converge unless the presented initial mesh was in the ``asymptotic regime'', 
i.e. the physics could by somewhat sufficiently represented.
These requirements made mesh generation a laborious and far from automated procedure.

The failure of many numerical methods in the ``pre-asymptotic regime'' can be characterized mathematically as a loss of stability on coarse meshes.
The stability characteristics of a broad class of finite element methods can be analyzed according to the Lady\v{z}enskaja-Babu\v{s}ka-Brezzi condition.
Leszek Demkowicz and Jay Gopalakrishnan first proposed the discontinuous Petrov-Galerkin method in 2009\cite{DPG1} in order to address stability issues for a 
very broad class of problems. The DPG method automatically satisfies stability criteria by construction which enables DPG simulations to remain stable and 
convergent even in the pre-asymptotic regime. 
By nature, the DPG method also comes with a built-in error representation function, effectively eliminating the need for other a posteriori error estimators.
Practically, this means that a simulation could start with just the coarsest mesh necessary to represent the geometry of the solution and adaptively refine toward a resolved solution in a very automatic way.
Carried to its logical conclusion, this capability could significantly cut down on the time intensive manual mesh generation (and tweaking) that dominates a good amount of simulation and analysis time.
Where a current numerical method might falter on a poorly designed mesh, necessitating an engineer to manually enter the problem and fix the offending mesh nodes, a DPG simulation would converge on the poor mesh, mark the offending cells, refine, and continue toward a solution.

Another benefit to the enhanced stability properties of DPG is the ability to consider high order and $hp$-adaptive methods. 
Many popular numerical methods for CFD (such as the discontinuous Galerkin method) are stable for low polynomial orders, but require additional stabilizing terms for higher orders. 
Additionally, one of the longstanding issues with $hp$-adaptive techniques was that they suffered stability problems when the polynomial order rose to high. 
Polynomial order presents no issue at all to DPG methods -- allowing us to recover the high order convergence rates of high uniform $p$ methods or even the exponential convergence rates of $hp$ methods.

The biggest limitation to past explorations of the DPG method is that they were all limited to steady state problems.
Obviously, this seriously limits the variety of interesting problems we could consider. 
The easiest extension of steady DPG to transient problems would be to do an implicit time stepping technique in time and use DPG for only the spatial solve at each time step.
We did indeed explore this approach, but it didn't seem to be a natural fit with the adaptive features of DPG.
Clearly the CFL condition was not binding since we were interested in implicit time integration schemes, but the CFL condition can be a guiding principle for temporal accuracy in this case.
So if we are interested in temporally accurate solutions, we are limited by the fact that our smallest mesh elements (which may be order of magnitude smaller than the largest elements) are constrained to proceed at a much smaller time step than the mesh as a whole. 
We can either restrict the whole mesh to the smallest time step, or we can attempt some sort of local time stepping.
A space-time DPG formulation presents an attractive choice as we will be able to preserve our natural adaptivity from the steady problems while extending it in time.
Thus we achieve an adaptive solution technique for transient problems in a unified framework.
The obvious downside to such an approach is that for 2D spatial problems, we now have to compute on a three dimensional mesh while a spatially 3D problem becomes four dimensional.

\subsection{Investigating a new methodology}
Much of science is driven by curiosity, and this especially holds for computational science. 
There is inherent value in exploring new methodologies because they may hold the keys to solving new problems or old problems in a better way.
A new method may also help us to better understand existing methods. 
The variational multiscale approach to finite element analysis helped to illucidate on some of the success of the much older streamwise upwind Petrov-Galerkin method while generalizing and improving it.
The DPG method itself can be viewed as a generalization of least-squares finite elements or even of mixed methods. 

Curiousity similarly motivates the desire to explore a space-time DPG formulation for computational fluid dynamics. 
Based on our past experience with steady DPG, we anticipate space-time DPG to be a very interesting technique that could extend the automaticity of DPG in very novel ways.

\section{Literature review}

\subsection{Computational fluid dynamics}

\subsubsection{Finite difference and finite volume methods}

\subsubsection{Stabilized finite element methods}
\paragraph{SUPG}
\paragraph{VMS}
\paragraph{DG}
\paragraph{HDG}

\subsection{Space-time finite elements}
Oden (first to propose), Bob Haber, Tayfun Tezduyar, Neum\"{u}ller
\cite{Klaij2006}
\cite{Rhebergen2013}
% \cite{Haber2006}

\subsection{DPG}
General ideas 1-2 pages


\section{Goal}



\chapter{Abstract DPG}



\chapter{Conservation in steady-state}
We summarize some of our completed work on a locally conservative DPG formulation that was invented to address mass loss concerns for standard DPG.
Locally conservative methods hold a special place for numerical analysts in
the field of fluid dynamics.
Perot\cite{Perot2011} argues
\begin{quote}
Accuracy, stability, and consistency are the mathematical concepts that are
typically used to analyze numerical methods for partial differential equations
(PDEs). These important tools quantify how well the mathematics of a PDE is
represented, but they fail to say anything about how well the physics of the
system is represented by a particular numerical method. In practice, physical
fidelity of a numerical solution can be just as important (perhaps even more
important to a physicist) as these more traditional mathematical concepts. A
numerical solution that violates the underlying physics (destroying mass or
entropy, for example) is in many respects just as flawed as an unstable
solution.
\end{quote}
There are also some mathematically attractive reasons to pursue local
conservation. The Lax-Wendroff theorem guarantees that a convergent numerical
solution to a system of hyperbolic conservation laws will converge to the
correct weak solution.

The discontinuous Petrov-Galerkin finite element method has been described as
least squares finite elements with a twist. The key difference is that least
squares methods seek to minimize the residual of the solution in the $L^2$
norm, while DPG seeks the minimization in a dual norm realized through the
inverse Riesz map. Exact mass conservation has been an issue that has long plagued
least squares finite elements. Several approaches have been
used to try to adress this. Bochev \etal\cite{Bochev2010} accomplish local
conservation by using a pointwise divergence free velocity space in the Stokes
formulation.  Chang and Nelson\cite{ChangNelson1997} developed the
\emph{restricted LSFEM}\cite{ChangNelson1997} by augmenting the least squares
equations with Lagrange multipliers explicitly enforcing mass conservation
element-wise. Our conservative formulation of DPG takes a similar approach and
both methods share similar negative of transforming a minimization method to a
saddle point problem. In the interest of crediting Chang and Nelson's
restricted LSFEM, we call the following locally conservative DPG method the
restricted DPG method (RDPG).


\section{DPG is a minimum residual method}
Roberts \etal presents a brief history and derivation of DPG with optimal test functions in
\cite{DPGStokes}. We follow his derivation of the standard DPG method as a
minimum residual method. Let $U$ be the trial Hilbert space and $V$ the test
Hilbert space for a well-posed variational problem $b(u,v)=l(v)$. In operator
form this is $Bu=l$, where $B:U\rightarrow V'$ and $\LRa{Bu,v}=b(u,v)$. We seek to minimize the
residual for the discrete space $U_h\subset U$:
\begin{equation}
u_h=\argmin_{u_h\in U_h}\frac{1}{2}\norm{Bu_h-l}^2_{V'}\,.
\label{minresidual}
\end{equation}
Recalling that the Riesz operator $R_V:V\rightarrow V'$ is an isometry defined
by
\[
\LRa{R_Vv,\delta v}=\LRp{v,\delta v}_V,\quad\forall\delta v\in V,
\]
we can use the Riesz inverse to minimize in the $V$-norm rather than its dual:
\begin{equation}
\frac{1}{2}\norm{Bu_h-l}^2_{V'}=\frac{1}{2}\norm{R_V^{-1}(Bu_h-l)}^2_V
=\frac{1}{2}\LRp{R_V^{-1}(Bu_h-l),R_V^{-1}(Bu_h-l)}_V\,.
\label{eq:rieszapplied}
\end{equation}
The first order optimality condition for \eqnref{rieszapplied} requires
the G\^ateaux derivative to be zero in all directions $\delta u \in
U_h$, i.e.,
\[
\left(R_V^{-1}(Bu_h-l),R_V^{-1}B\delta u\right)_V = 0, \quad \forall \delta u \in U.
\]
By definition of the Riesz operator, this is equivalent to
\begin{equation}
\LRa{Bu_h-l,R_V^{-1}B\delta u_h}=0\quad\forall\delta u_h\in U_h\,.
\label{eq:DPGbilinearform}
\end{equation}
Now, we can identify $v_{\delta u_h}\coloneqq R_V^{-1}B\delta u_h$ as the
optimal test function for trial function $\delta u_h$. Define $T:=R_V^{-1}B:U_h\rightarrow V$ as the trial-to-test operator. Now we can rewrite
\eqnref{DPGbilinearform} as
\begin{equation}
b(u_h,v_{\delta u_h})=l(v_{\delta u_h}).
\label{eq:DPGmethod}
\end{equation}
The DPG method then is to solve \eqnref{DPGmethod} with optimal test functions
$v_{\delta u_h}\in V$ that solve the auxiliary problem
\begin{equation}
\LRp{v_{\delta u_h},\delta v}_V=\LRa{R_Vv_{\delta u_h},\delta v}
=\LRa{B\delta u_h,\delta v}=b(\delta u_h,\delta v)\quad\forall\delta v\in V.
\label{eq:optimaltestproblem}
\end{equation}
Using a continuous test basis would result in a global solve for every optimal
test function. Therefore DPG uses a discontinuous test basis which makes each
solve element-local and much more computationally tractable. Of course,
\eqnref{optimaltestproblem} still requires the inversion of the
infinite-dimensional Riesz map, but approximating $V$ by a finite
dimensional space, $V_h$, which is of a higher polynomial degree than $U_h$ (hence
``enriched space'') works well in practice.

No assumptions have been made so far on the definition of the inner product on
$V$. In fact, proper choice of $\LRp{\cdot,\cdot}_V$ can make the difference
between a solid DPG method and one that suffers from robustness issues.


\section{Element conservative convection-diffusion}
We now proceed to develop a locally conservative formulation of DPG for
convection-diffusion type problems, but there are a few terms that we need to
define first. If $\Omega$ is our problem domain, then we can partition it into
finite elements $K$ such that
\[
\overline{\Omega} = \bigcup_K  \bar{K},\: \quad K \text { open},
\]
with corresponding {\em skeleton} $\Gamma_h$ and {\em interior
  skeleton} $\Gamma_h^0$,
\[
\Gamma_h := \bigcup_K \partial K\qquad \Gamma_h^0 := \Gamma_h - \Gamma.
\]
We define broken Sobolev spaces element-wise:
\[
\begin{array}{rl}
H^1(\Omega_h) & := \prod_K H^1(K), \\[8pt]
\bfH(\text{div},\Omega_h) & := \prod_K \bfH(\text{div},K).
\end{array}
\]
We also need the trace spaces:
\[
\begin{array}{rl}
H^\frac{1}{2}(\Gamma_h) & := \left\{ \hat{v} = \{\hat{v}_K \} \in \prod_K H^{1/2}(\partial K) \: :
\: \exists v \in H^1(\Omega) : v|_{\partial K} = \hat{v}_K \right\}, \\[8pt]
H^{-\frac{1}{2}}(\Gamma_h) & := \left\{ {\hat{\sigma}}_n = \{ {\hat{\sigma}}_{Kn} \}\in \prod_K H^{-1/2}(\partial K) \: : \: \exists \bfsigma \in \bfH(\text{div},\Omega)
: {\hat{\sigma}}_{Kn} = (\bfsigma \cdot \bfn)|_{\partial K} \right\},
\end{array}
\]
which are developed more precisely in \cite{DPGStokes}.

\subsection{Derivation}
Now that we have briefly outlined the abstract DPG method, let us apply it to
the convection-diffusion equation. The strong form of the steady
convection-diffusion problem with homogeneous Dirichlet boundary conditions reads
\[
\left\{
\begin{array}[c]{rrl}
\div(\bfbeta u)-\epsilon\Delta u & =f & \text{in }\Omega\\
u & =0 & \text{on }\Gamma\,,
\end{array}
\right.
\]
where $u$ is the property of interest, $\bs\beta$ is the convection vector,
and $f$ is the source term. Nonhomogeneous Dirichlet and Neumann boundary
conditions are straightforward but would add technicality to the following
discussion. Let us write this as an equivalent system of first
order equations:
\begin{align*}
\div(\bfbeta u-\bfsigma)&=f\\
\frac{1}{\epsilon}\bfsigma-\Grad u&=\bs0\,.
\end{align*}
If we then multiply the first equation by some scalar test function $v$ and the
bottom equation by some vector-valued test function $\bftau$, we can integrate by
parts over each element $K$:
\begin{equation}
\label{eq:preultraweak}
\begin{aligned}
-(\bfbeta u-\bfsigma,\nabla v)_K+((\bfbeta
u-\bfsigma)\cdot\mathbf{n},v)_{\partial K}&=(f,v)_K\\
\frac{1}{\epsilon}(\bfsigma,\bftau)_K+(u,\nabla\cdot\bftau)_K
-(u,\tau_n)_{\partial K}&=0\,.
\end{aligned}
\end{equation}
The discontinuous Petrov-Galerkin method refers to the fact that we are using
discontinuous optimal test functions that come from a space differing from the
trial space. It does not specify our choice of trial space. Nevertheless, many
versions of DPG in the literature (convection-diffusion \cite{DPG6},
linear elasticity \cite{BramwellDemkowiczGopalakrishnanQiu11}, linear
acoustics \cite{DemkowiczGopalakrishnanMugaZitelli12}, Stokes
\cite{DPGStokes}) associate DPG with the so-called ``ultra-weak formulation.''
We will follow the same derivation for the convection-diffusion equation, but
we emphasize that other formulations are available (in particular, the
Primal DPG\cite{PrimalDPG} method presents an alternative with
continuous trial functions). Thus, we seek field variables $u\in L^2(K)$ and
$\bfsigma\in\mathbf{L^2}(K)$. Mathematically, this leaves their traces on element
boundaries undefined, and in a manner similar to the hybridized discontinuous
Galerkin method, we define new unknowns for trace $\hat u$ and flux $\hat t$.
Applying these definitions to \eqnref{preultraweak} and adding the two
equations together, we arrive at our desired variational problem.

Find
$\bs u:=(u,\bfsigma,\hat u,\hat t)
\in\bs U:=L^2(\Omega_h)\times \bs L^2(\Omega_h)\times H^{1/2}(\Gamma_h)\times H^{-1/2}(\Gamma_h)$
such that
\begin{align}
\label{eq:variationalFormulation}
\underbrace{-(\bfbeta u-\bfsigma,\nabla v)_K+(\hat t,v)_{\partial K}
+ \frac{1}{\epsilon}(\bfsigma,\bftau)_K
+(u,\nabla\cdot\bftau)_K
-(\hat u,\tau_n)_{\partial K}}_{b(\mathbf{u}, \mathbf{v})}
&=\underbrace{\:(f,v)_K\genfrac{}{}{0pt}{}{}{}}_{l(\mathbf{v})} &\text{in }\Omega \\
\hat u&=0 &\text{on }\Gamma
\end{align}
for all $\bs v:=(v,\bftau)\in
\bs V:=H^1(\Omega_h)\times\bfH(\text{div},\Omega_h)$.

% jesse robustness section

We note that, for convection-diffusion problems, we are particularly
interested in designing a \textit{robust} DPG method.  Specifically, we are
interested in designing methods whose behavior does not change as the
diffusion parameter $\epsilon$ becomes very small.  Naive Galerkin methods for
convection-diffusion tend to suffer from a lack of robustness; specifically,
the finite element error is bounded by a constant factor of the best
approximation error, but the constant is often proportional to
$\epsilon^{-1}$.  Our aim is to design a DPG method with this in mind.  We
follow the methodology introduced by Heuer and Demkowicz in
\cite{DemkowiczHeuer}: the ultra-weak variational formulation for
convection-diffusion can be refactored as
\[
b\LRp{\LRp{u,\bfsigma,\uh,\hat t},\LRp{\bftau,v}} =
\sum_{K\in \Oh}\LRs{\LRa{\hat t,v}_{\delta K}
+\LRa{\uh,\tau_n}_{\delta K} + \LRp{u,\div \bftau
-\bfbeta\cdot\Grad v}_{L^2(K)}
+\LRp{\bfsigma,\frac{1}{\epsilon} \bftau + \Grad v}_{L^2(K)}},
\]
modulo application of boundary data.  If we choose specific
\textit{conforming} test functions satisfying the adjoint equations
\begin{align*}
\div \bftau - \bfbeta \cdot \Grad v &= u,\\
\frac{1}{\epsilon} \bftau + \Grad v &= \bfsigma,
\end{align*}
then evaluating $b\LRp{\LRp{u,\bfsigma,\uh,\fnh},\LRp{\bftau,v}}$ at these
specific test functions returns back $\norm{u}^2 + \norm{\bfsigma}^2$, the $L^2$
norm of our field variables.  Multiplying and dividing through by the test
norm $\norm{v}_V$, we have
\[
\norm{u}_{\L}^2 + \norm{\bfsigma}_{\L}^2 =
b\LRp{\LRp{u,\bfsigma,\uh,\fnh},\LRp{\bftau,v}} =
\frac{b\LRp{\LRp{u,\bfsigma,\uh,\fnh},\LRp{\bftau,v}}}{\norm{v}_V}\norm{v}_V
\leq \norm{u,\bfsigma,\uh,\fnh}_E\norm{v}_V,
\]
where
\[
\norm{u,\bfsigma,\uh,\fnh}_E = \sup_{v\in
V\setminus\LRc{0}}\frac{b\LRp{\LRp{u,\bfsigma,\uh,\fnh},\LRp{\bftau,v}}}{\norm{v}_V}
\]
is the DPG energy norm.  If we can robustly bound the test norm $\norm{v}_V
\lesssim \LRp{\norm{u}_{\L}^2+\norm{\bfsigma}^2_{\L}}^{1/2}$ (i.e. derive a
bound from above with a constant independent of $\epsilon$), then we can
divide through to get
\begin{equation}
\LRp{\norm{u}_{\L}^2 + \norm{\bfsigma}_{\L}^2}^{\frac{1}{2}} \lesssim
\norm{u,\bfsigma,\uh,\fnh}_E.
\label{eq:robustBound}
\end{equation}
In other words, the energy norm in which DPG is optimal bounds independently
of $\epsilon$ the $L^2$ norm; as we drive our energy error down to zero, we
can expect that the $L^2$ error will also decrease regardless of $\epsilon$.

We note that the construction of the test norm $\norm{v}_V$ for a robust DPG
method depends on two things: the test norm, as well as the adjoint equation.
In \cite{DemkowiczHeuer}, the standard problem with Dirichlet conditions
enforced over the entire boundary was considered; in
\cite{ChanHeuerThanhDemkowicz2012}, boundary conditions were chosen for the
forward problem such that the induced adjoint problem was regularized and
contained no strong boundary layers, allowing for the construction of a
stronger test norm on $V$.  We adopt a slight modification of the test norm
introduced in \cite{ChanHeuerThanhDemkowicz2012} for numerical experiments
here, which is motivated and explained in more detail in
% Section~\secref{sec:confusionPlate}.


Having reviewed and laid the foundation for DPG methods, we can now formulate our conservative DPG scheme.  % added by Jesse
Let $\bs U_h:=U_h\times\bs S_h\times\hat U_h\times\hat F_h\subset L^2(\Omega_h)\times\bs
L^2(\Omega_h)\times H^{\frac{1}{2}}(\Gamma_h)\times H^{-\frac{1}{2}}(\Gamma_h)$
be a finite-dimensional subspace, and let $\bs u_h:=(u_h.\bfsigma_h,\hat
u_h\hat t_h)\in\bs U_h$ be the group variable. The element conservative DPG scheme is
derived from the Lagrangian:
\begin{equation}
L(\bs u_h,\lambda_k)=\frac{1}{2}\norm{R_V^{-1}(b(\bs
u_h,\cdot)-(f,\cdot))}^2_{\bs V}-\sum_K\lambda_K(b(\bs u_h,(1_K,\bs0))-l((1_K,\bs0)))\,,
\label{eq:lagrangian}
\end{equation}
where $(1_K,\bs0)$ is the test function in which $v=1$ on element $K$ and 0 elsewhere and $\bftau=\bs0$ everywhere.

Taking the G\^ateaux derivatives as before, we arrive at the following system
of equations:
\begin{equation}
\left\{
\begin{array}[c]{rll}
b(\bs u_h,T(\delta\bs u_h))-\sum_K\lambda_K b(\bs u_h,(1_K,\bs0))
&=l(T(\delta\bs u_h)) & \forall\delta\bs u_h\in\bs U_h\\
b(\bs u_h,(1_K,\bs0)) &=l((1_K,\bs0)) & \forall K\,,
\end{array}
\right.
\label{eq:conservativeSystem}
\end{equation}
where $T:=R_V^{-1}B:\bs U_h\rightarrow\bs V$ is the same trial-to-test operator as in the original formulation.

Denote $T(\delta\bfu_h)=\LRp{\vdeltau,\taudeltau}\in H^1(\Omega_h)\times\bfH(div,\Omega_h)$.
Then, putting \eqref{eq:conservativeSystem} into more concrete terms for
convection-diffusion, we get:
\begin{equation}
\left\{
\begin{array}[c]{rll}
-(\bfbeta u-\bfsigma,\nabla \vdeltau)+\langle\hat t,\vdeltau\rangle
+ \frac{1}{\epsilon}(\bfsigma,\taudeltau)
+(u,\nabla\cdot\taudeltau)
-\langle\hat u,\taudeltau\cdot\bs n\rangle\\
-\sum_K\lambda_K (\delta\hat t,(1_K,\bs0))
&=(f,\vdeltau) & \forall\delta\bs u_h\in\bs U_h\\
\langle\hat t,(1_K,\bs0)\rangle &=(f,1_K) & \forall K\,.
\end{array}
\right.
\label{eq:conservativeConfusionSystem}
\end{equation}

\subsection{Stability analysis}
In the following analysis, we neglect the error due to the approximation of optimal test functions.
We follow the classical Brezzi's theory \cite{Brezzi1974,DBB05} for an abstract
mixed problem:
\begin{equation}
\left\{
\begin{array}{lll}
\bfu \in \bfU, p \in Q\\
a(\bfu,\bfw) + c(p,\bfw) & = l(\bfw) & \forall \bfw \in \bfU \\
c(q,\bfu) & = g(q) & \forall q \in Q
\end{array}
\right.
\end{equation}
where $\bfU,Q$ are Hilbert spaces, and $a,c,l,g$ denote the appropriate
bilinear and linear forms. Note that $a(\bfu,\bfw)=b(\bfu,T\bfw)=(T\bfu,T\bfw)_V$ in
the notation from the previous section.

Let function $\bfpsi$ denote the $\bfH(\text{div},\Omega)$ extension of flux $\hat{t}$
that realizes the minimum in the definition of the quotient (minimum energy
extension) norm.
The choice of norm for the Lagrange multipliers $\lambda_K$ is implied
by the quotient norm used for $H^{-1/2}(\Gamma_h)$ and continuity
bound for form $c(p,\bfw)$ representing the constraint:
\begin{equation}
\begin{array}{lll}
| c(\sum_K \lambda_K (1_K,{\bf 0}),(u,\bfsigma,\hat{u},\hat{t})) |
& = | \sum_K \lambda_K \langle \hat{t}, 1_K \rangle_{\partial K} | \\[8pt]
& = | \sum_K \lambda_K \langle v_n , 1_K \rangle_{\partial K} | \\[8pt]
& = | \sum_K \lambda_K \int_K \text{div} \bfpsi \: 1_K  | \\[8pt]
& \leq \sum_K  \lambda_K || \text{div} \bfpsi ||_{L^2(K)} \mu(K)^{1/2} \\[8pt]
& \leq (\sum_K \mu(K) \lambda_K^2)^{1/2} \: (\sum_K || \text{div} \bfpsi ||_{L^2(K)}^2 )^{1/2} \\[8pt]
& \leq \underbrace{\left(\sum_K \mu(K) \lambda_K^2\right)^{1/2}}_{=: || \bflambda ||} ||
\hat{t} ||_{H^{-1/2}(\Gamma_h)}\,,
\end{array}
\end{equation}
where $\mu(K)$ stands for the area (measure) of element $K$.
We proceed now with the discussion of the discrete inf-sup stability constants. We skip
index $h$ in the notation.

\paragraph{Inf Sup Condition} relating spaces $\bfU$ and $Q$ reads as follows:
\begin{equation}
   \sup_{\bfw \in \bfU} \frac{| c(p,\bfw) |}{|| \bfw ||_{\bfU}} \geq \beta ||
   p ||_Q\,.
\end{equation}
Let
\begin{equation}
R\, : \, L^2(\Omega) \ni q \to \bfpsi \in \bfH(\text{div},\Omega) \cap \bfH^1(\Omega)
=\bfH^1(\Omega)
\end{equation}
be
the continuous right inverse of the divergence operator constructed by
Costabel and McIntosh in \cite{CostabelMcIntosh}.
Let $\bfpsi_h$ denote the classical, lowest order Raviart-Thomas (RT) interpolant of
function
\begin{equation}
\bfpsi = R (\sum_K \lambda_K 1_K) \: .
\end{equation}
Note that $\text{div} \bfpsi_h = \text{div} \bfpsi = \lambda_K$ in element $K$.

Classical $h$-interpolation interpolation error estimates for the lowest error
Raviart-Thomas elements and continuity of operator $R$ imply the stability estimate:
\begin{equation}
\begin{array}{lll}
|| \bfpsi_h || & \leq || \bfpsi_h - \bfpsi || + || \bfpsi ||\\[8pt]
&\leq C h || \bfpsi ||_{H^1} +  || \bfpsi || \\[8pt]
& \leq C || \text{div} \bfpsi || = C (\sum_K \mu(K) \lambda_K^2)^{1/2}\,.
\end{array}
\label{eq:stab}
\end{equation}
Above, $C$ is a generic, mesh independent constant incorporating constant from
the interpolation error estimate and continuity constant of $R$.
Let $\hat{t}$ be the trace of $\bfpsi_h$. We have then,
\begin{equation}
\sup_{\hat{t} \in H^{-1/2}(\Gamma_h)} \frac{|  \sum_K \lambda_K \langle
\hat{t},1_K \rangle_{\partial K} |}{|| \hat{t} ||_{H^{-1/2}(\Gamma_h)}}
\geq \frac{| \sum_K \lambda_K \int_K \text{div} \bfpsi_h \: 1_K  |}
{|| \bfpsi_h ||_{H(\text{div},\Omega)}}
\geq \frac{1}{C} (\sum_K \mu(K) \lambda_K^2)^{1/2}\,,
\end{equation}
where $C$ is the constant from stability estimate~\eqref{eq:stab}.

Notice that we have considered traces of lowest order Raviart-Thomas elements
for the discretization of flux $\hat{t}$. The inf-sup condition for the lowest
order RT spaces implies automatically the analogous condition for elements of
arbitrary order; increasing the dimension of space $U$ only makes the discrete
inf-sup constant bigger.

\paragraph{Inf Sup in Kernel Condition} is satisfied automatically due to the use of optimal
test functions. First of all, we characterize the ``kernel'' space:
\begin{equation}
\begin{array}{rl}
\bfU_0  := & \{ \bfw \in \bfU \, : \, c(q,\bfw) = 0 \quad \forall q \in Q\} \\[8pt]
 = &\{ (u,\bfsigma,\hat{u},\hat{t}) \, : \, \langle \hat{t},1_K \rangle = 0
 \quad \forall K \}\,.
\end{array}
\end{equation}

In other words, the kernel space contains only the equilibrated fluxes.
With $\bfu \in \bfU_0$, we have then:
\begin{equation}
   \sup_{\bfw \in \bfU_0} \frac{| a(\bfu,\bfw) |}{|| \bfw ||_{\bfU} }
   \geq \frac{| b(\bfu,T \bfu) |}{|| \bfu ||}
   = \frac{| b(\bfu,T \bfu) |}{|| T\bfu ||}\frac{||T\bfu||}{||\bfu||}
   = \sup_{(v,\bftau)}\frac{| b((u,\bfsigma,\hat{u},\hat{t}), (v,\bftau)) |}{|| (v,\bftau) ||}
   \frac{||T\bfu||}{||\bfu||}
   \geq \gamma^2 || (u,\bfsigma,\hat{u},\hat{t}) ||\,,
\end{equation}
where $\gamma$ is the stability constant for the standard continuous DPG formulation.
The first inequality follows as we plug in the definition for $a$ and pick
$\bfw=\bfu$. The second equality is trivial, while the next one follows by definition of the optimal test
functions given through the trial-to-test operator $T$. The finally inequality
springs from the fact that
$\sup_{\bfv}\frac{|b(\bfu,\bfv)|}{||\bfv||}\geq\gamma||\bfu||$ and
$||T\bfu||_V=||R_V^{-1}B\bfu||_V=||B\bfu||_{V'}\geq\gamma||u||$.

With both discrete inf-sup constants in place, we have the standard result: the FE error
is bounded by the best approximation error. Notice that the exact Lagrange multipliers
are zero, so the best approximation error involves only solution $(u,\bfsigma,\hat{u},\hat{t})$.
\subsubsection{Robustness analysis}

\subsection{Robust test norms}
\subsubsection{A model problem}
\subsubsection{A modification of the robust test norm}
\subsubsection{Adaptation for a locally conservative formulation}
\subsubsection{Proof of robust stability estimate}


\section{Application to other fluid model problems}

\subsection{Inviscid Burgers' equation}

\subsection{Stokes flow}


\section{Numerical Experiments}

\subsection{Erickson-Johnson model problem}

\subsection{Vortex problem}

\subsection{Discontinuous source problem}

\subsection{Inviscid Burgers' problem}

\subsection{Stokes flow around a cylinder}

\subsection{Stokes flow over a backward facing step}



\chapter{Space-time DPG}



\section{Heat equation}

\subsection{Derivation}

\subsection{Problems considered}


\section{Convection-diffusion}

\subsection{Derivation}

\subsection{Problems considered}


\section{Burgers'}

\subsection{Derivation}

\subsection{Problems considered}


\section{Time-slabs vs full-time}



\chapter{Proposed work}



\section{Space-time DPG for 2D Incompressible Navier-Stokes}

\subsection{Derivation}

\subsection{Problems to consider}


\section{Space-time DPG for 2D Compressible Navier-Stokes}

\subsection{Derivation}

\subsection{Problems to consider}


\section{Area requirements}

\subsection{Area A: Applicable mathematics}
\subsubsection*{Completed:}
\subsubsection*{Proposed:}

\subsection{Area B: Numerical analysis and scientific computation}
\subsubsection*{Completed: Collaborative work with Nathan Robers on high order parallel adaptive DPG code Camellia}
Previous work focused on enabling locally conservative computations with Camellia. My work to this point has primarily been on the application side -- implementing new test problems and exploring how DPG (and Camellia) perform. I have also been instrumental in adding new features to facilitate these tests. I implemented mesh readers to read the GMSH and Triangle mesh formats to enable computations on nontrivial domains. I also wrote output code to interface Camellia with the VTK library, allowing us to visualize our results.
 \subsubsection*{Proposed: Continued development of Camellia with emphasis on enabling space-time DPG}
Since the proposed work is largely exploratory, the emphasis is not so much on high performance computing, algorithms, and optimization, but more on discovering whether the space-time DPG technology holds promise for the future. We focus on extending the adaptive nature of previous DPG work with local space-time adaptivity.

We are also interested in adding many auxiliary featurs to Camellia in the process. Solution output is currently done completely in serial, but parallel output is a desired feature as we move forward. We are also interested in removing our VTK dependency and switching to a new IO format using HDF5 and XDMF.

\subsection{Area C: Mathematical modeling and applications}
\subsubsection*{Completed:}
\subsubsection*{Proposed:}

\bibliographystyle{plain}
\bibliography{../DPG}

\end{document}
