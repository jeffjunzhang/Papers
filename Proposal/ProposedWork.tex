% -*- root: Proposal.tex -*-
\documentclass[Proposal.tex]{subfiles} 
\begin{document}
\chapter{Proposed work}
% In this chapter we outline the proposed work of solving the Navier-Stokes equations with space-time DPG. 
% Our emphasis will be on incompressible Navier-Stokes, but time permitting, we would like to investigate the compressible equations as well.

% %    /$$$$$$                                                                            /$$ /$$       /$$          
% %   /$$__  $$                                                                          |__/| $$      | $$          
% %  | $$  \__/  /$$$$$$  /$$$$$$/$$$$   /$$$$$$   /$$$$$$   /$$$$$$   /$$$$$$$  /$$$$$$$ /$$| $$$$$$$ | $$  /$$$$$$ 
% %  | $$       /$$__  $$| $$_  $$_  $$ /$$__  $$ /$$__  $$ /$$__  $$ /$$_____/ /$$_____/| $$| $$__  $$| $$ /$$__  $$
% %  | $$      | $$  \ $$| $$ \ $$ \ $$| $$  \ $$| $$  \__/| $$$$$$$$|  $$$$$$ |  $$$$$$ | $$| $$  \ $$| $$| $$$$$$$$
% %  | $$    $$| $$  | $$| $$ | $$ | $$| $$  | $$| $$      | $$_____/ \____  $$ \____  $$| $$| $$  | $$| $$| $$_____/
% %  |  $$$$$$/|  $$$$$$/| $$ | $$ | $$| $$$$$$$/| $$      |  $$$$$$$ /$$$$$$$/ /$$$$$$$/| $$| $$$$$$$/| $$|  $$$$$$$
% %   \______/  \______/ |__/ |__/ |__/| $$____/ |__/       \_______/|_______/ |_______/ |__/|_______/ |__/ \_______/
% %                                    | $$                                                                          
% %                                    | $$                                                                          
% %                                    |__/  
% \section{Space-time DPG for Compressible Navier-Stokes}
% For the sake of narrative, we start with the full compressible Navier-Stokes equations and then simplify to the incompressible case.
% The compressible Navier-Stokes equations are
% \begin{align}
% \frac{\partial}{\partial t}\svectthree{\rho}{\rho\bfu}{\rho e_0}
% +\Div\svectthree{\rho\bfu}{\rho\bfu\otimes\bfu+p\bfI-\mathbb{D}}{\rho\bfu e_0+\bfu p+\bfq-\bfu\cdot\mathbb{D}}
% %TODO: Possible error above. cfd-online seems to have T^T
% =\svectthree{f_c}{\bff_m}{f_e}\,,
% \end{align}
% where $\rho$ is the density, $\bfu$ is the velocity, $p$ is the pressure, $\bfI$ is the identity matrix,
% $\mathbb{D}$ is the deviatoric stress tensor or viscous stress, $e_0$ is the total energy, $\bfq$ is the heat flux, 
% and $f_c$, $\bff_m$, and $f_e$ are the source terms for the continuity, momentum, and energy equations, respectively.
% Assuming Stokes hypothesis that $\lambda=-\frac{2}{3}\mu$, 
% \begin{equation*}
% 	\mathbb{D}=2\mu\bfS^*=2\mu\LRs{\frac{1}{2}\LRp{\Grad\bfu+\LRp{\Grad\bfu}^T}-\frac{1}{3}\Div\bfu\bfI}\,,
% \end{equation*}
% where $\bfS^*$ is the trace-less viscous strain rate tensor.
% The heat flux is given by Fourier's law:
% \begin{equation*}
% 	\bfq=-C_p\frac{\mu}{Pr}\Grad T\,,
% \end{equation*}
% where $C_p$ is the specific heat at constant pressure and $Pr$ is the laminar Prandtl number: $Pr:=\frac{C_p\mu}{\lambda}$.
% We need to close these equations with an equation of state. An ideal gas assumption gives
% \begin{equation*}
% 	\gamma:=\frac{C_p}{C_v}\,,\quad p=\rho RT\,,\quad e=C_v T\,,\quad C_p-C_v=R\,,
% \end{equation*}
% where $\gamma$ is the ratio of specific heats, $C_v$ is the specific heat at constant volume, $R$ is the gas constant,
% $e$ is the internal energy, $T$ is the temperature,
% and $\gamma$, $C_p$, $C_v$, and $R$ are constant properties of the fluid.
% The total energy is defined by
% \begin{equation*}
% 	e_0=e+\frac{1}{2}\bfu\cdot\bfu\,.
% \end{equation*}

% We can write our first order system of equations in space-time as follows:
% \begin{subequations}
% \label{eq:compressibleNSFirstOrder}
% \begin{align}
% 	\mathbb{D}-\mu\LRp{\Grad\bfu+\LRp{\Grad\bfu}^T}+\frac{2\mu}{3}\Div\bfu\bfI&=0\\
% 	\bfq+C_p\frac{\mu}{Pr}\Grad T&=0\\
% 	\Divxt\vecttwo{\rho\bfu}{\rho}&=f_c\\
% 	\Divxt\vecttwo{\rho\bfu\otimes\bfu+\rho RT\bfI-\mathbb{D}}{\rho\bfu}&=\bff_m\\
% 	\Divxt\vecttwo{\rho\bfu\LRp{C_v T+\frac{1}{2}\bfu\cdot\bfu}+\bfu\rho RT+\bfq-\bfu\cdot\mathbb{D}}{\rho\LRp{C_v T+\frac{1}{2}\bfu\cdot\bfu}}&=f_e\,,
% \end{align}
% \end{subequations}
% where our solution variables are $\rho$, $\bfu$, $T$, $\mathbb{D}$, and $\bfq$.

% %   /$$$$$$$                      /$$                        /$$     /$$                    
% %  | $$__  $$                    |__/                       | $$    |__/                    
% %  | $$  \ $$  /$$$$$$   /$$$$$$  /$$ /$$    /$$  /$$$$$$  /$$$$$$   /$$  /$$$$$$  /$$$$$$$ 
% %  | $$  | $$ /$$__  $$ /$$__  $$| $$|  $$  /$$/ |____  $$|_  $$_/  | $$ /$$__  $$| $$__  $$
% %  | $$  | $$| $$$$$$$$| $$  \__/| $$ \  $$/$$/   /$$$$$$$  | $$    | $$| $$  \ $$| $$  \ $$
% %  | $$  | $$| $$_____/| $$      | $$  \  $$$/   /$$__  $$  | $$ /$$| $$| $$  | $$| $$  | $$
% %  | $$$$$$$/|  $$$$$$$| $$      | $$   \  $/   |  $$$$$$$  |  $$$$/| $$|  $$$$$$/| $$  | $$
% %  |_______/  \_______/|__/      |__/    \_/     \_______/   \___/  |__/ \______/ |__/  |__/
% %                                                                                           
% %                                                                                           
% %              
% \subsection{Derivation of Space-Time DPG Formulation}
% We start with \eqref{eq:compressibleNSFirstOrder} and multiply by test functions $\mathbb{S}$ (symmetric tensor), $\bftau$, $v_c$, $\bfv_m$, $v_e$, 
% then integrate by parts over each space-time element $K$:
% \begin{subequations}
% \label{eq:compressibleNSBF}
% \begin{align}
% 	\LRp{\mathbb{D},\mathbb{S}}+\LRp{2\mu\bfu,\Div\mathbb{S}}-\LRp{\frac{2\mu}{3}\bfu,\Grad\trace{\mathbb{S}}}
% 	-\LRa{2\mu\hat\bfu,\mathbb{S}\bfn_x}+\LRa{\frac{2\mu}{3}\hat\bfu,\mathbb{S}\bfn_x}&=0\\
% 	\LRp{\bfq,\bftau}-\LRp{C_p\frac{\mu}{Pr}T,\Div\bftau}+\LRa{C_p\frac{\mu}{Pr}\hat T,\tau_n}&=0\\
% 	-\LRp{\vecttwo{\rho\bfu}{\rho},\Gradxt v_c}+\LRa{\hat t_c,v_c}&=\LRp{f_c,v_c}\\
% 	-\LRp{\vecttwo{\rho\bfu\otimes\bfu+\rho RT\bfI-\mathbb{D}}{\rho\bfu},\Gradxt\bfv_m}+\LRa{\hat\bft_m,\bfv_m}&=\LRp{\bff_m,\bfv_m}\\
% 	-\LRp{\vecttwo{\rho\bfu\LRp{C_v T+\frac{1}{2}\bfu\cdot\bfu}+\bfu\rho RT+\bfq-\bfu\cdot\mathbb{D}}{\rho\LRp{C_v T+\frac{1}{2}\bfu\cdot\bfu}},\Gradxt v_e}
% 	+\LRa{\hat t_e,v_e}&=\LRp{f_e,v_e}\,,
% \end{align}
% \end{subequations}
% where 
% \begin{equation*}
% \begin{aligned}
% \hat\bfu&=\trace(\bfu)\\
% \hat T&=\trace(T)\\
% \hat t_c&=\trace\LRp{\rho\bfu}\cdot\bfn_x
% +\trace\LRp{\rho}n_t\\
% \hat\bft_m&=\trace\LRp{\rho\bfu\otimes\bfu+\rho RT\bfI-\mathbb{D}}\cdot\bfn_x
% +\trace\LRp{\rho\bfu} n_t\\
% \hat t_e&=\trace\LRp{\rho\bfu\LRp{C_v T+\frac{1}{2}\bfu\cdot\bfu}+\bfu\rho RT+\bfq-\bfu\cdot\mathbb{D}}\cdot\bfn_x
% +\trace\LRp{\rho\LRp{C_v T+\frac{1}{2}\bfu\cdot\bfu}}n_t\,.
% \end{aligned}
% \end{equation*}
% Note that integrating $\mathbb{S}$ against the symmetric gradient only picks up the symmetric part.
% This is a much more complicated system of equations than we had for the space-time heat equation, but the situation has many similarities.
% Test function $\bftau\in\HdivK$ where the divergence is taken only over spatial dimensions, $v_c,v_e\in\HOneK$, and $\bfv_m\in\HOneVecK$.
% These are all familiar spaces from our work with the heat equation.
% Unfortunately, $\mathbb{S}$ has some weird requirements: each $d\times d$ components must be at least in $L^2(K)$, $\Div\mathbb{S}\in\LVecK$, and
% $\Grad\trace{\mathbb{S}}\in\LVecK$.
% In practice, we will probably just seek each component in $\HOneK$.

% %   /$$       /$$                                         /$$                       /$$     /$$                    
% %  | $$      |__/                                        |__/                      | $$    |__/                    
% %  | $$       /$$ /$$$$$$$   /$$$$$$   /$$$$$$   /$$$$$$  /$$ /$$$$$$$$  /$$$$$$  /$$$$$$   /$$  /$$$$$$  /$$$$$$$ 
% %  | $$      | $$| $$__  $$ /$$__  $$ |____  $$ /$$__  $$| $$|____ /$$/ |____  $$|_  $$_/  | $$ /$$__  $$| $$__  $$
% %  | $$      | $$| $$  \ $$| $$$$$$$$  /$$$$$$$| $$  \__/| $$   /$$$$/   /$$$$$$$  | $$    | $$| $$  \ $$| $$  \ $$
% %  | $$      | $$| $$  | $$| $$_____/ /$$__  $$| $$      | $$  /$$__/   /$$__  $$  | $$ /$$| $$| $$  | $$| $$  | $$
% %  | $$$$$$$$| $$| $$  | $$|  $$$$$$$|  $$$$$$$| $$      | $$ /$$$$$$$$|  $$$$$$$  |  $$$$/| $$|  $$$$$$/| $$  | $$
% %  |________/|__/|__/  |__/ \_______/ \_______/|__/      |__/|________/ \_______/   \___/  |__/ \______/ |__/  |__/
% %                                                                                                                  
% %                                                                                                                  
% %   
% \subsubsection{Linearization}
% We follow the same linearization process as we have for previous problems.
% Let $U=\LRc{\rho,\bfu,T,\mathbb{D},\bfq,\hat\bfu,\hat e,\hat t_c,\hat\bft_m,\hat t_e}$ be a group solution variable which we can decompose into two parts:
% $U:=\tilde U+\Delta U$, where
% $\tilde U = \LRc{\tilde\rho,\tilde\bfu,\tilde T,\tilde{\mathbb{D}},\bs0,\bs0,0,0,\bs0,0}$ is the previous iteration approximation, 
% and $\Delta U=\LRc{\Delta\rho,\Delta\bfu,\Delta T,\Delta\mathbb{D},\bfq,\hat\bfu,\hat e,\hat t_c,\hat\bft_m,\hat t_e}$ is the update.
% Note that $\tilde U$ only contains terms which participate in nonlinearities in \eqref{eq:compressibleNSBF} 
% while $\Delta U$ contains the full linear terms and the updates to the nonlinear terms.
% Also, we drop the $\Delta$ and $\tilde\cdot$ notation for linear terms.
% Define residual $R(U)$ as the left hand side of \eqref{eq:compressibleNSBF} minus the right hand side.
% % Let $\tilde U$ be an approximate solution for the minimization of the residual. 
% % We wish to solve for an increment $\Delta U$ such that $U=\tilde U+\Delta U$ is a better approximation of the true solution.
% Approximating $R(U)=0$ by $R(\tilde U)+R'(\tilde U)\Delta U=0$, where $R'(\tilde U)$ is the Jacobian of $R$ evaluated at $\tilde U$, we get a linear system:
% \begin{equation}
% 	R'(\tilde U)\Delta U=-R(\tilde U)\,.
% \end{equation}
% We only need to define our Jacobian and residual for each component of \eqref{eq:compressibleNSBF}. 
% % Note that we can exclude our traces and fluxes from the residual since these unknowns are only involved in linear terms.
% % As such, we don't update these variables each iteration based on the previous iteration.
% % Instead, for example, we consider $\tilde{\hat\bfu}=0$, and let $\hat\bfu=\Delta\hat\bfu$ after each iteration.
% % To indicate this different treatment, we drop the delta and tilde notation on these variables in the following discussion.
% The Jacobian of our compressible Navier-Stokes system, $R'(\tilde U)\Delta U$ is
% \begin{equation}
% \label{eq:compressibleJacobian}
% \begin{aligned}
% 	&\LRp{\Delta\mathbb{D},\mathbb{S}}+\LRp{2\mu\Delta\bfu,\Div\mathbb{S}}-\LRp{\frac{2\mu}{3}\Delta\bfu,\Grad\trace{\mathbb{S}}}
% 	-\LRa{2\mu\hat\bfu,\mathbb{S}\bfn_x}+\LRa{\frac{2\mu}{3}\hat\bfu,\mathbb{S}\bfn_x}\\
% 	%
% 	&+\LRp{\bfq,\bftau}-\LRp{C_p\frac{\mu}{Pr}\Delta T,\Div\bftau}+\LRa{C_p\frac{\mu}{Pr}\hat T,\tau_n}\\
% 	%
% 	&-\LRp{\vecttwo{\Delta\rho\tilde\bfu+\tilde\rho\Delta\bfu}
% 	{\Delta\rho},\Gradxt v_c}
% 	+\LRa{\hat t_c,v_c}\\
% 	%
% 	&-\LRp{\vecttwo{\Delta\rho\tilde\bfu\otimes\tilde\bfu+\tilde\rho\Delta\bfu\otimes\tilde\bfu+\tilde\rho\tilde\bfu\otimes\Delta\bfu
% 	+\LRp{\Delta\rho R\tilde T+\tilde\rho R\Delta T}\bfI-\Delta\mathbb{D}}
% 	{\Delta\rho\tilde\bfu+\tilde\rho\Delta\bfu},\Gradxt\bfv_m}
% 	+\LRa{\hat\bft_m,\bfv_m}\\
% 	%
% 	&-\LRp{\vectthree{[C_v\Delta\rho\tilde T\tilde\bfu+C_v\tilde\rho\Delta T\tilde\bfu+C_v\tilde\rho\tilde T\Delta\bfu
% 	+\frac{1}{2}\LRp{\Delta\rho\tilde\bfu\cdot\tilde\bfu\tilde\bfu+\tilde\rho\Delta\bfu\cdot\tilde\bfu\tilde\bfu
% 	+\tilde\rho\tilde\bfu\cdot\Delta\bfu\tilde\bfu+\tilde\rho\tilde\bfu\cdot\tilde\bfu\Delta\bfu}}
% 	{+R\LRp{\Delta\rho\tilde T\tilde\bfu+\tilde\rho\Delta T\tilde\bfu+\tilde\rho\tilde T\Delta\bfu}
% 	+\bfq-\Delta\bfu\cdot\tilde{\mathbb{D}}-\tilde\bfu\cdot\Delta\mathbb{D}]}
% 	{C_v\Delta\rho\tilde T+C_v\tilde\rho\Delta T
% 	+\frac{1}{2}\LRp{\Delta\rho\tilde\bfu\cdot\tilde\bfu+\tilde\rho\Delta\bfu\cdot\tilde\bfu+\tilde\rho\tilde\bfu\cdot\Delta\bfu}},\Gradxt v_e}\\
% 	&+\LRa{\hat t_e,v_e}\,.
% \end{aligned}
% \end{equation}
% The residual, $R(\tilde U)$, is then
% \begin{equation}
% \begin{aligned}
% 	&\LRp{\tilde{\mathbb{D}},\mathbb{S}}+\LRp{2\mu\tilde\bfu,\Div\mathbb{S}}-\LRp{\frac{2\mu}{3}\tilde\bfu,\Grad\trace{\mathbb{S}}}\\
% 	&-\LRp{C_p\frac{\mu}{Pr}\tilde T,\Div\bftau}\\
% 	&-\LRp{\vecttwo{\tilde\rho\tilde\bfu}{\tilde\rho},\Gradxt v_c}-\LRp{f_c,v_c}\\
% 	&-\LRp{\vecttwo{\tilde\rho\tilde\bfu\otimes\tilde\bfu+\tilde\rho R\tilde T\bfI-\tilde{\mathbb{D}}}{\tilde\rho\tilde\bfu},
% 	\Gradxt\bfv_m}-\LRp{\bff_m,\bfv_m}\\
% 	&-\LRp{\vecttwo{\tilde\rho\tilde\bfu\LRp{C_v\tilde T+\frac{1}{2}\tilde\bfu\cdot\tilde\bfu}+\tilde\bfu\tilde\rho R\tilde T
% 	-\tilde\bfu\cdot\tilde{\mathbb{D}}}{\tilde\rho\LRp{C_v\tilde T+\frac{1}{2}\tilde\bfu\cdot\tilde\bfu}},
% 	\Gradxt v_e}-\LRp{f_e,v_e}\,.
% \end{aligned}
% \end{equation}

% %   /$$$$$$$$                       /$$           /$$   /$$                                  
% %  |__  $$__/                      | $$          | $$$ | $$                                  
% %     | $$     /$$$$$$   /$$$$$$$ /$$$$$$        | $$$$| $$  /$$$$$$   /$$$$$$  /$$$$$$/$$$$ 
% %     | $$    /$$__  $$ /$$_____/|_  $$_/        | $$ $$ $$ /$$__  $$ /$$__  $$| $$_  $$_  $$
% %     | $$   | $$$$$$$$|  $$$$$$   | $$          | $$  $$$$| $$  \ $$| $$  \__/| $$ \ $$ \ $$
% %     | $$   | $$_____/ \____  $$  | $$ /$$      | $$\  $$$| $$  | $$| $$      | $$ | $$ | $$
% %     | $$   |  $$$$$$$ /$$$$$$$/  |  $$$$/      | $$ \  $$|  $$$$$$/| $$      | $$ | $$ | $$
% %     |__/    \_______/|_______/    \___/        |__/  \__/ \______/ |__/      |__/ |__/ |__/
% %                                                                                            
% %                                                                                            
% %      
% \subsubsection{Test Norm}
% % We start by multiplying the mass, momentum and energy equations in \eqref{eq:compressibleJacobian} by -1. 
% We group terms in \eqref{eq:compressibleJacobian} by trial variable to get
% \begin{equation}
% \begin{aligned}
% &\LRp{\Delta\mathbb{D},\mathbb{S}+\Grad\bfv_m+\Grad v_e\otimes\tilde\bfu}\\
% +&\LRp{\bfq,\bftau-\Grad v_e}\\
% +&\left(\Delta\rho,-\tilde\bfu\cdot\Grad v_c-\frac{\partial v_c}{\partial t}
% -\tilde\bfu\otimes\tilde\bfu:\Grad\bfv_m
% -R\tilde T\Div\bfv_m-\tilde\bfu\cdot\frac{\partial\bfv_m}{\partial t}\right.\\
% &\left.-C_v\tilde T\tilde\bfu\cdot\Grad v_e-\frac{1}{2}\tilde\bfu\cdot\tilde\bfu\tilde\bfu\cdot\Grad v_e
% -R\tilde T\tilde\bfu\Grad v_e
% -C_v\tilde T\frac{\partial v_e}{\partial t}-\frac{1}{2}\tilde\bfu\cdot\tilde\bfu\frac{\partial v_e}{\partial t}
% \right)\\
% +&\left(\Delta\bfu,2\mu\Div\mathbb{S}-\frac{2\mu}{3}\Grad\trace{\mathbb{S}}
% -\tilde\rho\Grad v_c
% -\tilde\rho\tilde\bfu\cdot\Grad\bfv_m-\tilde\rho\Grad\bfv_m\cdot\tilde\bfu
% -\tilde\rho\frac{\partial\bfv_m}{\partial t}
% -C_v\tilde\rho\tilde T\Grad v_e\right.\\
% &\left.
% -\frac{1}{2}\tilde\rho\tilde\bfu\cdot\tilde\bfu\Grad v_e
% -\frac{1}{2}\tilde\rho\tilde\bfu\cdot\Grad v_e\tilde\bfu
% -\frac{1}{2}\tilde\rho\Grad v_e\cdot\tilde\bfu\tilde\bfu
% -R\tilde\rho\tilde T_\Grad v_e
% +\tilde{\mathbb{D}}\cdot\Grad v_e
% -\frac{1}{2}\tilde\rho\tilde\bfu\frac{\partial v_e}{\partial t}
% -\frac{1}{2}\tilde\rho\tilde\bfu\frac{\partial v_e}{\partial t}
% \right)\\
% +&\left(\Delta T,-C_p\frac{\mu}{Pr}\Div\bftau
% -R\tilde\rho\Div\bfv_m
% -C_v\tilde\rho\tilde\bfu\Grad v_e
% -R\tilde\rho\tilde\bfu\Grad v_e
% -C_v\tilde\rho\frac{\partial v_e}{\partial t}
% \right)\\
% +&\LRp{\hat\bfu,-2\mu\mathbb{S}\bfn_x+\frac{2\mu}{3}\mathbb{S}\bfn_x}\\
% +&\LRp{\hat T,C_p\frac{\mu}{Pr}\tau_n}\\
% +&\LRp{\hat t_c,v_c}\\
% +&\LRp{\hat\bft_m,\bfv_m}\\
% +&\LRp{\hat t_e,v_e}\,.
% \end{aligned}	
% \end{equation}

% Then the graph norm would be defined by
% \begin{equation}
% \begin{aligned}
% &\norm{\mathbb{S}+\Grad\bfv_m+\Grad v_e\otimes\tilde\bfu}^2\\
% +&\norm{\bftau-\Grad v_e}^2\\
% +&\left\|
% -\tilde\bfu\cdot\Grad v_c-\frac{\partial v_c}{\partial t}-\tilde\bfu\otimes\tilde\bfu:\Grad\bfv_m
% -R\tilde T\Div\bfv_m-\tilde\bfu\cdot\frac{\partial\bfv_m}{\partial t}\right.\\
% &\left.-C_v\tilde T\tilde\bfu\cdot\Grad v_e-\frac{1}{2}\tilde\bfu\cdot\tilde\bfu\tilde\bfu\cdot\Grad v_e
% -R\tilde T\tilde\bfu\Grad v_e
% -C_v\tilde T\frac{\partial v_e}{\partial t}-\frac{1}{2}\tilde\bfu\cdot\tilde\bfu\frac{\partial v_e}{\partial t}
% \right\|^2\\
% +&\left\|2\mu\Div\mathbb{S}-\frac{2\mu}{3}\Grad\trace{\mathbb{S}}
% -\tilde\rho\Grad v_c
% -\tilde\rho\tilde\bfu\cdot\Grad\bfv_m-\tilde\rho\Grad\bfv_m\cdot\tilde\bfu
% -\tilde\rho\frac{\partial\bfv_m}{\partial t}
% -C_v\tilde\rho\tilde T\Grad v_e\right.\\
% &\left.
% -\frac{1}{2}\tilde\rho\tilde\bfu\cdot\tilde\bfu\Grad v_e
% -\frac{1}{2}\tilde\rho\tilde\bfu\cdot\Grad v_e\tilde\bfu
% -\frac{1}{2}\tilde\rho\Grad v_e\cdot\tilde\bfu\tilde\bfu
% -R\tilde\rho\tilde T\Grad v_e
% +\tilde{\mathbb{D}}\cdot\Grad v_e
% -\frac{1}{2}\tilde\rho\tilde\bfu\frac{\partial v_e}{\partial t}
% -\frac{1}{2}\tilde\rho\tilde\bfu\frac{\partial v_e}{\partial t}
% \right\|^2\\
% +&\left\|-C_p\frac{\mu}{Pr}\Div\bftau
% -R\tilde\rho\Div\bfv_m
% -C_v\tilde\rho\tilde\bfu\Grad v_e
% -R\tilde\rho\tilde\bfu\Grad v_e
% -C_v\tilde\rho\frac{\partial v_e}{\partial t}
% \right\|\,.
% \end{aligned}
% \end{equation}

% An alternative norm to consider would be the graph norm defined on the convective terms plus the (scaled) mathematician's norm on the viscous terms:
% \begin{equation}
% \begin{aligned}
% % &\norm{\Grad\bfv_m+\Grad v_e\otimes\tilde\bfu}^2\\
% % +&\norm{\Grad v_e}^2\\
% % +&\left\|
% % \tilde\bfu\cdot\Grad v_c+\frac{\partial v_c}{\partial t}+\tilde\bfu\otimes\tilde\bfu:\Grad\bfv_m
% % +R\tilde T\Div\bfv_m+\tilde\bfu\cdot\frac{\partial\bfv_m}{\partial t}\right.\\
% % &\left.+C_v\tilde T\tilde\bfu\cdot\Grad v_e+\frac{1}{2}\tilde\bfu\cdot\tilde\bfu\tilde\bfu\cdot\Grad v_e
% % +R\tilde T\tilde\bfu\Grad v_e
% % +C_v\tilde T\frac{\partial v_e}{\partial t}+\frac{1}{2}\tilde\bfu\cdot\bfu\frac{\partial v_e}{\partial t}
% % \right\|^2\\
% % +&\left\|
% % +\tilde\rho\Grad v_c
% % +\tilde\rho\tilde\bfu\cdot\Grad\bfv_m+\tilde\rho\Grad\bfv_m\cdot\tilde\bfu
% % +\tilde\rho\frac{\partial\bfv_m}{\partial t}
% % +C_v\tilde\rho\tilde T\Grad v_e\right.\\
% % &\left.
% % +\frac{1}{2}\tilde\rho\tilde\bfu\cdot\tilde\bfu\Grad v_e
% % +\frac{1}{2}\tilde\rho\tilde\bfu\cdot\Grad v_e\tilde\bfu
% % +\frac{1}{2}\tilde\rho\Grad v_e\cdot\tilde\bfu\tilde\bfu
% % +R\tilde\rho\tilde T_\Grad v_e
% % -\tilde{\mathbb{D}}:\Grad v_e
% % +\frac{1}{2}\tilde\rho\tilde\bfu\frac{\partial v_e}{\partial t}
% % +\frac{1}{2}\tilde\rho\tilde\bfu\frac{\partial v_e}{\partial t}
% % \right\|^2\\
% % +&\left\|
% % +R\tilde\rho\Div\bfv_m
% % +C_v\tilde\rho\tilde\bfu\Grad v_e
% % +R\tilde\rho\tilde\bfu\Grad v_e
% % +C_v\tilde\rho\frac{\partial v_e}{\partial t}
% % \right\|^2\\
% % +&\norm{\mathbb{S}}^2+\norm{\Grad\trace{\mathbb{S}}}^2+\norm{\Div\mathbb{S}}^2\\
% % +&\norm{\bftau}^2+\norm{\Div\bftau}^2\,.
% \end{aligned}
% \end{equation}

% %   /$$$$$$$                      /$$       /$$                                  
% %  | $$__  $$                    | $$      | $$                                  
% %  | $$  \ $$  /$$$$$$   /$$$$$$ | $$$$$$$ | $$  /$$$$$$  /$$$$$$/$$$$   /$$$$$$$
% %  | $$$$$$$/ /$$__  $$ /$$__  $$| $$__  $$| $$ /$$__  $$| $$_  $$_  $$ /$$_____/
% %  | $$____/ | $$  \__/| $$  \ $$| $$  \ $$| $$| $$$$$$$$| $$ \ $$ \ $$|  $$$$$$ 
% %  | $$      | $$      | $$  | $$| $$  | $$| $$| $$_____/| $$ | $$ | $$ \____  $$
% %  | $$      | $$      |  $$$$$$/| $$$$$$$/| $$|  $$$$$$$| $$ | $$ | $$ /$$$$$$$/
% %  |__/      |__/       \______/ |_______/ |__/ \_______/|__/ |__/ |__/|_______/ 
% %                                                                                
% %                                                                                
% % 
% \subsection{Problems to Consider}
% \subsubsection{Steady Carter Plate Problem}
% The original inspiration to pursue a space-time DPG formulation stemmed from the inability to solve one particular steady-state compressible flow problem.
% Chan\cite{JesseDissertation} was unable to achieve convergence on the supersonic Carter plate problem\cite{Carter1973} with Reynolds numbers over 10,000.
% This convergence failure was also reported in \cite{KirkDissertation}.

% The Carter flat plate problem simulates laminar supersonic flow over an infinitesimally thin infinite flat plate.
% An oblique shock forms at the leading edge of the plate. 
% A laminar boundary layer forms at the same location along the length of the plate.
% The problem domain is $\Omega=[0,2]\times[0,1]$
% with free stream conditions are $\rho=1$, $\bfu=(1,0)^T$, and $T=1$.
% Chan assigned the following boundary conditions.
% \begin{description}
% 	\item[Symmetry boundary conditions:] upstream of the plate and on the upper surface, $u_n=q_n=\frac{\partial u_s}{\partial n}=0$.
% 	This implies that $u_2=q_2=\mathbb{D}_{12}=0$. 
% 	We can enforce these conditions with boundary conditions on the fluxes and traces. $u_2=0$ can clearly be enforced via $\hat\bfu_2=0$.
% 	Enforcing $q_n=0$ is just as easy, though less obvious - it requires setting $\hat t_e=0$.
% 	Recall that 
% 	\begin{equation*}
% 	\hat t_e=\trace\LRp{\rho\bfu\LRp{C_v T+\frac{1}{2}\bfu\cdot\bfu}+\bfu\rho RT+\bfq-\bfu\cdot\mathbb{D}}\cdot\bfn_x
% 	+\trace\LRp{\rho\LRp{C_v R+\frac{1}{2}\bfu\cdot\bfu}}n_t\,.
% 	\end{equation*}
% 	On the symmetry boundaries, $n_t=0$, and $\bfn_x=(0,\pm1)$ depending whether we are on the top or bottom surface. 
% 	Furthermore, $u_2=0$. Taken together, every term in $\hat t_e$ drops out except for the $\trace(\bfq)\cdot\bfn_x=\pm\trace(q_2)$ term.
% 	Thus, on the symmetry boundaries, setting a boundary condition on $\hat t_e$ is equivalent to setting a boundary condition on the trace of $q_2$.
% 	A similar situation occurs when attempting to set boundary conditions on $\mathbb{D}_{12}$. Recall that
% 	\begin{equation*}
% 	\hat\bft_m=\trace\LRp{\rho\bfu\otimes\bfu+\rho RT\bfI-\mathbb{D}}\cdot\bfn_x
% 	+\trace\LRp{\rho\bfu} n_t\,.
% 	\end{equation*}
% 	On the symmetry boundaries, most of the terms drop out of the second component except for the $\mathbb{D}$ term.
% 	Thus, setting $\hat t_{m2}=0$ is equivalent to setting $\trace(\mathbb{D}_{12})=0$ on these boundaries.
% 	\item[Flat plate boundary conditions:] $u_1=u_2=0$ and $T=T_w=[1+(\gamma-1)M_\infty^2/2]T_\infty=2.8T_\infty$ (for Mach 3 flow).
% 	These are imposed via $\hat\bfu$ and $\hat T$.
% 	\item[Inflow boundary conditions:] free stream conditions are enforced on all fluxes: $\hat t_e$, $\hat\bft_m$, $\hat t_e$. 
% 	Note that $n_t$ is zero on spatial boundaries.
% 	\item[Outflow boundary conditions:] there is not universal agreement about the correct boundary conditions to apply here.
% 	Most commonly it seems, practitioners enforce zero normal gradient conditions on $\bfu$ and $T$. 
% 	Others choose to enforce boundary conditions only on subsonic portions of the outflow\cite{DemkowiczCNS1990}.
% 	Papanastasiou first proposed a \emph{free boundary condition} or ``do nothing'' condition on the outflow.
% 	This proposal was further investigated by Renardy\cite{Renardy1997} and Griffiths\cite{Griffiths1997}.
% 	On the PDE level, this leaves an underdetermined system, but it was found that for a discretized system, this was a well-defined boundary condition.
% 	From previous experience with DPG, this outflow boundary condition appears to work very well.
% \end{description}

% The reported problem with the convergence was that the solution appeared to oscillate slightly upstream of the plate edge.
% Our theory is that this oscillation is a product of an unresolved mesh, but on a fully resolved space-time mesh these oscillations will dissipate, 
% and we can recover the steady state solution.
% Our goal is to repeat this problem with an adaptive space-time scheme and obtain results for much higher Reynolds numbers.

% %   /$$$$$$                                                                                               /$$ /$$       /$$          
% %  |_  $$_/                                                                                              |__/| $$      | $$          
% %    | $$   /$$$$$$$   /$$$$$$$  /$$$$$$  /$$$$$$/$$$$   /$$$$$$   /$$$$$$   /$$$$$$   /$$$$$$$  /$$$$$$$ /$$| $$$$$$$ | $$  /$$$$$$ 
% %    | $$  | $$__  $$ /$$_____/ /$$__  $$| $$_  $$_  $$ /$$__  $$ /$$__  $$ /$$__  $$ /$$_____/ /$$_____/| $$| $$__  $$| $$ /$$__  $$
% %    | $$  | $$  \ $$| $$      | $$  \ $$| $$ \ $$ \ $$| $$  \ $$| $$  \__/| $$$$$$$$|  $$$$$$ |  $$$$$$ | $$| $$  \ $$| $$| $$$$$$$$
% %    | $$  | $$  | $$| $$      | $$  | $$| $$ | $$ | $$| $$  | $$| $$      | $$_____/ \____  $$ \____  $$| $$| $$  | $$| $$| $$_____/
% %   /$$$$$$| $$  | $$|  $$$$$$$|  $$$$$$/| $$ | $$ | $$| $$$$$$$/| $$      |  $$$$$$$ /$$$$$$$/ /$$$$$$$/| $$| $$$$$$$/| $$|  $$$$$$$
% %  |______/|__/  |__/ \_______/ \______/ |__/ |__/ |__/| $$____/ |__/       \_______/|_______/ |_______/ |__/|_______/ |__/ \_______/
% %                                                      | $$                                                                          
% %                                                      | $$                                                                          
% %                                                      |__/
% \section{Space-time DPG for 2D Incompressible Navier-Stokes}
% The incompressible assumption simplifies the Navier-Stokes equations significantly.
% First of all, the continuity equation simplifies to $\Div\bfu=0$.
% This in turn leads to a simplified expression for the deviatoric stress tensor:
% \begin{equation*}
% 	\mathbb{D}=\mu\LRp{\Grad\bfu+\LRp{\Grad\bfu}^T}\,.
% \end{equation*}
% Incompressibility also renders the energy equation superfluous.
% Our first order system of equations for the incompressible Navier-Stokes equations are then
% \begin{align*}
% 	\mathbb{D}-\mu\LRp{\Grad\bfu+\LRp{\Grad\bfu}^T}&=0\\
% 	\Divxt\vecttwo{\rho\bfu\otimes\bfu+\tilde p\bfI-\mathbb{D}}{\rho\bfu}&=\tilde\bff_m\\
% 	\Div\bfu&=0\,,
% \end{align*}
% where $\tilde p$ and $\tilde\bff_m$ are used to differentiate these terms from their density rescaled values.
% For the problems we are interested, a constant density assumption is appropriate.
% In which case, we redefine $\mathbb{D}$ in terms of the kinematic viscosity $\nu=\mu/\rho$ and solve for a rescaled pressure $p:=\tilde p/\rho$.
% Our new system of equations is then
% \begin{subequations}
% \label{eq:incompressibleNSFirstOrder}
% \begin{align}
% 	\mathbb{D}-\nu\LRp{\Grad\bfu+\LRp{\Grad\bfu}^T}&=0\\
% 	\Divxt\vecttwo{\bfu\otimes\bfu+p\bfI-\mathbb{D}}{\bfu}&=\bff_m\\
% 	\Div\bfu&=0\,.
% \end{align}
% \end{subequations}

% %   /$$$$$$$                      /$$                        /$$     /$$                    
% %  | $$__  $$                    |__/                       | $$    |__/                    
% %  | $$  \ $$  /$$$$$$   /$$$$$$  /$$ /$$    /$$  /$$$$$$  /$$$$$$   /$$  /$$$$$$  /$$$$$$$ 
% %  | $$  | $$ /$$__  $$ /$$__  $$| $$|  $$  /$$/ |____  $$|_  $$_/  | $$ /$$__  $$| $$__  $$
% %  | $$  | $$| $$$$$$$$| $$  \__/| $$ \  $$/$$/   /$$$$$$$  | $$    | $$| $$  \ $$| $$  \ $$
% %  | $$  | $$| $$_____/| $$      | $$  \  $$$/   /$$__  $$  | $$ /$$| $$| $$  | $$| $$  | $$
% %  | $$$$$$$/|  $$$$$$$| $$      | $$   \  $/   |  $$$$$$$  |  $$$$/| $$|  $$$$$$/| $$  | $$
% %  |_______/  \_______/|__/      |__/    \_/     \_______/   \___/  |__/ \______/ |__/  |__/
% %                                                                                           
% %                                                                                           
% %
% \subsection{Derivation of Space-Time DPG Formulation}
% We start with \eqref{eq:incompressibleNSFirstOrder} and multiply by test functions $\mathbb{S}$ (symmetric tensor), $\bfv_m$, $q$, 
% then integrate by parts over each space-time element $K$:
% \begin{subequations}
% \label{eq:incompressibleNSBF}
% \begin{align}
% 	\LRp{\mathbb{D},\mathbb{S}}+\LRp{2\nu\bfu,\Div\mathbb{S}}-\LRa{2\nu\hat\bfu,\mathbb{S}\bfn_x}&=0\\
% 	-\LRp{\vecttwo{\bfu\otimes\bfu+p\bfI-\mathbb{D}}{\bfu},\Gradxt\bfv_m}+\LRa{\hat\bft_m,\bfv_m}&=\LRp{\bff_m,\bfv_m}\\
% 	-\LRp{\bfu,\Grad q}+\LRa{\hat\bfu,q\bfn_x}&=0\,,
% \end{align}
% \end{subequations}
% where 
% \begin{equation*}
% \begin{aligned}
% \hat\bfu&=\trace(\bfu)\\
% \hat\bft_m&=\trace\LRp{\bfu\otimes\bfu+p\bfI-\mathbb{D}}\cdot\bfn_x
% +\trace\LRp{\bfu} n_t\,.
% \end{aligned}
% \end{equation*}

% %   /$$       /$$                                         /$$                       /$$     /$$                    
% %  | $$      |__/                                        |__/                      | $$    |__/                    
% %  | $$       /$$ /$$$$$$$   /$$$$$$   /$$$$$$   /$$$$$$  /$$ /$$$$$$$$  /$$$$$$  /$$$$$$   /$$  /$$$$$$  /$$$$$$$ 
% %  | $$      | $$| $$__  $$ /$$__  $$ |____  $$ /$$__  $$| $$|____ /$$/ |____  $$|_  $$_/  | $$ /$$__  $$| $$__  $$
% %  | $$      | $$| $$  \ $$| $$$$$$$$  /$$$$$$$| $$  \__/| $$   /$$$$/   /$$$$$$$  | $$    | $$| $$  \ $$| $$  \ $$
% %  | $$      | $$| $$  | $$| $$_____/ /$$__  $$| $$      | $$  /$$__/   /$$__  $$  | $$ /$$| $$| $$  | $$| $$  | $$
% %  | $$$$$$$$| $$| $$  | $$|  $$$$$$$|  $$$$$$$| $$      | $$ /$$$$$$$$|  $$$$$$$  |  $$$$/| $$|  $$$$$$/| $$  | $$
% %  |________/|__/|__/  |__/ \_______/ \_______/|__/      |__/|________/ \_______/   \___/  |__/ \______/ |__/  |__/
% %                                                                                                                  
% %                                                                                                                  
% %   
% \subsubsection{Linearization}
% We follow the same linearization process as for the compressible system with $\tilde U=\LRc{\tilde\bfu,0,\bs0,\bs0,0}$ and 
% $\Delta U=\LRc{\Delta\bfu,p,\mathbb{D},\hat\bfu,\hat\bft_m}$.
% Notice that $\bfu$ is the only variable participating in nonlinear terms; nothing else needs to be linearized.
% The Jacobian of our incompressible Navier-Stokes system, $R'(\tilde U)\Delta U$ is
% \begin{align}
% 	&\LRp{\mathbb{D},\mathbb{S}}+\LRp{2\nu\Delta\bfu,\Div\mathbb{S}}-\LRa{2\nu\hat\bfu,\mathbb{S}\bfn_x}\\
% 	&-\LRp{\vecttwo{\Delta\bfu\otimes\tilde\bfu+\tilde\bfu\otimes\Delta\bfu
% 	+ p\bfI-\mathbb{D}}{\Delta\bfu},\Gradxt\bfv_m}+\LRa{\hat\bft_m,\bfv_m}\\
% 	&-\LRp{\Delta\bfu,\Grad q}+\LRa{\hat\bfu,q\bfn_x}\,.
% \end{align}
% The residual is then simply
% \begin{equation}
% 	-\LRp{\vecttwo{\tilde\bfu\otimes\tilde\bfu}{\tilde\bfu},\Gradxt\bfv_m}-\LRp{\bff_m,\bfv_m}\,.
% \end{equation}

% %   /$$$$$$$$                       /$$           /$$   /$$                                  
% %  |__  $$__/                      | $$          | $$$ | $$                                  
% %     | $$     /$$$$$$   /$$$$$$$ /$$$$$$        | $$$$| $$  /$$$$$$   /$$$$$$  /$$$$$$/$$$$ 
% %     | $$    /$$__  $$ /$$_____/|_  $$_/        | $$ $$ $$ /$$__  $$ /$$__  $$| $$_  $$_  $$
% %     | $$   | $$$$$$$$|  $$$$$$   | $$          | $$  $$$$| $$  \ $$| $$  \__/| $$ \ $$ \ $$
% %     | $$   | $$_____/ \____  $$  | $$ /$$      | $$\  $$$| $$  | $$| $$      | $$ | $$ | $$
% %     | $$   |  $$$$$$$ /$$$$$$$/  |  $$$$/      | $$ \  $$|  $$$$$$/| $$      | $$ | $$ | $$
% %     |__/    \_______/|_______/    \___/        |__/  \__/ \______/ |__/      |__/ |__/ |__/
% %                                                                                            
% %                                                                                            
% %      
% \subsubsection{Test Norm}

% \subsection{Problems to consider}


\end{document}