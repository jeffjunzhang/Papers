\documentclass{article}
% -----PACKAGES
%\usepackage[shortend,titlenumbered]{algorithm2e}
%\usepackage{algorithmic}
%\usepackage[plain]{algorithm}
\usepackage{multicol}
\usepackage{color}
\usepackage{multirow}
\usepackage{fancybox}
%\usepackage{index}
\usepackage{varioref}
\usepackage{psfrag}
\usepackage{epsfig}
\usepackage{boxedminipage}
\usepackage{graphicx}
\usepackage{rotating}
\usepackage{amsmath}
\usepackage{amssymb}
%\usepackage{amsfont}
\usepackage{latexsym}
\usepackage{alltt}
%\usepackage[small,bf]{caption}
\usepackage{url}
%\usepackage{citesort}
%\usepackage{crop}
\usepackage{array}
\usepackage{subfigure}
\usepackage{dcolumn}

% -----SETLENGTH
%\setlength{\captionmargin}{20pt} 

% -----NEWCOMMANDS
\newcommand{\nc}{\newcommand}
\nc{\mathsm}[1]{\text{\small{$#1$}}}
\nc{\ubar}[1]{\underset{-}{#1}}
\nc{\optype}{\textrm}
\nc{\EQ}[1]{(\ref{eq:#1})}
\nc{\TAB}[1]{\ref{tab:#1}}
\nc{\FIG}[1]{\ref{fig:#1}}
\nc{\SEC}[1]{\ref{sec:#1}}
\nc{\ALG}[1]{\ref{alg:#1}}
\nc{\CHAP}[1]{\ref{chap:#1}}
\nc{\mtrx}[1]{\boldsymbol{\mathbf{#1}}}
\nc{\vctr}[1]{\boldsymbol{\mathbf{#1}}}
\nc{\grad}{\mbox{\boldmath$\nabla$}}
\nc{\gradient}{\textsl{grad}\,}
\nc{\hessian}{\textsl{grad\,}^2}
\nc{\ii}{\iota}
\nc{\dd}{d}
\nc{\ee}{\mathrm{e}}
\nc{\pdiv}[2]{\partial{#1}/\partial{#2}}
\nc{\dpdiv}[2]{\displaystyle{\frac{\partial{#1}}{\partial{#2}}}}
\nc{\ddiv}[2]{\displaystyle{\frac{\dd{#1}}{\dd{#2}}}}
\nc{\inpr}{\hspace{-1pt}\cdot\hspace{-1pt}}
\nc{\IR}{\mathbb{R}}
\nc{\IN}{\mathbb{N}}
\nc{\IZ}{\mathbb{Z}}
\nc{\IC}{\mathbb{C}}
\nc{\half}{\frac{1}{2}}
\nc{\shalf}{\scriptstyle{\half}} 
\nc{\ds}[1]{\displaystyle{#1}}
\nc{\ts}[1]{\textstyle{#1}}
\nc{\sign}{\optype{sign}}
\nc{\spr}{\optype{spr}}
\nc{\dist}{\optype{dist}}
\nc{\rank}{\optype{rank}}
\nc{\codim}{\optype{codim}}
\nc{\supp}{\optype{supp}}
\nc{\diag}{\optype{diag}}
\nc{\meas}{\optype{meas}}
\nc{\cond}{\optype{cond}}
\nc{\kernel}{\optype{kernel}}
\nc{\spa}{\optype{span}}
\nc{\order}{\mathcal{O}}
\nc{\Fr}{\mathrm{Fr}}
\nc{\Rey}{\mathrm{Re}}
\nc{\Ord}{O}
\nc{\ord}{o}
\nc{\st}{\:{:}\:}
\nc{\closure}[1]{\overline{#1}}
\nc{\emin}[1]{\emph{#1}\index{#1}\/}
\nc{\rmin}[1]{#1\index{{}@{#1}}}
\nc{\Laplace}{\Delta}
\nc{\ie}{i.e.}
\nc{\eg}{e.g.}
%\nc{\union}{\cup}
\nc{\Union}{\bigcup}
\nc{\lf}[1]{\mathsf{#1}}
\nc{\dbar}[1]{\bar{\bar{#1}}}
\nc{\ul}[1]{\underline{#1}}
\nc{\hpt}{\hspace{0.5pt}}
\nc{\E}[1]{\times{}10^{#1}}
\nc{\inp}[2]{\langle{#1},{#2}\rangle}
\nc{\tmpcommand}{}

% -----RENEWCOMMANDS
\renewcommand{\baselinestretch}{1}
\renewcommand{\exp}{\optype{exp}\,}
\renewcommand{\cosh}{\optype{cosh}\,}
\renewcommand{\tanh}{\optype{tanh}\,}
\renewcommand{\sinh}{\optype{sinh}\,}
\renewcommand{\div}[1]{\optype{div}\,{#1}}
\renewcommand{\half}{\mbox{$\frac{1}{2}$}}
%\renewcommand{\descriptionlabel}[1]{\hspace{\labelsep}\emph{#1}}

% -----ETC
\raggedbottom


\DeclareMathOperator{\curl}{\bf curl}
\DeclareMathOperator{\rot}{\rm curl}
\DeclareMathOperator{\divv}{\rm div}
\newcommand{\tro}{\gamma_0}
\newcommand{\trt}{\gamma_{\sft}}
\newcommand{\trn}{\gamma_{\sfn}}

\newcommand{\PT}{{\partial T}}
\newcommand{\bbN}{{\mathbb{N}}}
\newcommand{\bbP}{{\mathbb{P}}}

\newcommand{\scC}{{\mathscr{C}}}
\newcommand{\caD}{{\mathcal{D}}}
\newcommand{\caL}{{\mathcal{L}}}

\newcommand{\sfe}{{\mathsf{e}}}
\newcommand{\sff}{{\mathsf{f}}}
\newcommand{\sft}{{\boldsymbol{\mathsf{t}}}}
\newcommand{\sfn}{{\boldsymbol{\mathsf{n}}}}

%   Common caligraphic abbrevs
\newcommand{\BB}{\mathcal{B}}
\newcommand{\CC}{\mathcal{C}}
\newcommand{\DD}{\mathcal{D}}
\newcommand{\EE}{\mathcal{E}}
\newcommand{\FF}{\mathcal{F}}
\newcommand{\GG}{\mathcal{G}}
\newcommand{\II}{\mathcal{I}}
\newcommand{\JJ}{\mathcal{J}}
\newcommand{\KK}{\mathcal{K}}
\newcommand{\LL}{\mathcal{L}}
\newcommand{\OO}{\mathcal{O}}
\newcommand{\QQ}{\mathcal{Q}}
\newcommand{\RR}{\mathcal{R}}
\newcommand{\TT}{\mathcal{T}}


 %% JAY'S PREAMBLE
 %%========================

%   Math symbol definitions
\def\d{\partial}
%\newsymbol\lee 132E
\newcommand{\union}{\mathop{\bigcup}}
\newcommand{\intersect}{\mathop{\bigcap}}
\newcommand{\binomial}[2]{\ensuremath{
		\begin{pmatrix}{#1}\\{#2}\end{pmatrix}}}
\newcommand{\smallbinomial}[2]{\ensuremath{
		(\begin{smallmatrix}{#1}\\{#2}\end{smallmatrix})}}
\newcommand{\tang}[1]{\ensuremath{{#1}_{\intercal}}} % can use \top
						     % also
\newcommand{\hypergeom}[2]{\ensuremath{\sideset{_{#1}}{_{#2}}{\mathop{F}}}}
%   Difficult names
\newcommand{\Babuska}{Babu{\v{s}}ka}       % Remember: Usage is \Babuska\
\newcommand{\Cea}{C{\'e}a}                 % with trailing `\' to give space
\newcommand{\Poincare}{Poincar{\'{e}}}     % when needed, but when ending
\newcommand{\Nedelec}{N{\'{e}}d{\'{e}}lec} % sentence use \Babuska.
\newcommand{\Frechet}{Fr{\'{e}}chet}
\newcommand{\Muller}{M{\"u}ller}
\newcommand{\LHospital}{L'H{\^{o}}spital}
%   Bold and beautiful
\newcommand{\ba}{{\boldsymbol{a}}}
\newcommand{\bA}{\boldsymbol{A}}
\newcommand{\balpha}{{\boldsymbol{\alpha}}}
\newcommand{\bB}{{\boldsymbol{B}}}
\newcommand{\bb}{{\boldsymbol{b}}}
\newcommand{\bbeta}{{\boldsymbol{\beta}}}
\newcommand{\etab}{{\boldsymbol{\eta}}}
\newcommand{\bC}{{\boldsymbol{C}}}
\newcommand{\bc}{{\boldsymbol{c}}}
\newcommand{\bD}{{\boldsymbol{D}}}
\newcommand{\bd}{{\boldsymbol{d}}}
\newcommand{\db}{{\boldsymbol{\d}}}
\newcommand{\bdelta}{{\boldsymbol{\delta}}}
\newcommand{\bDelta}{{\boldsymbol{\Delta}}}
\newcommand{\beps}{{\boldsymbol{\varepsilon}}}
\newcommand{\be}{{\boldsymbol{e}}}
\newcommand{\bg}{{\boldsymbol{g}}}
\newcommand{\bm}{{\boldsymbol{m}}}
\newcommand{\bn}{{\boldsymbol{n}}}
\newcommand{\bN}{{\boldsymbol{N}}}
\newcommand{\bp}{{\boldsymbol{p}}}
\newcommand{\bpsi}{{\boldsymbol{\psi}}}
\newcommand{\bq}{{\boldsymbol{q}}}
\newcommand{\bxi}{{\boldsymbol{\xi}}}
\newcommand{\bE}{{\boldsymbol{E}}}
\newcommand{\bF}{{\boldsymbol{F}}}
\newcommand{\bh}{{\boldsymbol{h}}}
\newcommand{\bH}{{\boldsymbol{H}}}
\newcommand{\bI}{{\boldsymbol{I}}}
\newcommand{\bj}{{\boldsymbol{j}}}
\newcommand{\bJ}{{\boldsymbol{J}}}
\newcommand{\bK}{{\boldsymbol{K}}}
\newcommand{\bk}{{\boldsymbol{k}}}
\newcommand{\bll}{{\boldsymbol{\ell}}}
\newcommand{\bL}{{\boldsymbol{L}}}
\newcommand{\blambda}{{\boldsymbol{\lambda}}}
\newcommand{\bmu}{{\boldsymbol{\mu}}}
\newcommand{\bM}{{\boldsymbol{M}}}
\newcommand{\bomega}{{\boldsymbol{\omega}}}
\newcommand{\bP}{{\boldsymbol{P}}}
\newcommand{\bphi}{{\boldsymbol{\phi}}}
\newcommand{\bQ}{{\boldsymbol{Q}}}
\newcommand{\bG}{{\boldsymbol{G}}}
\newcommand{\bu}{{\boldsymbol{u}}}
\newcommand{\bU}{{\boldsymbol{U}}}
\newcommand{\bV}{{\boldsymbol{V}}}
\newcommand{\bX}{{\boldsymbol{X}}}
\newcommand{\bv}{{\boldsymbol{v}}}
\newcommand{\bw}{{\boldsymbol{w}}}
\newcommand{\bW}{{\boldsymbol{W}}}
\newcommand{\bR}{{\boldsymbol{R}}}
\newcommand{\br}{{\boldsymbol{r}}}
\newcommand{\bS}{{\boldsymbol{S}}}
\newcommand{\bT}{{\boldsymbol{T}}}
\newcommand{\btau}{{\boldsymbol{\tau}}}
\newcommand{\bt}{{\boldsymbol{t}}}
\newcommand{\bx}{{\boldsymbol{x}}}
\newcommand{\by}{{\boldsymbol{y}}}
\newcommand{\bz}{{\boldsymbol{z}}}
\newcommand{\bzero}{{\boldsymbol{0}}}
\newcommand{\bZ}{{\boldsymbol{Z}}}
%   Common scalar fields
\newcommand{\RRR}{\mathbb{R}}
\newcommand{\CCC}{\mathbb{C}}
\newcommand{\ZZZ}{\mathbb{Z}}
\newcommand{\NNN}{\mathbb{N}}
%   Differential operators
\newcommand{\dive}{\mathop\mathrm{div}}
%\newcommand{\grad}{\ensuremath{\mathop{{\bf{grad}}}}}
%\newcommand{\curl}{{\ensuremath\mathop{\mathbf{curl}\,}}}
\newcommand{\Curl}{ {\bf Curl}}
\newcommand{\dx}{\ensuremath{\mathrm{d}x}}
\newcommand{\dy}{\ensuremath{\mathrm{d}y}}
\newcommand{\dr}{\ensuremath{\mathrm{d}r}}
\newcommand{\dR}{\ensuremath{\mathrm{d}R}}
\newcommand{\drho}{\ensuremath{\mathrm{d}\rho}}
\newcommand{\dz}{\ensuremath{\mathrm{d}z}}
\newcommand{\dzeta}{\ensuremath{\mathrm{d}\zeta}}
%   Wordy math symbols
\newcommand{\card}{\ensuremath{\mathop\mathrm{card}}}
%\newcommand{\diag}{\ensuremath{\mathop\mathrm{diag}}}
\newcommand{\diam}{\ensuremath{\mathop\mathrm{diam}}}
%\newcommand{\dist}{\mathop\mathrm{dist}}
\newcommand{\Ker}{\mathop\mathrm{Ker}}
\newcommand{\Range}{\mathop\mathrm{Range}}
%\newcommand{\rank}{\mathop\mathrm{rank}}
%\newcommand{\meas}{\mathop\mathrm{meas}}
\newcommand{\Forall}{\quad\text{for all }}
%\newcommand{\supp}{\mathop\mathrm{supp}}
\newcommand{\Span}{\mathop\mathrm{Span}}
\newcommand{\Hdiv}[1]{\bH(\dive,#1)}
%\newcommand{\Hcurl}[1]{\bH(\curl,#1)}
%   Common caligraphic abbrevs
%\newcommand{\BB}{\mathcal{B}}
%\newcommand{\CC}{\mathcal{C}}
%\newcommand{\DD}{\mathcal{D}}
%\newcommand{\EE}{\mathcal{E}}
%\newcommand{\FF}{\mathcal{F}}
%\newcommand{\GG}{\mathcal{G}}
%\newcommand{\II}{\mathcal{I}}
%\newcommand{\JJ}{\mathcal{J}}
%\newcommand{\KK}{\mathcal{K}}
%\newcommand{\LL}{\mathcal{L}}
%\newcommand{\OO}{\mathcal{O}}
%\newcommand{\QQ}{\mathcal{Q}}
%\newcommand{\RR}{\mathcal{R}}
%\newcommand{\TT}{\mathcal{T}}
%   Variations on standard symbols
\newcommand{\veps}{\varepsilon}
\newcommand{\vlam}{\varLambda}
\newcommand{\vpi}{\varPi}
\newcommand{\vPi}{\boldsymbol{\varPi}}
\newcommand{\vsig}{\varSigma}
\newcommand{\vbt}{\boldsymbol{\varTheta}}
\newcommand{\vPsi}{\boldsymbol{\varPsi}}
%\newcommand{\ii}{\hat{\imath}}
%   Innerproducts, norms, etc
\newcommand{\ntrip}[1]{|\!|\!| {#1} |\!|\!|}
\newcommand{\ip}[1]{\langle {#1} \rangle}
%   Utilities
\newcommand{\blnk}{\underline{\hspace{3cm}}\;}
\newcommand{\marg}[1]{\marginpar{\tiny{\framebox{\parbox{1.7cm}{#1}}}}}
\newcommand{\degreeC}[1]{\ensuremath{{#1\,}^\circ\!\text{C}}}
                        % try also  \textcelsius of textcomp package
%   Trademarked names \texttrademark, \textregistered
\newcommand{\matlab}{MATLAB\textregistered\renewcommand{\matlab}{MATLAB}}
\newcommand{\femlab}{FEMLAB\textregistered\renewcommand{\femlab}{FEMLAB}}

%   Style preferences
\renewcommand{\thefootnote}{\fnsymbol{footnote}} % Use symbols instead of
						 % numbers for footnotes
						 

\newcommand{\Eg}{\EE^\mathrm{grad}}
\newcommand{\Ec}{\boldsymbol{\EE}^\mathrm{curl}}
\newcommand{\Ed}{\boldsymbol{\EE}^\mathrm{div}}


\newcommand{\bfdu}{\mbox{\boldmath $\delta u$}}
\newcommand{\bfdv}{\mbox{\boldmath $\delta v$}}
\newcommand{\du}{{\delta u}}
\newcommand{\dv}{{\delta v}}
\newcommand{\bfnabt}{\widetilde{\bfnab}}
\newcommand{\bfepst}{\widetilde{\bfeps}}


\title{Robustness for Transient Problems}
\author{Truman E. Ellis and Jesse L. Chan}

\begin{document}
\maketitle

Assume that boundary conditions are applied on the boundary $\Gamma_0\subset \Gamma$.  Recall that, for the ultra-weak variational formulation
\[
b\LRp{\LRp{u,\uh},v} = \LRp{u,A^*_h v}_{\L} + \LRa{\uh, \jump{v}}_{\Gh\setminus \Gamma_0}
\]
we can recover
\[
\norm{u}_{\L}^2 = b(u,v^*)
\]
for conforming $v^*$ satisfying the adjoint equation
\begin{align*}
A^* v^* &= u\\
v^* &= 0 \text{ on } \Gh\setminus\Gamma_0.
\end{align*}
Together, these give necessary conditions on the test norm $\norm{\cdot}_V$ such that we have $L^2$ robustness (this gives robustness in the variable $u$; for the first order formulation, conditions for $\sigma$ must also be shown).  
\[
\norm{u}_{\L}^2 = b(u,v^*) \leq \frac{b(u,v^*)}{\norm{v^*}_V} \norm{v^*}_V \leq \norm{u}_E \norm{v^*}_V
\]
Thus, showing $\norm{v^*}_V \lesssim \norm{u}_{\L}$ gives the result that $\norm{u}_{\L} \lesssim \norm{u}_E$.  


\section{Reaction-diffusion}

Consider reaction diffusion
\begin{align*}
\pd{u}{t} + u - \epsilon \Delta u &= f\\
u &= 0 \text{ on } \Gamma_1\\
\pd{u}{n} &= 0 \text{ on } \Gamma_2\\
u(t=0) &= u_0.
\end{align*}
The adjoint equation satisfies
\begin{align*}
-\pd{v}{t} + v - \epsilon \Delta v &= u\\
v &= 0 \text{ on } \Gamma_1\\
\pd{v}{n} &= 0 \text{ on } \Gamma_2\\
v(t=T) &= 0.
\end{align*}
(The boundary conditions can be derived by taking the ultra-weak formulation and choosing boundary conditions such that the temporal flux and spatial flux terms $\LRa{\uh, \jump{\tau_n}}_{\Gamma_1}$ and $\LRa{\fnh,\jump{v}}_{\Gamma_2}$ are zero.)

We can then derive that the test norm
\[
\norm{v}_V^2 = \norm{\pd{v}{t}}^2 + \norm{v}^2 + \epsilon\norm{\Grad v}^2 
\]
provides the necessary bound $\norm{v^*}_V \lesssim \norm{u}_{\L}$.

To see, this we multiply the adjoint equation by two terms as follows:
\begin{enumerate}
\item Multiply by $v$ and integrate over $\Omega \times [0,T] = Q$ to get
\[
-\int_Q \pd{v}{t}v + \int_Q v^2 + \epsilon \int_Q \LRb{\Grad v}^2 - \epsilon \int_0^T\int_{\Gamma} \pd{v}{n}v = \int_Q uv.
\]
Noting that either $v = 0$ or $\pd{v}{n} = 0$ on the boundary removes the integral over $\Gamma$.  Next, we can factor the first term and use Young's inequality to get
\[
-\int_0^T  \pd{}{t}\int_{\Omega} v^2 + \norm{v}^2_Q + \epsilon \norm{\Grad v}^2_Q \leq \frac{1}{2}\norm{u}^2_Q + \frac{1}{2}\norm{v}^2_Q
\]
Integrating by parts the first term gives
\[
-\left.\int_{\Omega} v^2\right|_0^T + \frac{1}{2}\norm{v}^2_Q + \epsilon \norm{\Grad v}^2_Q \leq \frac{1}{2}\norm{u}^2_Q
\]
Using boundary condition $v=0$ at $t= T$ gives
\[
\frac{1}{2}\norm{v}^2_Q + \epsilon \norm{\Grad v}^2_Q \leq \int_{\Omega} v(t=0)^2 + \frac{1}{2}\norm{v}^2_Q + \epsilon \norm{\Grad v}^2_Q \leq \frac{1}{2}\norm{u}^2_Q.
\]

\item Multiply by $-\pd{v}{t}$ and integrate over $Q$.  Young's inequality changes the right hand side to 
\[
\int_Q\pd{v}{t}^2 - \int_Q v\pd{v}{t} + \epsilon\int_Q \Delta v \pd{v}{t} = \int_Q -u \pd{v}{t} \leq \frac{1}{2}\norm{u}_Q^2 + \frac{1}{2}\norm{\pd{v}{t}}_Q^2 .
\]
The term $\int_Q v\pd{v}{t}$ can be reduced to the positive contribution $\int_{\Omega}{v(t=0)}^2  $ as above.  We can then take the Laplacian term, integrate by parts in space to get
\[
\int_Q \Delta v \pd{v}{t} = \int_0^T \int_{\Omega} \Delta v \pd{v}{t} =  \int_0^T \int_{\Gamma} \pd{v}{t}\pd{v}{n} - \int_0^T \int_{\Omega}\Grad\LRp{\pd{v}{t}}\Grad v.
\]
Since either $v = 0$ or $\pd{v}{n} = 0$ on $\Gamma$, the first term disappears.  The second term can be bounded by noting
\[
- \int_0^T \int_{\Omega}\Grad\LRp{\pd{v}{t}}\Grad v = - \int_0^T \pd{}{t}\int_{\Omega}\LRb{\Grad v} ^2 = - \left.\int_{\Omega}\LRb{\Grad v}^2 \right|_0^T.
\]
Since $v = 0$ at $t=T$, $\Grad v = 0$ at $t=T$ as well, and we are left with the positive contribution $\int_{\Omega}\LRb{\Grad v(t=0)}^2$.  Then,
\[
\frac{1}{2}\norm{\pd{v}{t}}_Q^2 \leq \frac{1}{2}\norm{u}_Q.
\]
\end{enumerate}
Together, these two show that, under test norm
\[
\norm{v}_V^2 = \norm{\pd{v}{t}}^2 + \norm{v}^2 + \epsilon\norm{\Grad v}^2,
\]
the adjoint equation $v^*$ satisfies
\[
\norm{v^*}_V \lesssim \norm{u}_{\L}
\]
and thus the DPG energy norm robustly bounds the $L^2$ norm from above
\[
\norm{u}_{\L} \lesssim \norm{u}_E.
\]

\section{Convection-diffusion}

Consider convection-diffusion
\begin{align*}
\pd{u}{t} + \bfbeta\cdot\Grad u - \epsilon \Delta u &= f\\
u &= 0 \text{ on } \Gamma_{out}\\
\pd{u}{n} &= 0 \text{ on } \Gamma_{in}\\
u(t=0) &= u_0.
\end{align*}
Let $\tilde\bfbeta:=\vecttwo{\bfbeta}{1}$ and $\Gradxt:=\vecttwo{\Grad}{\pd{}{t}}$, then we can rewrite this as
\begin{align*}
\tilde\bfbeta\cdot\Gradxt u - \epsilon \Delta u &= f\\
u &= 0 \text{ on } \Gamma_{out}\\
\pd{u}{n} &= 0 \text{ on } \Gamma_{in}\\
u(t=0) &= u_0.
\end{align*}
The adjoint equation satisfies
\begin{align*}
-\tilde\bfbeta\cdot\Gradxt v - \epsilon \Delta v &= u\\
v &= 0 \text{ on } \Gamma_{in}\\
\pd{v}{n} &= 0 \text{ on } \Gamma_{out}\\
v(t=T) &= 0.
\end{align*}
(The boundary conditions can be derived by taking the ultra-weak formulation and choosing boundary conditions such that the temporal flux and spatial flux terms $\LRa{\uh, \jump{\tau_n}}_{\Gamma_{out}}$ and $\LRa{\fnh,\jump{v}}_{\Gamma_{in}}$ are zero.)
The $t=0$ and $t=T$ boundaries can be considered as an inflow and outflow boundary respectively in space-time 
and we denote $\partial Q_{in}:=\Gamma_{in}\cup t=0$ while $\partial Q_{out}:=\Gamma_{out}\cup t=T$.

We can then derive that the test norm
\[
\norm{v}_V^2 = \norm{\tilde\bfbeta\cdot\Gradxt v}^2 + \epsilon\norm{\Grad v}^2 
\]
provides the necessary bound $\norm{v^*}_V \lesssim \norm{u}_{\LQ}$.

To see this, we multiply the adjoint equation by two terms as follows:
\begin{enumerate}
\item Multiply by $-\tilde\bfbeta\cdot\Gradxt v$ and integrate over $Q$ to get
\begin{equation}
\label{eq:adj1}
\norm{\tilde\bfbeta\cdot\Gradxt v}=-\int_Q u\tilde\bfbeta\cdot\Gradxt v-\epsilon\int_Q\tilde\bfbeta\cdot\Gradxt v\Delta v\,.
\end{equation}
Note that
\begin{align*}
-\int_Q\tilde\bfbeta\cdot\Gradxt v\Delta v&=-\int_Q\tilde\bfbeta\cdot\Gradxt v\Div\Grad v \\
%
&=-\int_{\Gamma_x}\tilde\bfbeta\cdot\Gradxt v\Grad v\cdot\bfn_x
+\int_Q\Grad(\tilde\bfbeta\cdot\Gradxt v)\cdot\Grad v \\
%
&=-\int_{\Gamma_x}\tilde\bfbeta\cdot\Gradxt v\Grad v\cdot\bfn_x
+\int_Q(\Grad\tilde\bfbeta\cdot\Gradxt v)\cdot\Grad v\\
&\quad+\int_Q\tilde\bfbeta\cdot\Grad\Gradxt v\cdot\Grad v\\
%
&=-\int_{\Gamma_x}\tilde\bfbeta\cdot\Gradxt v\Grad v\cdot\bfn_x
+\int_Q(\Grad\bfbeta\cdot\Grad v)\cdot\Grad v\\
&\quad+\frac{1}{2}\int_Q\tilde\bfbeta\cdot\Gradxt(\Grad v\cdot\Grad v)\\
%
&=-\int_{\Gamma_x}\tilde\bfbeta\cdot\Gradxt v\Grad v\cdot\bfn_x
+\int_Q(\Grad\bfbeta\cdot\Grad v)\cdot\Grad v\\
&\quad+\frac{1}{2}\int_\Gamma\tilde\bfbeta\cdot\bfn(\Grad v\cdot\Grad v)
-\frac{1}{2}\int_Q\Divxt\tilde\bfbeta(\Grad v\cdot\Grad v)\\
%
&=-\int_{\Gamma_x}\tilde\bfbeta\cdot\Gradxt v\Grad v\cdot\bfn_x
+\int_Q(\Grad\bfbeta\cdot\Grad v)\cdot\Grad v\\
&\quad+\frac{1}{2}\int_\Gamma\tilde\bfbeta\cdot\bfn(\Grad v\cdot\Grad v)
-\frac{1}{2}\int_Q\Div\bfbeta(\Grad v\cdot\Grad v)\\
%
&=-\int_{\Gamma_x}\tilde\bfbeta\cdot\Gradxt v\Grad v\cdot\bfn_x
+\frac{1}{2}\int_\Gamma\tilde\bfbeta\cdot\bfn(\Grad v\cdot\Grad v)\\
&\quad+\int_Q\Grad v(\Grad\bfbeta-\frac{1}{2}\Div\bfbeta\bfI)\Grad v\\
\end{align*}
Plugging this into \eqref{eq:adj1}, we get
\begin{align*}
\norm{\tilde\bfbeta\cdot\Gradxt v}
&=-\int_Q u\tilde\bfbeta\cdot\Gradxt v
+\epsilon\int_Q\Grad v(\Grad\bfbeta-\frac{1}{2}\Div\bfbeta\bfI)\Grad v\\
&\quad-\epsilon\int_{\Gamma_x}\tilde\bfbeta\cdot\Gradxt v\Grad v\cdot\bfn_x
+\epsilon\frac{1}{2}\int_\Gamma\tilde\bfbeta\cdot\bfn(\Grad v\cdot\Grad v)\\
%
&=-\int_Q u\tilde\bfbeta\cdot\Gradxt v
+\epsilon\int_Q\Grad v(\Grad\bfbeta-\frac{1}{2}\Div\bfbeta\bfI)\Grad v\\
&\quad
-\int_{\Gamma_-}\tilde\bfbeta\cdot\Gradxt v\underbrace{\Grad v\cdot\bfn_x}_{=0}
-\int_{\Gamma_+}\LRp{\underbrace{\pd{v}{t}}_{=0}+\bfbeta\cdot\Grad v}\Grad v\cdot\bfn_x\\
&\quad
+\frac{1}{2}\int_{\Gamma_-}\underbrace{\bfbeta\cdot\bfn_x}_{<0}(\Grad v\cdot\Grad v)
+\frac{1}{2}\int_{\Gamma_+}\bfbeta\cdot\bfn_x(\Grad v\cdot\Grad v)\\
&\quad
+\frac{1}{2}\int_{\Gamma_0}\underbrace{n_t}_{<0}(\Grad v\cdot\Grad v)
+\frac{1}{2}\int_{\Gamma_T}n_t\underbrace{(\Grad v\cdot\Grad v)}_{=0}\\
%
&\leq-\int_Q u\tilde\bfbeta\cdot\Gradxt v
+\epsilon\int_Q\Grad v(\Grad\bfbeta-\frac{1}{2}\Div\bfbeta\bfI)\Grad v\\
&\quad
+\int_{\Gamma_+}\LRp{-\pd{v}{\bfn_x}\bfbeta
+\frac{1}{2}\bfbeta\cdot\bfn_x\Grad v}\cdot\Grad v\\
%
&=-\int_Q u\tilde\bfbeta\cdot\Gradxt v
+\epsilon\int_Q\Grad v(\Grad\bfbeta-\frac{1}{2}\Div\bfbeta\bfI)\Grad v\\
&\quad
+\int_{\Gamma_+}\LRp{-\pd{v}{\bfn_x}\bfbeta
+\frac{1}{2}\bfbeta\cdot\bfn_x\pd{v}{\bfn_x}\bfn_x}\cdot\pd{v}{\bfn_x}\bfn_x\\
%
&=-\int_Q u\tilde\bfbeta\cdot\Gradxt v
+\epsilon\int_Q\Grad v(\Grad\bfbeta-\frac{1}{2}\Div\bfbeta\bfI)\Grad v\\
&\quad
\underbrace{-\frac{1}{2}\int_{\Gamma_+}\LRp{\pd{v}{\bfn_x}}^2\bfbeta\cdot\bfn_x}_{<0}\\
%
&\leq-\int_Q u\tilde\bfbeta\cdot\Gradxt v
+\epsilon\int_Q\Grad v(\Grad\bfbeta-\frac{1}{2}\Div\bfbeta\bfI)\Grad v\\
%
&\leq-\frac{\norm{u}}{2}+\frac{\norm{\tilde\bfbeta\cdot\Gradxt v}}{2}
+\epsilon\int_Q\Grad v(\Grad\bfbeta-\frac{1}{2}\Div\bfbeta\bfI)\Grad v\\
%
&\leq-\frac{\norm{u}}{2}+\frac{\norm{\tilde\bfbeta\cdot\Gradxt v}}{2}
+\epsilon C\norm{\Grad v}^2
\end{align*}

\item Define $w=e^{T-t}v$ and note that $\pd{w}{t}=\LRp{\pd{v}{t}-v}e^{T-t}$ while $\Grad w=\cancelto{0}{\Grad e^{T-t}} v+e^{T-t}\Grad v$ and
$\Div(\bfbeta w)=\Div(\bfbeta)e^{T-t} v+\bfbeta\cdot e^{T-t}\Grad v$ and $\Delta w=e^{T-t}\Delta v$. 
Also, $\Gradxt w=\pd{e^{T-t} v}{t}+\Grad{e^{T-t} v}=e^{T-t}(\Gradxt v-v)$.
Plugging this into the adjoint equation, we get
\begin{equation*}
-\tilde\bfbeta\cdot\Gradxt(w)-\epsilon\Delta w=u-\epsilon\Div\bfsigma
\end{equation*}
or 
\begin{equation*}
\tilde\bfbeta\cdot\Gradxt(v)-v+\epsilon\Delta v=e^{t-T}(-u+\epsilon\Div\bfsigma)
\end{equation*}
Multiply by $-v$ and integrate to get
\begin{align*}
\int_Q-\tilde\bfbeta\cdot\Gradxt vv+v^2-\epsilon\Delta vv=\int_Qe^{t-T}uv-\epsilon\int_Qe^{t-T}\Div\bfsigma v
\end{align*}
Then
\begin{align*}
\norm{v}^2&=\int_Qe^{t-T}uv
-\epsilon\int_Qe^{t-T}\Div\bfsigma v
+\int_Q\tilde\bfbeta\cdot\Gradxt vv
+\epsilon\int_Q\Delta vv\\
%
&=\int_Qe^{t-T}uv
-\epsilon\int_Qe^{t-T}\Div\bfsigma v
+\frac{1}{2}\int_Q\tilde\bfbeta\cdot\Gradxt(v)^2
-\epsilon\int_Q\Grad v\Grad v
+\epsilon\int_\Gamma v\Grad v\cdot\bfn
\\
\end{align*}
Or
\begin{align*}
\norm{v}^2+\epsilon\norm{\Grad v}^2&=\int_Qe^{t-T}uv
-\epsilon\int_Qe^{t-T}\Div\bfsigma v\\
&\quad-\frac{1}{2}\int_Q\underbrace{\Divxt\tilde\bfbeta}_{=0}(v)^2
+\frac{1}{2}\int_\Gamma v^2\tilde\bfbeta\cdot\bfn
+\epsilon\int_{\Gamma_x} v\Grad v\cdot\bfn_x
\\
&=\int_Qe^{t-T}uv
+\epsilon\int_Qe^{t-T}\bfsigma\cdot\Grad v
-\epsilon\int_{\Gamma_-}v\underbrace{\bfsigma\cdot\bfn_x}_{=\epsilon\pd{v}{n}=0}
-\epsilon\int_{\Gamma_+}\underbrace{v}_{=0}\bfsigma\cdot\bfn_x\\
&\quad+\frac{1}{2}\int_{\Gamma_-} v^2\underbrace{\bfbeta\cdot\bfn_x}_{<0}
+\frac{1}{2}\int_{\Gamma_+} \underbrace{v^2}_{=0}\bfbeta\cdot\bfn_x\\
&\quad+\frac{1}{2}\int_{\Gamma_0} \underbrace{v^2 (-n_t)}_{<0}
+\frac{1}{2}\int_{\Gamma_T} \underbrace{v^2}_{=0} n_t\\
&\quad+\epsilon\int_{\Gamma_-} v\underbrace{\Grad v\cdot\bfn_x}_{=0}
+\epsilon\int_{\Gamma_+} \underbrace{v}_{=0}\Grad v\cdot\bfn_x
\\
&\leq \norm{e^{t-T}}_{L_\infty(Q)}\LRp{\int_Quv+\epsilon\int_Q\bfsigma\cdot\Grad v}
\\
&\leq \LRp{\frac{\norm{u}^2}{2}+\frac{\epsilon\norm{\bfsigma}^2}{2}+\frac{\norm{v}^2}{2}+\frac{\epsilon\norm{\Grad v}^2}{2}}
\\
\end{align*}
\end{enumerate}


\section{Robustness for transient problems given spatial robustness}

Suppose we have the transient problem
\[
\pd{u}{t} + Au = f
\]
with initial condition $u(x,0) = u_0$.  Suppose that DPG is robust under the ultra-weak variational formulation for the steady problem
\[
\LRp{u,A^*_h v}_{\L} + \LRa{\uh, \jump{v}}_{\Gamma_h\setminus \Gamma_0} = \LRp{f,v}
\]
with test norm $\norm{v}_{V}$.  Then, can we show that 
\[
\norm{v}_{V,t} \coloneqq \norm{v}_V + \norm{\pd{v}{t}}_{\L}
\]
also leads to a robust upper bound of the $L^2$ norm by the DPG energy norm?  I believe this may be possible.  The adjoint equation for robustness for the transient problem gives
\[
-\pd{v}{t} + A^*v = u
\]
with $v = 0$ at $t=T$...  


\end{document}