\documentclass{article}

\usepackage{fullpage}
\usepackage{array}
\usepackage{amsmath,amssymb,amsfonts,mathrsfs,amsthm}
\usepackage[utf8]{inputenc}
\usepackage{listings}
\usepackage{mathtools}
\usepackage{pdfpages}
\usepackage[textsize=footnotesize,color=green]{todonotes}
\usepackage{bm}
\usepackage{tikz}
\usepackage[normalem]{ulem}

\usepackage{graphicx}
\usepackage{subfigure}

\usepackage{color}
\usepackage{pdflscape}
\usepackage{pifont}

\usepackage{bibentry}
\nobibliography*

\renewcommand{\topfraction}{0.85}
\renewcommand{\textfraction}{0.1}
\renewcommand{\floatpagefraction}{0.75}

\newcommand{\vect}[1]{\ensuremath\boldsymbol{#1}}
\newcommand{\tensor}[1]{\underline{\vect{#1}}}
\newcommand{\del}{\triangle}
\newcommand{\grad}{\nabla}
\newcommand{\curl}{\grad \times}
\renewcommand{\div}{\grad \cdot}
\newcommand{\ip}[1]{\left\langle #1 \right\rangle}
\newcommand{\eip}[1]{a\left( #1 \right)}
\newcommand{\td}[2]{\frac{d#1}{d#2}}
\newcommand{\pd}[2]{\frac{\partial#1}{\partial#2}}
\newcommand{\pdd}[2]{\frac{\partial^2#1}{\partial#2^2}}

\newcommand{\circone}{\ding{192}}
\newcommand{\circtwo}{\ding{193}}
\newcommand{\circthree}{\ding{194}}
\newcommand{\circfour}{\ding{195}}
\newcommand{\circfive}{\ding{196}}

\newcommand{\Reyn}{\rm Re}

\newcommand{\bs}[1]{\boldsymbol{#1}}
\DeclareMathOperator{\diag}{diag}

\newcommand{\equaldef}{\stackrel{\mathrm{def}}{=}}

\newcommand{\tablab}[1]{\label{tab:#1}}
\newcommand{\tabref}[1]{Table~\ref{tab:#1}}

\newcommand{\theolab}[1]{\label{theo:#1}}
\newcommand{\theoref}[1]{\ref{theo:#1}}
\newcommand{\eqnlab}[1]{\label{eq:#1}}
\newcommand{\eqnref}[1]{\eqref{eq:#1}}
\newcommand{\seclab}[1]{\label{sec:#1}}
\newcommand{\secref}[1]{\ref{sec:#1}}
\newcommand{\lemlab}[1]{\label{lem:#1}}
\newcommand{\lemref}[1]{\ref{lem:#1}}

\newcommand{\mb}[1]{\mathbf{#1}}
\newcommand{\mbb}[1]{\mathbb{#1}}
\newcommand{\mc}[1]{\mathcal{#1}}
\newcommand{\nor}[1]{\left\| #1 \right\|}
\newcommand{\snor}[1]{\left| #1 \right|}
\newcommand{\LRp}[1]{\left( #1 \right)}
\newcommand{\LRs}[1]{\left[ #1 \right]}
\newcommand{\LRa}[1]{\left\langle #1 \right\rangle}
\newcommand{\LRb}[1]{\left| #1 \right|}
\newcommand{\LRc}[1]{\left\{ #1 \right\}}

\newcommand{\Grad} {\ensuremath{\nabla}}
\newcommand{\Div} {\ensuremath{\nabla\cdot}}
\newcommand{\Nel} {\ensuremath{{N^\text{el}}}}
\newcommand{\jump}[1] {\ensuremath{\LRs{\![#1]\!}}}
\newcommand{\uh}{\widehat{u}}
\newcommand{\Bh}{\widehat{B}}
\newcommand{\fnh}{\widehat{f}_n}
\renewcommand{\L}{L^2\LRp{\Omega}}
\newcommand{\pO}{\partial\Omega}
\newcommand{\Gh}{\Gamma_h}
\newcommand{\Gm}{\Gamma_{-}}
\newcommand{\Gp}{\Gamma_{+}}
\newcommand{\Go}{\Gamma_0}
\newcommand{\Oh}{\Omega_h}

\newcommand{\eval}[2][\right]{\relax
  \ifx#1\right\relax \left.\fi#2#1\rvert}

\def\etal{{\it et al.~}}


\def\arr#1#2#3#4{\left[
\begin{array}{cc}
#1 & #2\\
#3 & #4\\
\end{array}
\right]}
\def\vecttwo#1#2{\left[
\begin{array}{c}
#1\\
#2\\
\end{array}
\right]}
\def\vectthree#1#2#3{\left[
\begin{array}{c}
#1\\
#2\\
#3\\
\end{array}
\right]}
\def\vectfour#1#2#3#4{\left[
\begin{array}{c}
#1\\
#2\\
#3\\
#4\\
\end{array}
\right]}

\newcommand{\G} {\Gamma}
\newcommand{\Gin} {\Gamma_{in}}
\newcommand{\Gout} {\Gamma_{out}}


\title{Robustness for transient problems}
\begin{document}
\maketitle

Assume that boundary conditions are applied on the boundary $\Gamma_0\subset \Gamma$.  Recall that, for the ultra-weak variational formulation
\[
b\LRp{\LRp{u,\uh},v} = \LRp{u,A^*_h v}_{\L} + \LRa{\uh, \jump{v}}_{\Gh\setminus \Gamma_0}
\]
we can recover
\[
\nor{u}_{\L}^2 = b(u,v^*)
\]
for conforming $v^*$ satisfying the adjoint equation
\begin{align*}
A^* v^* &= u\\
v^* &= 0 \text{ on } \Gh\setminus\Gamma_0.
\end{align*}
Together, these give necessary conditions on the test norm $\nor{\cdot}_V$ such that we have $L^2$ robustness (this gives robustness in the variable $u$; for the first order formulation, conditions for $\sigma$ must also be shown).  
\[
\nor{u}_{\L}^2 = b(u,v^*) \leq \frac{b(u,v^*)}{\nor{v^*}_V} \nor{v^*}_V \leq \nor{u}_E \nor{v^*}_V
\]
Thus, showing $\nor{v^*}_V \lesssim \nor{u}_{\L}$ gives the result that $\nor{u}_{\L} \lesssim \nor{u}_E$.  


\section{Reaction-diffusion}

Consider reaction diffusion
\begin{align*}
\pd{u}{t} + u - \epsilon \Delta u &= f\\
u &= 0 \text{ on } \Gamma_1\\
\pd{u}{n} &= 0 \text{ on } \Gamma_2\\
u(t=0) &= u_0.
\end{align*}
The adjoint equation satisfies
\begin{align*}
-\pd{v}{t} + v - \epsilon \Delta v &= u\\
v &= 0 \text{ on } \Gamma_1\\
\pd{v}{n} &= 0 \text{ on } \Gamma_2\\
v(t=T) &= 0.
\end{align*}
(The boundary conditions can be derived by taking the ultra-weak formulation and choosing boundary conditions such that the temporal flux and spatial flux terms $\LRa{\uh, \jump{\tau_n}}_{\Gamma_1}$ and $\LRa{\fnh,\jump{v}}_{\Gamma_2}$ are zero.)

We can then derive that the test norm
\[
\nor{v}_V^2 = \nor{\pd{v}{t}}^2 + \nor{v}^2 + \epsilon\nor{\grad v}^2 
\]
provides the necessary bound $\nor{v^*}_V \lesssim \nor{u}_{\L}$.

To see, this we multiply the adjoint equation by two terms as follows:
\begin{enumerate}
\item Multiply by $v$ and integrate over $\Omega \times [0,T] = Q$ to get
\[
-\int_Q \pd{v}{t}v + \int_Q v^2 + \epsilon \int_Q \LRb{\grad v}^2 - \epsilon \int_0^T\int_{\Gamma} \pd{v}{n}v = \int_Q uv.
\]
Noting that either $v = 0$ or $\pd{v}{n} = 0$ on the boundary removes the integral over $\Gamma$.  Next, we can factor the first term and use Young's inequality to get
\[
-\int_0^T  \pd{}{t}\int_{\Omega} v^2 + \nor{v}^2_Q + \epsilon \nor{\grad v}^2_Q \leq \frac{1}{2}\nor{u}^2_Q + \frac{1}{2}\nor{v}^2_Q
\]
Integrating by parts the first term gives
\[
-\left.\int_{\Omega} v^2\right|_0^T + \frac{1}{2}\nor{v}^2_Q + \epsilon \nor{\grad v}^2_Q \leq \frac{1}{2}\nor{u}^2_Q
\]
Using boundary condition $v=0$ at $t= T$ gives
\[
\frac{1}{2}\nor{v}^2_Q + \epsilon \nor{\grad v}^2_Q \leq \int_{\Omega} v(t=0)^2 + \frac{1}{2}\nor{v}^2_Q + \epsilon \nor{\grad v}^2_Q \leq \frac{1}{2}\nor{u}^2_Q.
\]

\item Multiply by $-\pd{v}{t}$ and integrate over $Q$.  Young's inequality changes the right hand side to 
\[
\int_Q\pd{v}{t}^2 - \int_Q v\pd{v}{t} + \epsilon\int_Q \Delta v \pd{v}{t} = \int_Q -u \pd{v}{t} \leq \frac{1}{2}\nor{u}_Q^2 + \frac{1}{2}\nor{\pd{v}{t}}_Q^2 .
\]
The term $\int_Q v\pd{v}{t}$ can be reduced to the positive contribution $\int_{\Omega}{v(t=0)}^2  $ as above.  We can then take the Laplacian term, integrate by parts in space to get
\[
\int_Q \Delta v \pd{v}{t} = \int_0^T \int_{\Omega} \Delta v \pd{v}{t} =  \int_0^T \int_{\Gamma} \pd{v}{t}\pd{v}{n} - \int_0^T \int_{\Omega}\grad\LRp{\pd{v}{t}}\grad v.
\]
Since either $v = 0$ or $\pd{v}{n} = 0$ on $\Gamma$, the first term disappears.  The second term can be bounded by noting
\[
- \int_0^T \int_{\Omega}\grad\LRp{\pd{v}{t}}\grad v = - \int_0^T \pd{}{t}\int_{\Omega}\LRb{\grad v} ^2 = - \left.\int_{\Omega}\LRb{\grad v}^2 \right|_0^T.
\]
Since $v = 0$ at $t=T$, $\grad v = 0$ at $t=T$ as well, and we are left with the positive contribution $\int_{\Omega}\LRb{\grad v(t=0)}^2$.  Then,
\[
\frac{1}{2}\nor{\pd{v}{t}}_Q^2 \leq \frac{1}{2}\nor{u}_Q.
\]
\end{enumerate}
Together, these two show that, under test norm
\[
\nor{v}_V^2 = \nor{\pd{v}{t}}^2 + \nor{v}^2 + \epsilon\nor{\grad v}^2,
\]
the adjoint equation $v^*$ satisfies
\[
\nor{v^*}_V \lesssim \nor{u}_{\L}
\]
and thus the DPG energy norm robustly bounds the $L^2$ norm from above
\[
\nor{u}_{\L} \lesssim \nor{u}_E.
\]

\section{Convection-diffusion}

Truman, your turn :).

\section{Robustness for transient problems given spatial robustness}

Suppose we have the transient problem
\[
\pd{u}{t} + Au = f
\]
with initial condition $u(x,0) = u_0$.  Suppose that DPG is robust under the ultra-weak variational formulation for the steady problem
\[
\LRp{u,A^*_h v}_{\L} + \LRa{\uh, \jump{v}}_{\Gamma_h\setminus \Gamma_0} = \LRp{f,v}
\]
with test norm $\nor{v}_{V}$.  Then, can we show that 
\[
\nor{v}_{V,t} \coloneqq \nor{v}_V + \nor{\pd{v}{t}}_{\L}
\]
also leads to a robust upper bound of the $L^2$ norm by the DPG energy norm?  I believe this may be possible.  The adjoint equation for robustness for the transient problem gives
\[
-\pd{v}{t} + A^*v = u
\]
with $v = 0$ at $t=T$...  


\end{document}