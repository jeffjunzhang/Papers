\documentclass[11pt,c]{article}
\usepackage[margin=1.0in]{geometry}
\usepackage{amsmath}
\usepackage{amsfonts}
\usepackage{amsthm}
\usepackage{amssymb}
\usepackage{amsbsy}

\theoremstyle{remark}
\newtheorem*{remark}{Remark}

\newcommand{\bfsigma}{\boldsymbol\sigma}

\begin{document}
\baselineskip=16pt
\parskip= 4pt
%*************************************************page 1


\begin{center}
{\large \bf Reply to the Reviewers and the Editor}
\end{center}

We are very grateful to the reviewer for reading our paper and the critiqe. The reviewers
questions along with our answers are listed below. All changes in the revised version
of the paper are in red.




{\em 
I read the paper by T. Ellis, J. Chan, and L. Demkowicz. I think the paper addresses
an interesting issue, and the results are interesting. On the other hand, I believe
that the paper has a few rought spots that need clarification/correction, and also that
it should have a written in a wider sense than the one it is currently written, especially
considering that it is aimed at appearing in a survey volume like the one you are editing.
Then, I detail now what are my main suggestions for this paper:}

\begin{enumerate}

  \item {\em Make, possibly in the concluding remarks, some comments on the relationship this
work has with HdG, and also with multiscale FEM. I definitely think there is a connection
between the approaches, and then I would encourage the authors to make this connection
explicit (for example, the propeties of broken test functions in S 2.2.4 are very similar
to the properties of Multiscale basis functions)}.

A discussion of how DPG relates to multiscale methods and HDG has been added to the introduction.




  \item {\em In the problem presentation, is the convective field $\beta$ a time-dependent function?
If so, there is clearly an overlap between the distinct components of the boundary of Q.
This might have an impact in the boundary conditions chosen.}



We are not sure we understand the question. The boundary of the space-time domain 
$Q := \Omega \times (0,T)$ consists of three parts: $\Omega \times {0}$,
$\Omega \times {T}$ and the lateral boundary $\partial \Omega \times (0,T)$.
The lateral boundary is partitioned into two parts dependent upon the advection 
vector that may vary in time,
$$
\begin{array}{rl}
\Gamma_{in} := & \{ (x,t) \in \partial \Omega \times (0,T) \: : \:
\beta(x,t) \cdot n(x) < 0 \} \\
\Gamma_{out} := & \{ (x,t) \in \partial \Omega \times (0,T) \: : \:
\beta(x,t) \cdot n(x) \geq  0 \} 
\end{array}
$$
Clearly, one and only one of the defining conditions is satisfied which means
that the lateral boundary is {\em partitioned} into the two sets.





  \item {\em In equation (3), I believe there is a boundary term missing (this term is then present
in (4), containing a numerical flux).}

The are three variational formulations. In the first one (equation (3)), test functions
satisfy boundary conditions present in the definition of the $L^2$-adjoint and, as a consequence,
no boundary terms are present. Removing boundary conditions from test functions in the
second formulation (equation (4)) results in the introduction of additional unknowns - the
traces and fluxes that live on the domain boundary. Finally, the ultimate variational
formulation with broken test spaces (equation (5)) uses traces and fluxes on {\em the whole
mesh skeleton} including the domain boundary. The intermediate formulation (4) is supposed
to help reader to transition to the final formulation (5).


  \item {\em I think the first paragraph in page 6 should be in the introduction. It states the main
idea of the paper, so....}

We decided to maintain this paragraph in its current place, but added the following extra
comment to the introduction.


In the sense, the main challenge is to come up with a correct test norm. The residual is measured
in the dual test norm, and the DPG method minimizes the residual. The residual can be interpreted
as a special {\em energy norm}. In other words, the DPG method delivers an orthogonal
projection in the energy norm. The task is especially challenging for singular perturbation
problems. Given a trial norm, we strive to determine a quasi-optimal test norm such that
the corresponding energy norm is robustly equivalent to the trial  norm of choice. An additional
difficulty comes from the fact that the optimal test functions should be easily approximated
with a simple enrichment strategy. For convection dominated diffusion, this means that
the test functions should not develop boundary layers. The task of determining the
quasi optimal test norm (we call it a {\em robust test norm} leads then to a stability analysis
for the adjoint equation which is the subject of this paper. For a more general discussion
on the subject, see \cite{DPGEncyclopedia}.




  \item {\em Also in page 6, could the authors give some extra motivation on the inclusion of the
$L^2$ norm of u in the norm? It doesn't seem to come directly from the formulation. Also,
am I correct in supposing that, unless this term is included in the norm, the local problems
appearing by using the broken norms given in (7) and (8) would not be well-posed?}

The following remark has been added to the text.

\begin{remark}
In the DPG technology, the test norm must be {\em localizable}, i.e.,
$$
\Vert v \Vert^2_{V} = \sum_{K} \Vert v \Vert^2_{V(K)}
$$
where $\Vert v \Vert_{V(K)}$ denotes a test norm (and not just a seminorm) for the
element test space. In practice this means the addition of properly scaled $L^2$-terms.
Without those terms, we could not invert the Riesz operator on the element level.
Addition of the $L^2$ terms does not necessarily contradict the robustness of the norm,
see the discussion in \cite{DPGEncyclopedia} on bounded below operators. An alternate
strategy has been explored in \cite{EllisLC} where we enforce element conservation
property 
by securing the presence of a constant function in the element test space. The residual
is then minimized only over the orthogonal complement to the constants which eliminates
the need for adding the $L^2$-term to the test norm.
\end{remark}


  \item {\em Could the authors include some more comments in their proofs? They seem to be correct,
and they follow somehow familiar arguments, but still I believe some more details should
be given on how we go from one point to the next one. Also, some comments on how to
link one result to the following one would be welcome, just to make the paper easier
to read.}

We have added a number of additional comments in the derivations to make the logical
flow easier (all in red).

  \item {\em At the end of Section 3 the authors could give some comments on what is the significance
of the previous results. In particular, do they lead, naturally, to an error estimate of the method?}

A couple lines have been  added at the end of Section 3 to summarize.

In conclusion, with either robust test norm, we can claim the following stability result,
$$
\begin{array}{ll}
\Vert u - u_h \Vert & \lesssim \Vert (u,\bfsigma,\hat{u},\hat{t}) - (u_h,\bfsigma_h,\hat{u}_h,\hat{t}_h) \Vert_E \\[8pt]
& = \inf_{(u_h,\bfsigma_h,\hat{u}_h,\hat{t}_h)} \Vert (u,\bfsigma,\hat{u},\hat{t}) - (u_h,\bfsigma_h,\hat{u}_h,\hat{t}_h) \Vert_E \, .
\end{array}
$$
Notice that, contrary to the steady-state case, we have not been able to secure a robust $L^2$ bound
for the stress. The best approximation error in the energy norm can be estimated locally, i.e.
element-wise, see \cite{DemkowiczHeuer,ChanHeuerThanhDemkowicz2012}. This leads to an ultimate, final $h$ estimate but
not necessarily with robust constants. The loss of robustness in the best approximation error
estimate is the consequence of rescaling the $L^2$-terms to avoid boundary layers in the optimal
test functions. However, similarly to the steady-state case, with refinements, the mesh-dependent
$L^2$-terms converge to the optimal ones so we hope to regain robustness in the limit. 
We do not attempt to analyze the best approximation error in this contribution and restrict
ourselves to numerical experiments only.








  \item {\em The legends in the plots are barely readable. Can the authors please make them clearer?
Also, it could be interesting to see an elevation of a numerical solution containing a boundary,
or inner, layer.}

Fixed.

  \item{\em Finally, there are several typos and inconsistencies in the references. Please unify the format,
and order them alphabetically.}

Fixed.

\end{enumerate}


\bibliographystyle{plain} 
\bibliography{../Papers}

\end{document}